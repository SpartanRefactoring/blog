As a whole, the stackoverlow\urlref{http://stackoverlow.com} community 
recommedns against asking coding style and coding convetions related questions. 
Tag \cc{code-styling}, and its many synonyms, \cc{code-convention}, 
\cc{coding-guidelines} are specifically 
earmarked\urlref{http://stackoverflow.com/tags/coding-style/info} as 
\begin{quote}
DO NOT USE! This tag refers to an entirely opinionated subject and is therefore 
no longer on-topic. Refactoring, braces, indentation, Hungarian notation, and 
other stylistic issues relating to code. 
\end{quote}
But it is not only curly braces matters that may raise 
this kind of ``religious'' wars. Even the famous convention maxim ``\emph{gotos 
are evil}''~\cite{Dijksta:must be in bib} was challenged in the scientific 
literature~\cite{Knuth: and there are several others}.

Telling the story of ``Spartan Programming'', one particular sytle guideline for 
\Java, this manuscript is also likely to be contentious, even more so, since 
``spartanized'' code such as 
\begin{code}{JAVA}
  public static <T, C extends Collection<T>> 
                                accumulate<T, C> to(final C c) {
    return new accumulate<T, C>() {
      @Override public accumulate<T, C> add(final @Nullable T t) {
        if (t == null)
          return this;
        assert t != null;
        c.add(t);
        return this;
      }
      @Override public C elements() {
        return c;
      }
    };
  }
  \end{code}

\begin{code}{JAVA}

default accumulate<T, C> addAll(final Iterable<? extends T>... tss) {
  for (final Iterable<? extends T> ts : tss)
    addAll(ts);
  return this;
}
 
\end{code}

  
may look bizzare to eyes accustomed to traditional \Java code.


This manuscript reports 

in 

\cc{code-styling}
\Java
Software engineers tend to engage  on religious wars on styles are

Examples from stakoverflow, or even usenet... 

Stackoverflow\footnote{http://stackoverflow.com/} reports many discussions on style and code conventions. 
However, it is worth to not how it is suggested not to use \kk{code-styling} (and its synonims: \kk{code-style}, 
\kk{code-convetions} and the like) as tag, because it is claimed that 
``This tag refers to an \emph{entirely opinionated subject} and is therefore no longer on-topic''(\emph{sic}, 
emphasis is ours). 
This seems to suggest that they want to prevent \emph{ex ante} some of these religious wars to take place.
\kk{code formatting} is the tag used to 

A famous example is that of curly brackets around a single statement making
the body of control flow commands, such as \kk{if} and 
\kk{while}.
Mention that \Go forces you to use them!
Mention that are compulsory in \kk{synchronized}, 
\kk{try},  
\kk{catch},  
\kk{finally},  

but on the other hand, in a recent addition lambda expressions\cite{lambda}
the options are maintained.
You can write
\begin{JAVA}
()-> 3
\end{JAVA}
and
\begin{JAVA}
  ()-> {3}
\end{JAVA}
check this.

``\emph{the braces war}''


Why should this be important?
\begin{itemize}
    \item Easy to write in  your own style, but how can 
      you always adjust to the style of others.
      Waste of neurons.
    \item Code comparison. Which part of my change is semantical and which 
      part of this my own personal style.
    \item Code analysis tools should adjust to different styles and many
        different ways for doing the same thing.
    \item METRICS: IN a recent paper Gal and I showed that size 
      is the only thing that matters.
    \item METRICS: IN a recent paper Gal and I showed that the notion of 
      "size" is very different in different projects.
\end{itemize}

The holly grail of "canonical representation of code" .
Does LOC mean anything?
Not if you apply different indentation rules.
A better metric is NOT = number of tokens.
But even this is influenced by coding styles, including but 
not limited to the famous braces war.

\textbf{Mention naturlaization (MAT: moved here from the beginning)}

Some recent studies investigated the presence of some regularities in code that 
make source code repetitive and predictable as well as natural language.
In Hindle et al. \cite{Hindle:Bar:Su:Gabel:Devanbu:2012} analyzed a large corpora composed
by ... of software systems and compared the properties with natural language.
They found out that source code present many statistical properties that 

\subsection{Background}
There is very little scientific work on style.
Mention the Latin quote.

Mention the style guides on C++ and Java. Oracle.

But where these styles a matter of a careful and raional design process?
What were the underlying principles.

Last paragraph: many people believe that style is irrelevant and unimportant. 
We believe that it is.

\subsection{Spartan programming}
First occurred in a stupid TOOLS USA 199X. tutorial.
The point of making and perfecting it was to "be rational" and 
reexamine all assumptions. 
Rules have changed.
Utter despise of backward compatability the need to rewrite the code.

Mention it.  The spartanization tool.

Say that some of your alma mater gang love to do this kind of metrics, but 
no one really knows whether the results are applicable to different projects
since 
\cite{Turnu:Concas:Marchesi:Tonelli:11}


the "Idionsyncratic Projects Hypothesis" I also had a plateau paper with 
Dany and Maayan about this. A number o years ago.


This paper gives an overview of spartanization, future work will be
experiments.

\subsection{Contribution}

\textbf{(the following text has been adapted by the wiki)}\\

Spartan programming is an approach to writing code aimed at improving at
reducing code complexity by appling many techininques, some already known in
literature, some others more novel. It was introduced in the mid on nienties
and it has been a part of some teachings.  Its rationale is based on some
principles, but at the end of the day it is a techninque to pursue an extreme
minimalism code writing.

One of these principles is the "Babylonian Tower Principle": It states that the
number of abstraction levels of a software systems is limited.  A famous study
by Miller, also proved that the cognitive burden people might bear, namely the
number of cognitive items that can be processed in human being working memory,
is limited to 7 (plus or minus two) elements. By simplifying the basic element
of more complex, structured composite component, spartan programming reduce the
developers' cognitive burden.

However, being in a certain sense, counter intuitive, in order to apply
Spartanization on their working routines, the developers need to respect a firm
discipline.  From a certain standpoint, Spartan Programming might be seen as a
list of code style presciption, but it would just scratch the surface.  It is
far more than that, since it 

Being minialized, spartanized code might be less readable that the original.
This is ...  Spartan programming is not directly concerned with readability, at
least not in its subjective and cultural-dependent sense. In fact, spartan
programs will bring much misery to anyone preferring long, verbose programs.

Spartanization aims at minimizing Vertical complexity (the number of lines of code), 
Horizontal complexity, the number of tokens, character, parameters and variables, 
as well as reduce the number of loops and conditional statements.
Spartan programming prescribes, a careful use of the variables, stressing the need 
to inline variables that are used only once and emphasizing the use of foreach loops 
in languages implementing it. Variables visibility and accessibility should also be 
possibly minimized, favouring the use of private over public variables whenever it 
is possible. 

Also the names of variables should be taken short for local variables, with short scope, 
leaving to their class names to explicit their behaviour. Minimizing variability using 
final and @NonNull annotation. Stress the use of generic names technique. Minimize lifespan 
avoiding persistent variables when it is possible. Favour the use of collection instead 
of arrays. Interfaces should present a minimal number of parameters, also minimizing 
the interaction between them. Favour ternarization over the use of if-then-else constructs 
and use a simplified version of the conditionals that implies an early return. Moreover, 
use an early exit logic with continue, break and return instructions. 

Everything should be formatted in order to make an efficient use of the screen space, 
by removing unnecessary tokens (i.e. redundant parentheses, etc.). 
Methods should be as short as possible. 


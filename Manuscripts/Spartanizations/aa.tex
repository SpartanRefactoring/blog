As a whole, the Stackoverlow\urlref{http://stackoverlow.com} community
recommends against asking coding style and coding conventions related questions.
Tag \cc{code-styling} and its many synonyms: \cc{code-convention},
\cc{coding-guidelines} are specifically
earmarked\urlref{http://stackoverflow.com/tags/coding-style/info} as

\begin{tcolorbox}[colback=green!5!white,colframe=blue!25!white,notitle]
    \textsc{Do Not Use!}
This tag refers to an entirely opinionated subject and is therefore
no longer on-topic. Refactoring, braces, indentation, Hungarian notation, and
other stylistic issues relating to code.
\end{tcolorbox}

As pointed out by this statement, \emph{‟curly brackets wars”}, namely discussions
on weather, when and how using curly brackets (and the like), are somehow discouraged.
However, such discussions can not be simply relegate to something about personal
tastes.
As a matter of fact, some of this options are included also in the syntax definition of some languages.
For example, in \Go forces you to use curly brackets \urlref{https://golang.org/doc/faq#semicolons}.
Another example might be that of some Java statement blocks, such as \kk{synchronized}, \kk{try}, \kk{catch},
\kk{finally}, for which the curly brackets are compulsory.
On the other hand, they can be ignored, like in a recent addition to
the lambda expressions \urlref{http://docs.oracle.com/javase/tutorial/java/javaOO/lambdaexpressions.html#syntax}
in Java, so you car write both the following versions of the same code: \texttt{()->3} or \texttt{() ->\{3\}}.
\footnote{\emph{``Where we put the curly braces in code such as Listing 5 will likely dominate Java message boards and blogs for years to come''}(sic).
http://www.oracle.com/technetwork/articles/java/architect-lambdas-part1-2080972.html}.
But it is not only curly braces matters that may raise
this kind of ‟religious” wars. Even the Dijksta's maxim ‟\emph{gotos are evil}”~\cite{Dijkstra:68} 
was challenged in the scientific literature~\cite{Knuth:74,Ramshaw:88,Bochmann:73,Sennesh:Gil:16,Zoethout:79,Wulf:79,Clark:84}.

It seems that both practitioners and researchers consider code conventions, code style
something unrelated to other most sensitive issues, such as the design and productivity.
We think that using an appropriate code convention might have a meaningful impact on the
design of code and also on its quality.

% Why should this be important?
The following scenarios might be helpful to clarify this point.
\begin{itemize}
    \item Developers would certainly find easy to write code in their own style. However, software
    development is carried out in a collaborative environment and developers works together on
    the same files. Adjusting the code to others style require a certain effort and it is time consuming.
    % Waste of neurons.
    \item A common task performed during software development is code comparison.
    Since part of the changes might be due to stylistic reasons, it is mandatory to distinguish the
    semantically sensitive changes from the stylistic ones.
% Which part of my change is semantically and which
% part of this my own personal style.
    \item Also code analysis tools should adjust to different styles, since there are many
        different ways for doing the same thing.
    \item Code conventions might affect several metrics. The number of lines of code (LOC) or the number of
    tokes (NOT) are influenced by coding styles. In a recent one of the author demonstrated that size
      is the only thing that matters~\cite{Gil:Lalouche:16}.
    \item Moreover, the same notion of size has been shown to be very different in different projects~\cite{Gil:Lalouche:16}.
\end{itemize}

\textbf{The holy grail of ``canonical representation of code''.  
  Does LOC mean anything?  Not if you apply different indentation 
rules.}

This manuscript present ‟Spartan Programming”, a programming style based on
many techniques, including some already reported in the literature and novel
ones and combines them in a systematic way to reduce simultaneously many
different measures of complexity. In other words, a Spartan programmer strive
to achieve an extreme minimalism and simplicity. Even if it can be seen as one
of the many code styles available, Spartan Programming differs from all of them
because of its unifying principle of minimalism and simplicity.
% This manuscript reports in

\subsection{Background}

Despite the fact that coding style is a topic that might lead developers to
engage heated discussions, to the best of the author's knowledge, there is
actually very little scientific work on that.

If one queries for ‟code conventions” †{The search was
performed on August 2016 on text, abstract and keywords and
restraining the outcomes to the ‟Computer Science” disciplinary area} on
Scopus\urlref{http://scopus.org}, one of the most important bibliographic
databases, what it comes out is around thousands of results. However, very few of them have \kk{code
conventions} among they keywords and deal with code conventions

This alleged \emph{nonchalance} shown by the scientific community towards the
investigation of a rationale approach to the problem of defining effective code
conventions, might be related to the fact that when it comes to code styling
most of the dialectic efforts are devoted
to make each personal preferences prevail against the others.

This tendency to transform technical discussions on disputes about personal preferences
in matter of code readability, maintainability (not to mention the developers' efficiency) might
have led Stackoverflow managers to the decision reported in the Introduction, embracing the
wiseness of the ancient latins who were used to say that \emph{De gustibus non est disputandum}.

Notwithstanding what reported above, code conventions are considered a worthwhile issue during
the development process, otherwise we would not find style guides for the most important
programming languages like (Java, C++, etc.) and promoted by the major players in
the ITC arena worldwide, like Oracle and Google~\cite{}.

\textbf{But where these styles a matter of a careful and rational design process?
What were the underlying principles.}

Last paragraph: many people believe that style is irrelevant and unimportant.
We believe that it is.

\subsection{Spartan programming}

\begin{quote}
  ‟Let us change our traditional attitude to the construction of programs:
  Instead of imagining that our main task is to instruct a computer what to do,
  let us concentrate rather on explaining to human beings what we want a computer
  to do…”
  \begin{flushright}
   \upshape Donald E Knuth. ‟Literate Programming (1984)”
  \end{flushright}
\end{quote}

% The coding guidelines began with a dozen or so printed pages,
% "a little book of style", which the author handed out to computer science students
% in the Hebrew university of Jerusalem. The term "Spartan Programming" was coined in 1996,
% when the author gave a tutorial on "Spartan C++" at the TOOLS (Techniques of Object Oriented
% Languages and Systems) scientific conference, held in Santa Barbara, CA USA. The guidelines
% were taught under this name in numerous Technion courses since then.

The definition of ‟Spartan Programming” first occurred in a tutorial taken at
the TOOL (Techniques of Object Oriented Languages and Systems) scientific conference,
held in Santa Barbara, CA USA. % in in a stupid TOOLS USA 199X. tutorial.
The aim was to present a rational approach to programming by examining all the previous
assumptions about how to code.

\emph{point of making and perfecting it was to "be rational" and
reexamine all assumptions.}

\emph{Rules have changed.
Utter despise of backward compatibility the need to rewrite the code.
Mention it.}

% The spartanization tool.

Spartanization principles were implemented in a tool automatically apply code transorfation
to \emph{spartanize} the code, namely transformation of codes according to the spartan rules, has been developed.
This tools works for Java code and it is implemented as a plugin for the Eclipse\urlref{https://eclipse.org/}
IDE\@. It is available through the Eclipse marketplace\urlref{https://marketplace.eclipse.org/}

Say that some of your alma mater gang love to do this kind of metrics, but
no one really knows whether the results are applicable to different projects
since~\cite{Turnu:Concas:Marchesi:Tonelli:11}

the ``Idiosyncratic Projects Hypothesis'' I also had a plateau paper with
Dany and Maayan about this. A number o years ago.\cite{Gil:2011:Goldstein:Moshkovich:2011}

\subsection{Contribution}

The present manuscript makes the following contributions:

This manuscript present ‟Spartan Programming”, a programming style based on many techniques,
including some already reported in the literature and novel ones and combines them
in a systematic way to reduce simultaneusly many different measures of complexity.
In other words, a Spartan programmer strive to achieve an extreme minimalism and simplicity.
Even if it can be seen as one of the many code styles available, Spartan Programming differs from
all of them because of its unifying principle of minimalism and simplicity.

Telling the story of ‟Spartan Programming”, one particular style guideline for
\Java, this manuscript is also likely to be contentious. Perhaps even more so
after examining some ‟spartanized” code such as in \cref{figure:first}
may look bizarre to eyes accustomed to traditional \Java code.

% \textbf{(the following text has been adapted by the wiki)}⏎

However, being in a certain sense, counter intuitive, in order to apply
Spartanization on their working routines, the developers need to respect a
firm discipline. From a certain standpoint, Spartan Programming might be
seen as a list of code style prescription, but it would just scratch the
surface. It is far more than that, since it…  

This paper gives an overview of spartanization, future work will be
experiments.


\subsection{Background}


\subsection{Contribution}


Spartan programming is an approach to writing code aimed at improving at
reducing code complexity by appling many techininques, some already known in
literature, some others more novel. It was introduced in the mid on nienties
and it has been a part of some teachings.  Its rationale is based on some
principles, but at the end of the day it is a techninque to pursue an extreme
minimalism code writing.

One of these principles is the "Babylonian Tower Principle": It states that the
number of abstraction levels of a software systems is limited.  A famous study
by Miller, also proved that the cognitive burden people might bear, namely the
number of cognitive items that can be processed in human being working memory,
is limited to 7 (plus or minus two) elements. By simplifying the basic element
of more complex, structured composite component, spartan programming reduce the
developers' cognitive burden.

However, being in a certain sense, counter intuitive, in order to apply
Spartanization on their working routines, the developers need to respect a firm
discipline.  From a certain standpoint, Spartan Programming might be seen as a
list of code style presciption, but it would just scratch the surface.  It is
far more than that, since it 

Being minialized, spartanized code might be less readable that the original.
This is ...  Spartan programming is not directly concerned with readability, at
least not in its subjective and cultural-dependent sense. In fact, spartan
programs will bring much misery to anyone preferring long, verbose programs.
Spartanization aims at minimizing:

DO NOT REPEAT THE TABLES IN GOOGLE DOCS.
Just say: 26 projects mention summary results only.
Check out all sheets. I think only bzip2 makes sense.⏎

Cite naturalness of software.

Preliminary results supporting the
"Hypothesis that can never be proved"
Because this is not what you like.⏎

\hrule

In \cref{table:original} are reported the results obtained by applying Jack to
our dataset. After removing the unessential code and obtaining boilerplate
code we computed compression ratio. We also applied the Burroughs-Wheeler
algorithm to the outcome of the Jack algorithm. As we can see if we compare
the results reported in \cref{table:original} and \cref{table:original} the
compression ratio is higher if the BW algorithm is applied to the boilerplate
code.

\begin{table}
  \label{table:original}
  \caption{The results of the compression using the Jack algorithm alone and both Jack and
  Burroughs-Wheeler algorithms combined}
  \centering
  \scalebox{0.75}{
    \begin{tabular}{l*5r}
      \toprule
              & \multicolumn{2}{c}{\textit{Jack}} & \multicolumn{2}{c}{\textit{Jack + BZip2}} & \textit{Combined-}\\
               \cmidrule(r){2-3} \cmidrule(r){4-5}
              & \textit{Size (bytes)} & \textit{Ratio} & \textit{Size (bytes)} & \textit{Ratio} & \textit{-ratio}\\
      \midrule
      Average & 5,078,034            & 2.78                 & 729,934                   & 6.60           & 18.37\\
      Minimum & 456,050             & 1.73                 & 77,706                    & 4.89           & 10.18\\
      Maximum & 19,697,634           & 12.85                & 3,414,889                  & 8.68           & 83.65\\
      Range   & 19,241,584           & 11.12                & 3,337,183                  & 3.79           & 73.47\\
      Median  & 1,822,510            & 2.34                 & 343,815                   & 6.50           & 15.95\\
      \bottomrule
    \end{tabular}}
\end{table}

We than applied a set of spartanizations (partial spartanization) to all the
software systems. Afterwards we applied both the Jack and the BW algorithm.
The results are reported in \cref{table:partial} where it is also
reported the improvement in the compression due to the spartanization. It is
worth to note that we have an improvement in any statistics, in other words the
spartanization affected positively the compression, increasing the compression
ratio. This is counter intuitive, being a small file most difficult to
compress than a larger one (REFORMULATE)

\begin{table}
  \label{table:partial}
  \caption{The results after performing a partial spartanization of the code, compared
  with the compression ratios obtained with the Jack and Burroughs-Wheeler algorithm}
  \centering
  \scalebox{0.75}{
    \begin{tabular}{l*5r}
      \toprule
           & \multicolumn{2}{c}{\textit{Size (bytes)}} & \textit{Ratio} &\multicolumn{2}{c}{\textit{Improvement}} \\
               \cmidrule(r){2-3} \cmidrule(r){5-6}
              & \textit{Jack} & \textit{Jack + BZip2} & & \textit{with Spartan.} & \textit{Total}\\
      \midrule
      Average & 5,050,647  & 719,488  & 6.66 & 0.06  & 1.01\\
      Minimum & 439,825   & 75,391   & 4.90 & -0.37 & 0.21\\
      Maximum & 19,173,224 & 3,275,918 & 8.79 & 0.49  & 1.07\\
      Range   & 18,733,399 & 3,200,527 & 3.89 & 0.86  & 0.87\\
      Median  & 1,959,917  & 341,851  & 6.79 & 0.08  & 1.04\\
      \bottomrule
    \end{tabular}}
\end{table}

\begin{table}
  \label{table:total}
  \caption{The results after performing a total spartanization of the code, compared
  with the compression ratios obtained with the Jack and Burroughs-Wheeler algorithm.}
  \centering
  \scalebox{0.70}{
    \begin{tabular}{l*5r}
      \toprule
                 & \multicolumn{2}{c}{\textit{Size (bytes)}} & \textit{Ratio} &\multicolumn{2}{c}{\textit{Improvement over}} \\
               \cmidrule(r){2-3} \cmidrule(r){5-6}
              & \textit{Jack} & \textit{Jack + BZip2} & & \textit{Overall} & \textit{Partial Spartan.}\\
      \midrule
      Average & 4,947,319  & 713,857  & 6.56 & -0.03 & -0.09⏎
      Minimum & 430,375   & 75,160   & 4.89 & -0.40 & -0.28⏎
      Maximum & 18,890,899 & 3,244,851 & 8.72 & 0.41  & -0.01⏎
      Range   & 18,460,524 & 3,169,691 & 3.83 & 0.81  & 0.27⏎
      Median  & 1,923,056  & 339,930  & 6.70 & 0.00  & -0.09⏎
      \bottomrule
    \end{tabular}}
\end{table}

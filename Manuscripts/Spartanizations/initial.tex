
DO NOT REPEAT THE TABLES IN GOOGLE DOCS.
Just say: 26 projects mention summary results only.
Check out all sheets. I think only bzip2 makes sense.\\

Cite naturalness of software.

Preliminary results supporting the 
"Hypothesis that can never be proved"
Because this is not what you like.\\

------------------------------------

In Table \ref{tab:proj-stat-original} are reported the results obtained by applying Jack to our dataset. 
After removing the unessential code and obtaining boilerplate code we computed compression ratio. 
We also applied the Burroughs-Wheeler algorithm to the outcome of the Jack algorithm.
As we can see if we compare the results reported in Table \ref{tab:proj-stat-original}
and Table \ref{tab:proj-stat-original} the compression ratio is higher if the BW algorithm
is applied to the boilerplate code.

\begin{table*}
\centering
\scalebox{0.8}{
\begin{tabular}{lccccc}
\hline
& \textbf{Jack size} & \textbf{comp. ratio} & \textbf{Jack size BZip2} & \textbf{Ratio} & \textbf{Combined ratio}\\ 
 \hline
Average & 5078034 & 2.78 & 729934 & 6.60 & 18.37\\
Minimum & 456050 & 1.73 & 77706 & 4.89 & 10.18\\
Maximum & 19697634 & 12.85 & 3414889 & 8.68 & 83.65\\
Range & 19241584 & 11.12 & 3337183 & 3.79 & 73.47\\
Median & 1822510 & 2.34 & 343815 & 6.50 & 15.95\\
\end{tabular}}
\label{tab:proj-stat-original}
\caption{The results of the compression using the Jack algorithm alone and both Jack and 
Burroughs-Wheeler algorithms combined.}
\end{table*}

We than applied a set of spartanizations (partial spartanization) to all the software systems.
Afterwards we applied both the Jack and the BW algorithm. The results are reported in Table \ref{tab:proj-stat-partial} where
it is also reported the improvement in the compression due to the spartanization. 
It is worth to note that we have an improvement in any statistics, in other words the spartanization affected positively
the compression, increasing the compression ratio. 
This is counter intuitive, being a small file most difficult to compress than a larger one (REFORMULATE)

\begin{table*}
\centering
\scalebox{0.8}{
\begin{tabular}{lccccc}
\hline
 & \textbf{Jack size} &	\textbf{Jack size BZip2} & \textbf{Ratio} & 
 \textbf{Improvement in Comp. Ratio with Spartanization} & 
 \textbf{Total Improvement} \\
 \hline
Average & 5050647 & 719488 & 6.66 & 0.06 & 1.01\\
Minimum & 439825 & 75391 & 4.90 & -0.37 & 0.21\\
Maximum & 19173224 & 3275918 & 8.79 & 0.49 & 1.07\\
Range & 18733399 & 3200527 & 3.89 & 0.86 & 0.87\\
Median & 1959917 & 341851 & 6.79 & 0.08 & 1.04\\
\end{tabular}}
\label{tab:proj-stat-partial}
\caption{The results after performing a partial spartanization of the code, compared 
with the compression ratios obtained with the Jack and Burroughs-Wheeler algorithm.}
\end{table*}

\begin{table*}
\centering
\scalebox{0.85}{
\begin{tabular}{lccccc}
\hline
 & \textbf{Jack size} &	\textbf{Jack size BZip2} & \textbf{Ratio} & 
 \textbf{Overall Improvement} & 
 \textbf{Improvement over Partial Spartanization} \\
 \hline
Average & 4947319 & 713857 & 6.56 & -0.03 & -0.09
\\
Minimum & 430375 & 75160 & 4.89 & -0.40 & -0.28
\\
Maximum & 18890899 & 3244851 & 8.72 & 0.41 & -0.01 \\
Range & 18460524 & 3169691 & 3.83 & 0.81 & 0.27
\\
Median & 1923056 & 339930 & 6.70 & 0.00 & -0.09 \\
\end{tabular}}
\label{tab:proj-stat-total}
\caption{The results after performing a total spartanization of the code, compared 
with the compression ratios obtained with the Jack and Burroughs-Wheeler algorithm.}
\end{table*}
This paper made a brief introduction the spartan programming style, and gave
plenty of examples of the somewhat different look of spartan programs which is
due to the strive for minimalism. Spartanization positions certain traditional
values concerning brevity at top, insisting on the full extent of their application 
mostly with the help of automatic tools, and at the same time, forsaking other
values and principles that might contradict our central objective of minimalism. 

The rationale was that even if different people rank values differently, the
extremism of spartanization makes it possible to examine the resulting code,
and even scientifically evaluate its merits.

We spent time explaining the two major fragments of spartanization process, and
reported on an Eclipse plugin, which implements most, but not all, of the
wrings in it. An experiment conducted with this plugin showed that at least
structural spartanization contributes to a notion often called regularity,
but likened to naturalness here.

Due to space limitations, one interesting fragment of spartanization, 
method extraction (and inlining), was not demonstrated here. Future 
work is to present the \textbf{M} fragment and explore techniques for 
automating it.

We argued that the combination of spartanization and jacking should give rise
to a more canonical representation of code, in which much of personal
taste, project idiosyncrasies  and boilerplate code and text are eliminated.
It remains to be seen whether the canonical representation is better than
traditional size metrics. An application of this representation might be in
detecting plagiarism. Other products of this presentation is better tools
for code comparison as it occurs in the non-academic, software engineering
environment.
\endinput
 




% Spartan programming tries to imitate the laconic, terse speak.  
% 
% \textbf{However, it is worth to mention that the average value is quite small (0.06) so 
% this result might be influenced by statistical fluctuations.}
% \textbf{As well as reported in the previous section, it is worth not note that 
% these values are quite small.}
% 
% In this paper we presented a novel approach to coding. We introduced the
% problem, illustrated the rationale behind (the philosophy) of the Spartan
% Programming, next we thoroughly described the techniques to apply to spartanize
% the code (or to write it in the spartan way since the beginning) and eventually
% we report the result of an experiment performed on a representative dataset of
% Java software systems.  From the results of this experiment, that show that the
% compression of spartanized code is higher than not spartanized one, we can
% evince that spartanization have a meaningful impact on source code (???).  The
% aim of this paper is mainly that of giving an overview of the spartanization
% techniques.  In the future we intend to perform a number of experiments, in
% order to carry on the research on the impact of the spartanization on software
% systems' properties, including its quality.
% 
% Remind the reader of canonical code.
% 
% Automatic method extraction.

\section{Conclusion}
Spartan programming tries to imitate the laconic, terse speak.  
\label{Conclusion}

In this paper we presented a novel approach to coding. We introduced the
problem, illustrated the rationale behind (the philosophy) of the Spartan
Programming, next we thoroughly described the techniques to apply to spartanize
the code (or to write it in the spartan way since the beginning) and eventually
we report the result of an experiment performed on a representative dataset of
Java software systems.  From the results of this experiment, that show that the
compression of spartanized code is higher than not spartanized one, we can
evince that spartanization have a meaningful impact on source code (???).  The
aim of this paper is mainly that of giving an overview of the spartanization
techniques.  In the future we intend to perform a number of experiments, in
order to carry on the research on the impact of the spartanization on software
systems' properties, including its quality.

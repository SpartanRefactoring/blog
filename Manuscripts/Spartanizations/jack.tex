% Then apply Burroughs-Wheeler, explain why better than zip.
% For entire project.

The experiment is based on two compression algorithms.
The first compression algorithm is called Jack, and it rips off from the file all the code note related to the programming logic, leaving
only the code intentionally written by the developers, namely the so called boilerplate code.

Specifically, in order to do that, Jack automatically removes:

\begin{itemize}
\item imports instructions
\item package
\item comments
\item strings perhaps use
\item keywords are replaced by a single symbol rep
but not identifiers. Take note that it is comon wisdom that the bulk of code is identifiers.
\item same for operators such as -> ++, <<<<=
\end{itemize}

The second compression algorithm was invented by Burroughs and Wheeler
\cite{Burrows:Wheeler:1994}. It is based on a transformation that performs a
permutation in the order of the characters If the original string had several
substrings that occurred often, then the transformed string will have several
places where a single character is repeated multiple times in a row.

The dataset of software used in this analysis is composed by 26 popular Java
projects whose size spans from around 1 Mbyte to almost 47 Mbyte - the main
statistics is reported in \cref{table:corpus}

\begin{table}
  \label{table:corpus}
  \caption{The results of the compression using the Burroughs-Wheeler algorithm}
  \centering
  \scalebox{0.8}{
    \begin{tabular}{lccc}
      \hline
              & \textbf{Java size} & \textbf{Java size BZip2} & \textbf{Java comp. ratio}⏎
      \hline
      Average & 12370223 & 1272331 & 9.25⏎
      Minimum & 1 019125 & 129 114 & 7.22⏎
      Maximum & 46965422 & 5607599 & 12.96⏎
      Range & 45946297 & 5478485 & 5.74⏎
      Median & 5128541 & 607535 & 8.91⏎
    \end{tabular}}
\end{table}

Boilerplate does not require much programming skills
Can be automated 
Remove;
\begin{itemize}
\item imports
\item package
\item comments
\item strings perhaps use
\item keywords are replaced by a single symbol rep 
but not identifiers. Take note that it is comon wisdom that the bulk of code is identifiers.
\item same for operators such as -> ++, <<<<= 
\end{itemize}

Then apply Burroughs-Wheeler, explain why better than zip.
For entire project.

Cite naturalness of software.

Preliminary results supporting the 
"Hypothesis that can never be proved"
Because this is not what you like.

Our dataset is composed by 26 popular Java projects whose size spans from around 1 Mbyte to almost 47 Mbyte - 
the main statistics on their size is reported in \ref{tab:proj-stat-source}

\begin{table}
\centering
\scalebox{0.8}{
\begin{tabular}{lccc}
\hline
& \textbf{Java size} &  \textbf{Java size BZip2} &  \textbf{Java comp. ratio}\\ 
 \hline
Average & 12370223 & 1272331 & 9.25\\
Minimum & 1 019125 & 129 114 & 7.22\\
Maximum & 46965422 & 5607599 & 12.96\\
Range & 45946297 & 5478485 & 5.74\\
Median & 5128541 & 607535 & 8.91\\
\end{tabular}}
\label{tab:proj-stat-source}
\caption{The results of the compression using the Burroughs-Wheeler algorithm.}
\end{table}

\begin{table*}
\centering
\scalebox{0.8}{
\begin{tabular}{lccccc}
\hline
& \textbf{Jack size} & \textbf{comp. ratio} & \textbf{Jack size BZip2} & \textbf{Ratio} & \textbf{Combined ratio}\\ 
 \hline
Average & 5078034 & 2.78 & 729934 & 6.60 & 18.37\\
Minimum & 456050 & 1.73 & 77706 & 4.89 & 10.18\\
Maximum & 19697634 & 12.85 & 3414889 & 8.68 & 83.65\\
Range & 19241584 & 11.12 & 3337183 & 3.79 & 73.47\\
Median & 1822510 & 2.34 & 343815 & 6.50 & 15.95\\
\end{tabular}}
\label{tab:proj-stat-original}
\caption{The results of the compression using the Jack algorithm alone and both Jack and 
Burroughs-Wheeler algorithms combined.}
\end{table*}

\begin{table*}
\centering
\scalebox{0.8}{
\begin{tabular}{lccccc}
\hline
 & \textbf{Jack size} &	\textbf{Jack size BZip2} & \textbf{Ratio} & 
 \textbf{Improvement in Comp. Ratio with Spartanization} & 
 \textbf{Total Improvement} \\
 \hline
Average & 5050647 & 719488 & 6.66 & 0.06 & 1.01\\
Minimum & 439825 & 75391 & 4.90 & -0.37 & 0.21\\
Maximum & 19173224 & 3275918 & 8.79 & 0.49 & 1.07\\
Range & 18733399 & 3200527 & 3.89 & 0.86 & 0.87\\
Median & 1959917 & 341851 & 6.79 & 0.08 & 1.04\\
\end{tabular}}
\label{tab:proj-stat-partial}
\caption{The results after performing a partial spartanization of the code, compared 
with the compression ratios obtained with the Jack and Burroughs-Wheeler algorithm.}
\end{table*}

\begin{table*}
\centering
\scalebox{0.85}{
\begin{tabular}{lccccc}
\hline
 & \textbf{Jack size} &	\textbf{Jack size BZip2} & \textbf{Ratio} & 
 \textbf{Overall Improvement} & 
 \textbf{Improvement over Partial Spartanization} \\
 \hline
Average & 4947319 & 713857 & 6.56 & -0.03 & -0.09
\\
Minimum & 430375 & 75160 & 4.89 & -0.40 & -0.28
\\
Maximum & 18890899 & 3244851 & 8.72 & 0.41 & -0.01 \\
Range & 18460524 & 3169691 & 3.83 & 0.81 & 0.27
\\
Median & 1923056 & 339930 & 6.70 & 0.00 & -0.09 \\
\end{tabular}}
\label{tab:proj-stat-total}
\caption{The results after performing a total spartanization of the code, compared 
with the compression ratios obtained with the Jack and Burroughs-Wheeler algorithm.}
\end{table*}


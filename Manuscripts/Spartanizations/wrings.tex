\newlength\ruleLength

One can also note that \cref{figure:free} demonstrates
  a structural wring that puts first the shortest branch of a
  an \kk{if}, at the price of negating the conditional.
However, the number of structural wrings is too large to enumerate and
demonstrate all here. The following categorization provides a summary:

\begin{description}
  \item[S0] \emph{Null impact elimination.} The tooblox contains wrings to make
    trivial simplifcations of computation whose effect is null. Examples
    include comparison with \kk{true} (always yielding the compared value),
    use of unary ``\cc{+}'',  addition of \cc{0}, multiplication by \cc{1}, a
    conjunction with a \kk{true} in it, disjunction with \kk{false} in it, concatenation 
    with the empty string (\cc{""}), etc.

    Nullary computation is not frequent in actual code, but it often generated
    out of other, more complicated wrings.

  \item[S1] \emph{Removal of syntactic baggage.}
    Syntactic baggage is keywords and other language tokens,
     whose presence does not contribute to the code semantics.
    In other words, they are safe to remove while preserving meaning.
    Other than the famous braces around a singleton statement,
    The spartan toolbox includes wrings to remove redundant modifiers,
    such as \kk{final} (when present on \kk{private} methods) \kk{abstract}
    (when present on \kk{interface}s), spurious \kk{super}\cc{()} calls, redundant
    semicolon at the of initialization lists, redundant calls to 
    \cc{toString()} of objects involved in string concatenation, etc.

Transformation rules are a convenient shorthand for specifying wrings, e.g.,
rules:

\begin{adjustbox}{max width=\columnwidth}
  {\footnotesize
    \setlength\ruleLength{1.19\columnwidth}
    \let\columnwidth \ruleLength
    \begin{minipage}{\columnwidth} % Can use values greater than one.
      \begin{align}
        & \text{\cc{\kk{if}~($C$)~\kk{return}~$E$;~\kk{else}~$S$;}} ⇒
        \text{\cc{\kk{if}~($C$)~\kk{return}~$E$;~$S$;}} ⏎
        & \text{\cc{\kk{if}~($C$)~$S$;~\kk{else}~;}} ⇒ \text{\cc{\kk{if}~($C$)~$S$;}}⏎
        & \text{\cc{\kk{if}~($C$)~;~\kk{else}~$S$}} ⇒ \text{\cc{\kk{if}~(!$C$)~$S$;}}⏎
        & \text{\cc{\kk{while}~($C$)~❴❵}} ⇒ \text{\cc{\kk{while}~($C$)~;}}
      \end{align}
    \end{minipage}
  }
\end{adjustbox}

\noindent remove redundant \kk{else} branches, an empty main branch of conditional, and
convert a~\cc{❴}…\cc{❵} pair to~\cc{;}. Wrings however may be more complicated:
removing an empty \kk{else} branch may change semantics if attention is not paid
to occasions in which the remaining \kk{if} attaches, as a result of the
transformation, to a different \kk{else} branch.

\item[S2] \emph{Factorization with the distributive propety.}
  The distributive property occurs in code for many more occasions
  than plain arithmetics, including Boolean algebra, assignments, and
  conditionals. The structural wrings in the toolbox that take
  advantage of this property are

\begin{adjustbox}{max width=\columnwidth}
  {\footnotesize
    \setlength\ruleLength{1.19\columnwidth}
    \let\columnwidth \ruleLength
  \begin{minipage}{\columnwidth} % Can use values greater than one.
    \begin{align}
      & \text{\cc{$A$*$C$ +~$C$*$B$}} ⇒ \text{\cc{($A$+$C$) *~$B$}},⏎
      & \text{\cc{$A$ &&~$B$ ||~$C$ &&~$B$}} ⇒ \text{\cc{($A$ ||~$C$) &&~$B$}},⏎
      & \text{\cc{$A$ ||~$B$ &&~$B$ ||~$B$}} ⇒ \text{\cc{$A$ &&~$C$ ||~$B$}},⏎
      & \text{\cc{\kk{if}~($C₁$)~$S$; \kk{if}~($C₂$)~$S$;}} ⇒ \text{\cc{\kk{if} ($C₁$ &&~$C₂$)~$S$;}}⏎
      & \text{\cc{\kk{if}~($C₁$)~\kk{if}~($C₂)$~$S$; \kk{if}~($C₂$)~$S$;}} ⇒ \text{\cc{\kk{if} ($C₁$ ||~$C₂$)~$S$;}} &⏎
      & \text{\cc{\kk{if}~($C$)❴$S₁$;$S₂$❵~\kk{else}~❴$S₁$;~$S₃$❵}} ⇒ \text{\cc{$S₁$;\kk{if}~($C$)~$S₂$;~\kk{else}~$S₃$}}⏎
      & \text{\cc{\kk{if}~($C$)❴$S₁$;$S₃$❵~\kk{else}~❴$S₂$;~$S₃$❵}} ⇒
      \text{\cc{\kk{if}~($C$)~$S₁$;~\kk{else}~$S₂$};~$S₃$}⏎
      & \text{\cc{$V₁$ =~$X$;~$V₂$ =~$X$}} ⇒ \text{\cc{$V₁$ =~$V₂$ =~$X$}}⏎
      \label{eq:ternary}
      & \text{\cc{$C$~? f($X$) : f($Y$)}} ⇒ \text{\cc{f($C$~?~$X$:~$Y$)}}⏎
      & \text{\cc{\kk{if}~($C$)~\kk{return}~$E₁$;~\kk{else}~\kk{return}~$E₂$}} ⇒ \text{\cc{\kk{return}~$C$~?~$E₁$~:~$E₂$;}}
  \end{align}
  \end{minipage}
}
\end{adjustbox}

Worthy of notice in this partial list are symmetries of distrubtion: right and
left distribution (as in \kk{if}), the replaceability of operators (\cc{&&}
is can be replaceable by \cc{||}, as long as the mirror replacement of
\cc{||} by \cc{&&} is carried out collaterally), etc.

\item[S3] \emph{Ternarization} The third kind of non-trivial wrings are those
  that attempt to change an \kk{if} statement into a ternary expression.
  The basic transofrmation rule is:
  \begin{quote}
  \footnotesize
    \begin{align}
      & \text{\cc{\kk{if}~($C$)~$S$;~\kk{else}~$S₂$;}} ⇒ \text{\cc{$C$~?~$S₁$~:~$S₂$}}
  \end{align}
\end{quote}
except that more caution must be applied before applying this transformation.

Firstly, notice that the original code is a statement, while the transformed is
an expression. The transfomration can only work if the distributive rule
\cref{eq:ternary} can be applied next to to can further push down the ternary
operator onto~$S₁$ and~$S₂$ whereby making the transformed into a statement.

Secondly, since in many the \kk{else} branch of a wring is missing, the
spartanization may need to complete it, by transformation rules such as:
\begin{quote}
  \footnotesize
    \begin{align*}
      & \text{\cc{\kk{if}~($C$)~\kk{return} $E_1$;\kk{return}~$E₂$;}} ⇒
      \text{\cc{\kk{if}~($C$)~\kk{return} $E_1$;\kk{else} \kk{return}~$E₂$;}} ⇒
      \text{\cc{$C$~?~$S₁$~:~$S₂$}}.
  \end{align*}
\end{quote}
Observe that the above transformation rule is not a wring: It increases the 
number of tokens, to make the input more amenable to ternarization.

\item[S4] \emph{Canonic form} Wrings of this kind to bring the 
  code into a more canonical form, by sorting by length terms of addition
  factors of multiplication, pulling out multiplication by \cc{-1}, and  
   even  pushing down logical negation using De Morgan
   laws (possibly flipping comparison operations when necessary), etc.

   Also included in this group is the wring that converts \cc{x.toString()} to
   \cc{""+x}, a wring that favors, whenever possible, postfix increment and decrement
   operations to their prefix version, and more.

\item[S5] \emph{Inlining and scope reduction}
Use canonical form for the most common expressions, as much as possibile.

    Follow patterns: Pattern: A*x + B
    Sorting by size, but also as per the previous rule.
    De Morgan pushdown logical negation.
    Pullup arirthmetic negation.
    Pushdown ternarizations. For example:
        % Example:
        \begin{code}{java}
  \end{code}
    Apply the distributive rule on booleans. For instance:
        % Example:
 This is one variant (there are other three).
    More specific first: How should you order X == Y.
        Rule: you start from the more specific and increase specifity, i.e.,
        S(X) > S(Y)
            null/this/true/false
            0, 1, ""
            12, 13.4
            Classes: public, protected, package, private
            Fields: Static fields, Fields,
            parameters,d
            local variables ordered by scope.
            ¢
\item{Inlining}

Spartan programming promote a coding style that do not use many variables,
meaning that it strives to inline instructions instead of storing their results
on variables. The spartanized version of a code not originally written in the
Spartan way, would probably see many of the included variable deleted whenever
and wherever it is possible.

The idea behind this approach is to reduce the cognitive burden of the
developers, either in case of whom is coding who has to spent time and effort
in finding effective names, either in the case of a reader who need to put some
effort to figure out what is the meaning of a variable, starting from its name.

Since variables change, it needs to be sure that it happens when it is supposed
to do. Moreover it is needed to associate a role to a variables, so that the
code that reads that variable can understand and appreciate it and figure out
the list of location where it may be read (CHANGE).

These are some technique for variable elimination.

The following is a self explanatory example of inlining:

\begin{code}{JAVA}
double a = p.area();
return a;
\end{code}

becomes

\begin{code}{JAVA}
return p.area();
\end{code}

\item{Early returns}

Another technique for variable elimination is the early return and it consist
in returning the result of a computation as soon as it is available, instead of
storing it in a variable (and then returning it). An example of early return
is reported in the following code.

\begin{code}{JAVA}
private boolean isKeyword(String token, String[] keywords) {££
    for (int i = 0; i < keywords.length; i++)d
        if (token.equals(keywords[i]))
            return true;
    return false;
}
\end{code}

Using \kk{forech} construct available in several langues it is possible to
avoid the index (in those case in which there are no instructions to access to
it).
\begin{code}{JAVA}
private boolean isKeyword(String token, String[] keywords) {££
    for (final String element : keywords)
        if (element.equals(token))
            return true;
    return false;
}
\end{code}
The same result can be obtained using the some construct to loop over
collections, preferred instead of the arrays, without the use of the
\cc{Iterator} class, as in the previous versions of Java.

% With the early return it becomes:
% \begin{code}{JAVA}
% private ResultRenderer getRenderer() throws InstantiationException,
% IllegalAccessException, ClassNotFoundException {££
% returnᵣesultRenderer != null ? (ResultRenderer)
% Class.forName(ᵣesultRenderer).newInstance() : DefaultResultRenderer();
% }
% \end{code}

\item{Control simplification}
Spartan programming foster the use of the ternary operator over the
\kk{if-else} construct. This way the implementation of the factorial
\begin{code}{JAVA}
int factorial(int n) {d
    if (n <= 1)
        return n;
    else
        return n * fact(n - 1);
}
\end{code}
becomes
\begin{code}{JAVA}
int factorial(int n) {££
  return n <= 1 ££? n : n * fact(n - 1);
}
\end{code}
in this case in a return statement that allows inlining of the instruction.

\item{String simplification Techniques}

    String transformations
    instead of \cc{££.toString()} use \cc{"" +}
    x.equals(“a”) -> “a”.equals(b)
    Auto insert “”

\item{Miscellaneous Techniques}

Here we report a series of techniques that not fall expressly in one of the
previous category.

    Favor prefix operators over postfix ones.
    In expressions order the operator according to their length (shortest
        come first) or specificity (more specific come first).
    Associative rule of ifs:
        \begin{code}{JAVA}
if (a) if (b) x;
  \end{code}
  \begin{code}{JAVA}
if (a && b) x
  \end{code}
  % Associative rule of ifs: if (a)
  % \begin{code}{JAVA}if (b) x;\end{code}
  % \begin{code}{JAVA}if (a && b) x\end{code}
  \begin{code}{JAVA}
if (this == null) return super();
  \end{code}

  INLINING??

      Off load exceptional cases quickly.
      inline single use of variables.
      Inline multiple use of variables: if 1) result is shorter and 2
) the expression has no side effects.
      Pseudo inlining with initializers.
      Scope:

  Scope:

      Move into for
      Move into try
      Move next to use

  Variability

      Variability (add final when possible.)
      Variables (inlining)
      What’s first

  Sorting

      Most external escape
      Shortest: 2*112*a*ab
      Least specific

  Canonical ordering

      Simplificatons

              Algebraic simplifications
              Logical simplification
              Literal simplification
              Arithmetical simplification

      Patterns: 2 * x + 1d
      Pushdown:
      Logical Negation
      Arithmetical Negation
      Auto distributive
      Repeated deterministic expressions
      Auto

              Auto Collect
              Auto insert/remove 0+ in ternarization.
              Auto insert/remove 1*
              Auto add “” +

  \subsection{Renaming}

      In constructors: Arguments to corresponding field names
      In fields: to canonical instance name or pluralized
      In getters: Change name to fieldName if it is a getter.
      In Methods
      In local variables:
      return variable to \$

    Change argument to one single case letter
        % In world amalgams, use the first letter of the last word.
        % Multiplicities

            Deal with multiplicities of arrays
            Deal with multiplicities of collections
            Deal with nested multiplicities

    Change return type of void to ClassName and return this

            Single argument:

                    To \_\_ if unused
                    To \$ in case it is returned
                    To fieldName in case it is a setter
                    To ¢ in case method has no inner methods and is not a
                        setter

                    Any number of arguments:
                    To \$ in case it is returned
                    To \cc{\_\_}
                        if it is unused and documented as such

                % Change argument to one single case letter
                % In world amalgams, use the first letter of the last word.
            Deal with multiplicities

%
% Ternarize
% Early return
% Example:
% \begin{code}{JAVA}
% @SuppressWarnings("unused") private
% void earlyReturn(int a) {££
% if (new Random().nextInt() <= 3) {££
% earlyReturn(12);
% earlyReturn(124);
% } else {££
% earlyReturn(4);
% earlyReturn(5);
% earlyReturn(44);
% }
% }
% \end{code}
%

\Cref{figure:eclipse:spartan} shows the spartanized version of the program in
the \cref{figure:eclipse}. Employed were nominal spartanization, e.g., renaming
variable \cc{names} to~\cc{\$}, structural spartanization, e.g., for the
removal of braces, and modularization.

The product of external cleanup, which in this case amounts to adding
\kk{final}s and \kk{@Override}, is not included in this example.

The main difference between the figures is due to method extractions, conducted
manually. Auxiliary \kk{private} function \cc{main()} minimizes exploitation of
scope by denying access to the \cc{args} argument of function \cc{main}.
Function \cc{makeCompilationUnit()} takes advantage of the fact that variable
\cc{parser} is only used in a limited section of the \kk{public} function
\cc{main(String[] args}, etc.

Examining the body of the anonymous class (that one that \cc{extends}
\cc{ASTVisitor}) we see some more extracted methods. The reduction in clutter
due to spartanization makes it possible to identify another spot of useful
method extraction.

\begin{figure}
  \caption{Spartanized version including structural, nominal, modular, but
    not cleanups, of the \Java program in \cref{figure:eclipse}.}
  \label{figure:eclipse:spartan}
  \begin{adjustbox}{max width = \columnwidth}
\begin{code}[minipage, width=1.13\columnwidth]{JAVA}
public class Test {££ // comments and £\kk{import}£ directives omitted for brevity
  public static void main(String[] £\_\_£) {££ main(); }
  private static void main() {££
    CompilationUnit u = makeCompilationUnit();
    u.accept(new ASTVisitor() {££
      Set £\ignore$£$ = new HashSet();
      public boolean visit(SimpleName ¢) {££
        note(¢);
        return true;
      }
      public boolean visit(VariableDeclarationFragment ¢) {££
        collect(¢.getName());
        return false;
      }
      void note(SimpleName ¢) {££
        if (£\ignore$£$.contains(¢.getIdentifier()))
          System.out.println("Usage of '" + ¢ + //
            "' at line " + u.getLineNumber(¢.getStartPosition()));
      }
      void collect(SimpleName ¢) {££
        £\ignore$£$.add(¢.getIdentifier());
        System.out.println("Declaration of '" + ¢ + //
          "' at line" + u.getLineNumber(¢.getStartPosition()));
      }
    });
  }
  private static CompilationUnit makeCompilationUnit() {££
    return (CompilationUnit) makeParser().createAST(null);
  }
  private static ASTParser makeParser() {££
    ASTParser £\ignore$£$ = ASTParser.newParser(AST.JLS3);
    £\ignore$£$.setSource(
      "public class A {\n" +
      " int i = 9;\n"+ //
      " int j;\n" + //
      " ArrayList<Integer> al = new ArrayList<Integer>();\n" + //
      " j=1000;\n" + //
      "}\n" + //
      .toCharArray()
);
    £\ignore$£$.setKind(ASTParser.K£\_£COMPILATION£\_£UNIT);
    return £\ignore$£$;
  }
}
\end{code}
  \end{adjustbox}
\end{figure}

% Yet another, further simplified version of the code of \cref{figure:eclipse} shows
% up in \cref{figure:eclipse:extract}. This figure depicts only the anonymous
% class found in previous figures, but with extra method extracted to simplify
% the code. Of course, a demand for clever method extraction can not be part of a
% strucured We argue that the
% process of spartanization helped in
% this case to highlight

% However, being in a certain sense, counter intuitive, in order to apply
% Spartanization on their working routines, the developers need to respect a firm
% discipline. From a certain standpoint, Spartan Programming might be seen as a
% list of code style prescription, but it would just scratch the surface. It is
% far more than that, since it
%
% Being minimized, spartanized code might be less readable that the original.
% This is … Spartan programming is not directly concerned with readability, at
% least not in its subjective and cultural-dependent sense. In fact, spartan
% programs will bring much misery to anyone preferring long, verbose programs.
%
% Spartan programming prescribes, a careful use of the variables, stressing the need
% to inline variables that are used only once and emphasizing the use of foreach loops
% in languages implementing it. Variables visibility and accessibility should also be
% possibly minimized, favoring the use of private over public variables whenever it
% is possible.
%
% Also the names of variables should be taken short for local variables, with
% short scope, leaving to their class names to explicit their behavior.
% Minimizing variability using final and @NonNull annotation. Stress the use of
% generic names technique. Minimize lifespan avoiding persistent variables when
% it is possible. Favor the use of collection instead of arrays. Interfaces
% should present a minimal number of parameters, also minimizing the interaction
% between them. Favor ternarization over the use of if-then-else constructs and
% use a simplified version of the conditionals that implies an early return.
% Moreover, use an early exit logic with continue, break and return instructions.
%
% Everything should be formatted in order to make an efficient use of the screen
% space, by removing unnecessary tokens (i.e., redundant parentheses, etc.).
% Methods should be as short as possible.

\end{description}

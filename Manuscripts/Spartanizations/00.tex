\documentclass[preprint,10pt,nonatbib]{sigplanconf}
\title{Code Spartanization}
\subtitle{\scriptsize \emph{one rational approach for\\ resolving religious style wars}}
\authorinfo
  {Yossi Gil \and Matteo Orrù \and Gal Lalouche}
  {%
    Dept.\@ of Comp.\@ Sc.\@\\
    The Technion---I.I.T.\\ 
    Haifa 32000, Israel
  }
  {❴ YoGi~$|$~Matteo.Orru ❵ @cs.technion.ac.il}


\usepackage{\jobname}
\begin{document}

\maketitle

\def\ignore#1{}
\def\gal{\marginpar[G$\Rightarrow$]{$\Leftarrow$G}}
\def\yossi{\marginpar[Y$\Rightarrow$]{$\Leftarrow$Y}}
\def\matteo{\marginpar[M$\Rightarrow$]{$\Leftarrow$M}}

\begin{abstract}
  <<<<<<< HEAD
% Inspired by the declarative features of \cc{make}, we propose \Reap: a tiny
% programming language in which the notion of a class defining methods and fields
% is replaced by that of an \emph{environment} defining \emph{properties}, some
% of which are adorned by \emph{recipes}.  Properties are similar to variables,
%    except that a property whose value was not explicitly set may still yield a
%    value: \Reap's underlying inference engine applies any required recipes to
%    compute this value from values of other properties. 
% 
% Traditional inheritance is replaced in \Reap by both acquisition and reaping.
% In \emph{acquisition}, an environment may either add or overload properties and
% recipes of some other environment. An environment \emph{reaps} another
% environment, by first making a copy of its content, and then adding and
% overloading properties and recipes. 
% 
% Even though recipe application is caching and lazy as in
% \NonCitingUse{Haskell}, \Reap is not functional. In the course of
% computation, properties may be repeatedly set and reset.
% 
% If properties are likened to cells and recipes to functions that reside in them,
% then \Reap can be equated to traditional spreadsheets, except that cells
%    are named rather than being addressed, and \Reap's engine being lazy. 
%    
% \Reap's inheritance model is similar to that of \NonCitingUse{Self} and
% \NonCitingUse{JavaScript}: Reaping is similar to the standard emulation of
% inheritance of classes in prototype based languages, while the acquisition is the
% usual extension in such languages.  However, unlike \NonCitingUse{Self} and
% \NonCitingUse{JavaScript} \Reap is statically typed.
% 
% Interestingly, \Reap embeds in it not only traditional object-oriented and
% prototype-based programming, but also traditional by-value and by-reference
% parameter passing.  
% 
% Presently \Reap is manually transliterated into a \NonCitingUse{Java} program
% which connects to the inference engine library by means of terse calls of
% fluent API. Usability and ensuing particular programming style are demonstrated
% in an implementation of an Eclipse plugin. 


=======
Inspired by the declarative features of \cc{make}, we propose \Reap: a tiny
programming language in which the notion of a class defining methods and fields
is replaced by that of an \emph{environment} defining \emph{properties}, some
of which are adorned by \emph{recipes}.  Properties are similar to variables,
   except that a property whose value was not explicitly set may still yield a
   value: \Reap's underlying inference engine applies any required recipes to
   compute this value from values of other properties. 

Traditional inheritance is replaced in \Reap by both acquisition and reaping.
In \emph{acquisition}, an environment may either add or overload properties and
recipes of some other environment. An environment \emph{reaps} another
environment, by first making a copy of its content, and then adding and
overloading properties and recipes. 

Even though recipe application is caching and lazy as in
\NonCitingUse{Haskell}, \Reap is not functional. In the course of
computation, properties may be repeatedly set and reset.

If properties are likened to cells and recipes to functions that reside in them,
then \Reap can be equated to traditional spreadsheets, except that cells
   are named rather than being addressed, and \Reap's engine being lazy. 
   
\Reap's inheritance model is similar to that of \NonCitingUse{Self} and
\NonCitingUse{JavaScript}: Reaping is similar to the standard emulation of
inheritance of classes in prototype based languages, while the acquisition is the
usual extension in such languages.  However, unlike \NonCitingUse{Self} and
\NonCitingUse{JavaScript} \Reap is statically typed.

Interestingly, \Reap embeds in it not only traditional object-oriented and
prototype-based programming, but also traditional by-value and by-reference
parameter passing.  

Presently \Reap is manually transliterated into a \NonCitingUse{Java} program
which connects to the inference engine library by means of terse calls of
fluent API. Usability and ensuing particular programming style are demonstrated
in an implementation of an Eclipse plugin. 
>>>>>>> ba9046dcc594cc92ec4643e2a45f6218c6dad2ff

\end{abstract}

\section{Introduction}
As a whole, the stackoverlow\urlref{http://stackoverlow.com} community
recommedns against asking coding style and coding convetions related questions.
Tag \cc{code-styling} and its many synonyms: \cc{code-convention},
\cc{coding-guidelines} are specifically
earmarked\urlref{http://stackoverflow.com/tags/coding-style/info} as
\begin{quote}
DO NOT USE! This tag refers to an entirely opinionated subject and is therefore
no longer on-topic. Refactoring, braces, indentation, Hungarian notation, and
other stylistic issues relating to code.
\end{quote}
But it is not only curly braces matters that may raise
this kind of ‟religious” wars. Even the famous convention maxim ‟\emph{gotos
are evil}”~\cite{Dijksta:must be in bib} was challenged in the scientific
literature~\cite{Knuth: and there are several others}.

Telling the story of ‟Spartan Programming”, one particular style guideline for
\Java, this manuscript is also likely to be contentious, even more so, since
‟spartanized” code such as

\begin{figure}[h]
  \begin{Code}{JAVA}{Where was this taken from?}
public static <T, C extends Collection<T>>
                              accumulate<T, C> to(final C c) {
  return new accumulate<T, C>() {
    @Override public accumulate<T, C> add(final @Nullable T t) {
      if (t == null)
        return this;
      assert t != null;
      c.add(t);
      return this;
    }
    @Override public C elements() {
      return c;
    }
  };
}
\end{Code}
\label{figure:shock}
\caption{An example of spartan code}
\end{figure}

\begin{figure}[h]
\begin{Code}{JAVA}{Where was this taken from?}
default accumulate<T, C> addAll(final Iterable<? extends T>... tss) {
  for (final Iterable<? extends T> ts : tss)
    addAll(ts);
  return this;
}
\end{Code}
\label{figure:shock-2}
\caption{An example of spartan code}
\end{figure}

\begin{Code}{JAVA}{Where was this taken from?}
/** Tests for primality.
  * @param ¢ candidate to be tested
  * @return <code><b>true</b></code> <i>iff</i> the parameter is prime. */
public static boolean isPrime(final int ¢) {
  return ¢ < -1 && isPrime(-¢) // deal with negative values
      || ¢ > 1 && isPrime¢(¢); // any integer >- 2
}
private static boolean isPrime¢(final int ¢) {
  for (int d = 2; d * d <= ¢; d++)
    if (¢ % d == 0)
      return false;
  return true;
}
\end{Code}
\begin{figure}[h]
  \begin{Code}{JAVA}{Where was this taken from?}
default Set<Variable> freeVariables() {
    final Set<Variable>~$ = new LinkedHashSet<>();
    if (isVariable())
~$.add((Variable) this);
    else
      for (final Term ¢ : this)
~$.addAll(¢.freeVariables());
    return~$;
  }
\end{Code}
\label{figure:shock-3}
\caption{An example of spartan code}
\end{figure}

may look bizzare to eyes accustomed to traditional \Java code.

This manuscript reports in

Mention that \Go forces you to use them!
Mention that are compulsory in \kk{synchronized},
\kk{try},
\kk{catch},
\kk{finally},

but on the other hand, in a recent addition lambda expressions~\cite{lambda}
the options are maintained.
You can write
\begin{JAVA}
()-> 3
\end{JAVA}
and
\begin{JAVA}
  ()-> {3}
\end{JAVA}
check this.

‟\emph{the braces war}”

Why should this be important?
\begin{itemize}
    \item Easy to write in your own style, but how can
      you always adjust to the style of others.
      Waste of neurons.
    \item Code comparison. Which part of my change is semantical and which
      part of this my own personal style.
    \item Code analysis tools should adjust to different styles and many
        different ways for doing the same thing.
    \item METRICS: IN a recent paper Gal and I showed that size
      is the only thing that matters.
    \item METRICS: IN a recent paper Gal and I showed that the notion of
      "size" is very different in different projects.
\end{itemize}

The holly grail of "canonical representation of code" .
Does LOC mean anything?
Not if you apply different indentation rules.
A better metric is NOT = number of tokens.
But even this is influenced by coding styles, including but
not limited to the famous braces war.

\textbf{Mention naturlaization (MAT: moved here from the beginning)}

Some recent studies investigated the presence of some regularities in code that
make source code repetitive and predictable as well as natural language.
In Hindle et al.~\cite{Hindle:Bar:Su:Gabel:Devanbu:2012} analyzed a large corpora composed
by ... of software systems and compared the properties with natural language.
They found out that source code present many statistical properties that

\subsection{Background}
There is very little scientific work on style.
Mention the Latin quote.

Mention the style guides on C++ and Java. Oracle.

But where these styles a matter of a careful and raional design process?
What were the underlying principles.

Last paragraph: many people believe that style is irrelevant and unimportant.
We believe that it is.

\subsection{Spartan programming}
First occurred in a stupid TOOLS USA 199X. tutorial.
The point of making and perfecting it was to "be rational" and
reexamine all assumptions.
Rules have changed.
Utter despise of backward compatability the need to rewrite the code.

Mention it. The spartanization tool.

Say that some of your alma mater gang love to do this kind of metrics, but
no one really knows whether the results are applicable to different projects
since
\cite{Turnu:Concas:Marchesi:Tonelli:11}

the "Idionsyncratic Projects Hypothesis" I also had a plateau paper with
Dany and Maayan about this. A number o years ago.

This paper gives an overview of spartanization, future work will be
experiments.

\subsection{Contribution}

\textbf{(the following text has been adapted by the wiki)}⏎

Spartan programming is an approach to writing code aimed at improving at
reducing code complexity by appling many techininques, some already known in
literature, some others more novel. It was introduced in the mid on nienties
and it has been a part of some teachings. Its rationale is based on some
principles, but at the end of the day it is a techninque to pursue an extreme
minimalism code writing.

One of these principles is the "Babylonian Tower Principle": It states that the
number of abstraction levels of a software systems is limited. A famous study
by Miller, also proved that the cognitive burden people might bear, namely the
number of cognitive items that can be processed in human being working memory,
is limited to 7 (plus or minus two) elements. By simplifying the basic element
of more complex, structured composite component, spartan programming reduce the
developers' cognitive burden.

However, being in a certain sense, counter intuitive, in order to apply
Spartanization on their working routines, the developers need to respect a firm
discipline. From a certain standpoint, Spartan Programming might be seen as a
list of code style presciption, but it would just scratch the surface. It is
far more than that, since it

Being minialized, spartanized code might be less readable that the original.
This is ... Spartan programming is not directly concerned with readability, at
least not in its subjective and cultural-dependent sense. In fact, spartan
programs will bring much misery to anyone preferring long, verbose programs.

Spartanization aims at minimizing Vertical complexity (the number of lines of code),
Horizontal complexity, the number of tokens, character, parameters and variables,
as well as reduce the number of loops and conditional statements.
Spartan programming prescribes, a careful use of the variables, stressing the need
to inline variables that are used only once and emphasizing the use of foreach loops
in languages implementing it. Variables visibility and accessibility should also be
possibly minimized, favouring the use of private over public variables whenever it
is possible.

Also the names of variables should be taken short for local variables, with short scope,
leaving to their class names to explicit their behaviour. Minimizing variability using
final and @NonNull annotation. Stress the use of generic names technique. Minimize lifespan
avoiding persistent variables when it is possible. Favour the use of collection instead
of arrays. Interfaces should present a minimal number of parameters, also minimizing
the interaction between them. Favour ternarization over the use of if-then-else constructs
and use a simplified version of the conditionals that implies an early return. Moreover,
use an early exit logic with continue, break and return instructions.

Everything should be formatted in order to make an efficient use of the screen space,
by removing unnecessary tokens (i.e. redundant parentheses, etc.).
Methods should be as short as possible.


\paragraph{Outline}
The present work is structured as it follows.
\cref{section:principia} illustrates the Process of Spartanization, 
including its rational, manner, extent of application as well as
the automation of the process. 
Next, to get the reader more accustomed to this new coding style, 
\cref{section:look} presents some example of spartanized code (KEEP THIS SECTION?).
\cref{section:initial} illustrates and discussses 
the results of our experiments.
% \cref{section:zz}
Finally, in \cref{section:zz} we draw some conclusions and we shortly
outline future work.

\section{The Process of Code Spartanization}
\label{section:principia}
The spartan programming style tries to achieve terse, clear and modular code,
by \textit{simultaneous} minimization of these measures:
\begin{enumerate}
  \item \textit{code size,} which here means all of number of lines, number of
        characters, \emph{and}, number of language tokens;
  \item \textit{vertical complexity,} e.g., the number of lines in each
        method;
  \item \textit{horizontal complexity,} including, the number of parameters to a
        module (method or generic class), \emph{and} the level of nesting of
        control commands, \kk{if}, \kk{while}, \kk{switch}, etc. However,
        horizontal complexity does not concern line length, which is generally
        discarded by spartanization.
  \item \textit{control complexity,} which is the total number of control
        commands in the code.
  \item \textit{variability,} which means minimizing visibility, ability to
        change, lifetime, and scope of variables, just as their number.
        This means the introduction of \kk{final} whenever possible,
        variable inlining, preferring local variables to parameters, parameters
        to fields, and fields to \kk{static} fields, careful use of
        \cc{public}, etc.
  \item \textit{exploitation of scope,} which means the number of distinct
        identifiers (of fields, methods, etc.) used in any module.
  \item \textit{life and size of stacked context,} which is the most intricate of the
        spartan minimization ends.

        To understand what terms stacked context and its life mean, consider
        this code snippet

\begin{code}[minipage,width=54ex]{java}
ThingTwo ii = new ThingTwo();
ii.wreck(house);
System.out.println(ii);
ThingOne i = new ThingOne(); // Stack context of £\cc{ii}£
System.out.println(i); // Context of £\cc{ii}£ still stacked
i.wreck(house); // Stacked still£…£
System.out.println(i); // Still!
cat.join(i, ii); // Finally: recalling context of £\cc{ii}£.
  \end{code}
  The last line in this snippet joins the three objects,
  \cc{cat},~\cc{i}, and \cc{ii}. The four preceding lines prepare
  and print object~\cc{i}.

  Object~\cc{i} is prepared and printed in the remainder, i.e., the first three
  lines of the snippet. The fact that it existed is stacked then during the
  preparation of~\cc{i}, and restored only at the last line.

  By swapping the order of construction, we obtain code in which the stacked is
  a bit shorter,~\cc{i} instead of~\cc{ii}, and in a linear read of the
  program, it is stacked for three lines rather than four.

\begin{code}[minipage,width=54ex]{java}
ThingOne i = new ThingOne();
System.out.println(i);
i.enter(house);
System.out.println(i);
ThingTwo ii = new ThingTwo(); // Stack context of £\cc{i}£.
ii.enter(house); // Context of £\cc{i}£ still stacked.
System.out.println(ii); // Context is stacked still.
cat.join(i, ii); // Recalling context of £\cc{ii}£.
  \end{code}
\end{enumerate}

\subsection{Rationale}
\label{section:rationale}
To an extent, these minimization ends are reformulation of traditional belief
on software: that software should be organized in short cohesive modules whose
code depends on few globals, and is frugal in its use of control. Conversely,
complaints on code quality, include deep and complex nesting \kk{if}s and
\kk{for}s, endless parameters' lists, piles of global variables and many
methods and classes that should be understood before understanding any given
method, etc.

Also, the end of minimizing life and size of saved contexts captures in it
principles such as early \cc{throw}, \cc{return} and \cc{break} (from loops) on
errors, ordering by increasing length to increase readability, and also the
practice of placing at first the short branch of an \cc{if} conditional, etc.

Unlike other coding guidelines, spartanization prescribes a (semi-automatic)
uncompromising process by which code can be spartanized. The central novelty
therefore in spartan programming is not in these numerical minimization ends,
but in the \emph{manner}, \emph{extent}, and \emph{automation} of the quest
for their optimization. The last three \emph{emphasized} words make the titles
of \cref{section:manner}, \cref{section:extent}, and \cref{section:automation}
ahead.

\subsection{Manner of application}
\label{section:manner}
The spartanization process is governed by the spartan
\emph{toolbox}, which is a collection of many (currently around sixty)
structural and nominal wrings. Each \emph{wring} is a refactoring that does
not degrade quality as measured by any of minimization ends: LOC, number of
tokens, nesting level, size of stacked scope, etc., while improving on at least
one of them.

A \emph{structural wring} is a wring that modifies the structure of the code,
A simple structural wring is the one removing
useless \kk{abstract} and \kk{public} modifiers from
declarations made withing an \kk{interface}.
The application of a chain of structural wrings converts
\begin{lcode}{JAVA}
if (f() == true) {££
  return false;;
} else {££
  return true;;
}
\end{lcode}
\noindent into
\begin{lcode}{JAVA}
return !f();
\end{lcode}
\noindent
This chain includes a wring to remove redundant curly brackets and one for the
removal of redundant semicolons.
Also involved is a wring for the conversion of an \kk{if} into to a
\emph{ternary} expression (one employing the operator~\cc{$·$?$·$:$·$}), and a
wring simplification of ternary expressions involving boolean literals.

In contrast, a \emph{nominal wring} is a wring that does not modify the
code itself, but renames some entities in it, most particularly, variables.
For example, the toolbox includes a nominal wring to convert the name of the
variable returned by a method, to the one character identifier ‟\cc{\$}”,
which is the short spartan name for result.

If none of the wrings in the toolbox is applicable to the code, then
spartanization has reached a (local) optimum. Otherwise, code spartanization
proceeds by selecting one such wring and continuing with the code obtained
after applying this wring.

This process can be automated or carried out interactively. Many of the wrings
are implemented as part of an open source spartan refactoring Eclipse
plugin\urlref{https://github.com/SpartanRefactoring/spartan-refactoring} which
makes it possible for programmers to apply much of it.

In addition to wrings, spartanization may use
\begin{itemize}
    \item cleanup by external tools, specifically, Eclipse's cleanup operation,
      to tasks such as minimizing variability as planting \kk{final} modifier
      whenever possible, removing unused variables and methods, removing
      unnecessary parenthesis, etc.
    \item user intervention for appropriate
        clever method extraction, and other modular restructuring to
        minimize e.g., exploitation of scope.
\end{itemize}
Naturally, size increases with the addition of \cc{final}s; other cleanups, such
as the automatic addition of \cc{@Override} annotation may increase size as well.
Also, typically method extraction and other modular transformations increases
code size, be it measured in lines, tokens, characters or bytes.

Thus, there are four components to the spartanization process:
\emph{\textbf{(i)}} structural,
\emph{\textbf{(ii)}} nominal,
\emph{\textbf{(iii)}} cleanup, and,
\emph{\textbf{(iv)}} modular.
Current technology can automate all but the last component;
automation of method extraction requires more research,
paying attention to current state of the art \matteo~\cite{%%
    we should have many citations
    somewhere can you please add some citations
}.
This section is concerned with the full process of spartanization.
The empirical \cref{section:initial} below investigates how structural and
nominal spartanizations contribute to the naturalness of code.

\subsection{Extent of application}
\label{section:extent}
An underlying principle is not only that the minimization of
the above ends matters, but also that \emph{nothing else matters}.
In other words, wrings are applied repeatedly, with explicit disregard
to any other extent†{%
  In the Spartan's words: ‟\textit{As far as this can reach}.” K. Agesilaus the
  great (\textsl{being asked for the extent of Sparta's border, while
  holding his spear}).
}.

Heavily spartanized code is often quite different than the original. A case in
point can be found in generic class \cc{C0}, defined by the programmer to
include only instance field \cc{inner} of the generic type parameter.

\begin{figure}[H]
  \caption{A generic class representing a cell, along with its Eclipse
  automatically generated methods (28 lines, 94 words, and 649 characters).}
    \label{figure:cell0}
    \begin{adjustbox}{max width=\columnwidth}
% VIM: /begin/+,/end/-!wc
\begin{code}[minipage, width=1.25\columnwidth]{JAVA}
public class C0<T> {
  private T inner;
  public C0(T inner) {
    super();
    this.inner = inner;
  }
  public int hashCode() {
    final int prime = 31;
    int result = 1;
    result = prime * result + ((inner == null) ? 0 : inner.hashCode());
    return result;
  }
  public boolean equals(Object obj) {
    if (this == obj)
      return true;
    if (obj == null)
      return false;
    if (getClass() != obj.getClass())
      return false;
    C0 other = (C0) obj;
    if (inner == null) {
      if (other.inner != null)
        return false;
    } else if (!inner.equals(other.inner))
      return false;
    return true;
  }
}
\end{code}
\end{adjustbox}
\end{figure}

\Cref{figure:cell0} portrays this class along with function members
automatically generated by Eclipse: a constructor and the trivial methods
\cc{equals($·$)} and \cc{hashCode()} methods.

Compare this figure to its spartanized terse version in \cref{figure:cell1}.

\begin{figure}[H]
  \caption{A spartanized version of the \Java class in \cref{figure:cell0}
    automatically generated methods (16 lines, 70 words and 424 characters).}
    \label{figure:cell1}
    \begin{adjustbox}{max width=\columnwidth}
% VIM: /begin/+,/end/-!wc
\begin{code}[minipage, width=1.25\columnwidth]{JAVA}
public class C1<T> {
  private final T inner;
  public C1(T inner) {
    this.inner = inner;
  }
  public int hashCode() {
    return 31 + ((inner == null) ? 0 : inner.hashCode());
  }
  public boolean equals(Object ¢) {
    return ¢ == this || //
      ¢ != null && getClass() == ¢.getClass() && equals((C1) ¢);
  }
  private boolean equals(C1 ¢) {
    return inner == null ? ¢.inner == null : inner.equals(¢.inner);
  }
}
\end{code}
\end{adjustbox}
\end{figure}

Evidently, spartanization has removed all conditionals, replacing these with
the ternary operator and the short-circuit operators, ‟\cc{\textbar\textbar}”
and ‟\cc{&&}”. Spartanization also abbreviated \cc{obj} to~\cc{o} and then
renamed it to~\cc{¢}, eliminated the spurious \cc{\kk{super}()} call in the
constructor, as well as the two variables use in \cc{hashCode()}, turning the
function into a single expression.

The resulting code is evidently shorter, and not as nested. Arguably, the code
is also easier explain and understand: For example, function \cc{equals(Object
¢)} in the figure is described by its own code, while relying on the
understanding of the Boolean operators

\begin{quote}\itshape\scriptsize
  If the sent parameter is the same as \cc{this}, then equality is guaranteed.
  Otherwise, the parameter must not be \kk{null}, and must be of the same class
  of \kk{this}. Further, the parameter must also be of the same as class as
  \kk{this}, and equal to \kk{this} also as an instance of this class.
\end{quote}

Notice though that line length, which is not optimized by the wrings, suffered
a bit. We do have fewer lines, but the average number of characters in them
increased from 23 characters to 26.5. Also, the maximal line length increased
from 71 characters to 81. This last increase is due though due to the
extraction of a \kk{private} version of \cc{equals($·$)}, which is carried
out manually, at user's discretion.

\subsection{Automation}
\label{section:automation}
Consider the \Java program presented in \cref{figure:eclipse}, drawn from the
Eclipse JDT Tutorial series\urlref{%
http://www.programcreek.com/2011/01/a-complete-standalone-example-of-astparser},
but slightly amended for brevity.

\begin{figure}[H]
  \caption{\label{figure:eclipse}%
    A complete standalone example of using class \cc{ASTParser}, drawn
    from the Eclipse JDT Tutorial Series, abbreviated and reformatted.}
    \begin{adjustbox}{max width=\columnwidth}
      \begin{code}[minipage, width=1.13\columnwidth]{JAVA}
public class Test0 { // comments and £\kk{import}£ directives omitted for brevity
  public static void main(String args[]) {££
    ASTParser parser = ASTParser.newParser(AST.JLS3);
    parser.setSource((
      "public class A {\n" +
      " int i = 9;\n"+ //
      " int j;\n" + //
      " ArrayList<Integer> al = new ArrayList<Integer>();\n" + //
      " j=1000;\n" + // sic (syntax error in the original)
      "}\n").toCharArray());
    parser.setKind(ASTParser.K£\_£COMPILATION£\_£UNIT);
    CompilationUnit cu = (CompilationUnit) parser.createAST(null);
    cu.accept(new ASTVisitor() {££
      Set names = new HashSet();
      public boolean visit(VariableDeclarationFragment node) {££
        SimpleName name = node.getName();
        this.names.add(name.getIdentifier());
        System.out.println(
          "Declaration of '" + name + "' at line" + //
          cu.getLineNumber(name.getStartPosition()));
        return false;
      }
      public boolean visit(SimpleName node) {££
        if (this.names.contains(node.getIdentifier())) {££
          System.out.println(
            "Usage of '" + node + "' at line " +//
            cu.getLineNumber(node.getStartPosition()));
        }
        return true;
      }
    });
  }
}
\end{code}
  \end{adjustbox}
\end{figure}

\Cref{figure:eclipse:automatic} shows the automatically spartanized version of
the this code.

\begin{figure}
  \caption{\label{figure:eclipse:automatic}%
    An automatically spartanized version of
      the \Java program of \cref{figure:eclipse}.
    }
  \begin{adjustbox}{max width=\columnwidth}
\begin{code}[minipage, width=1.22\columnwidth]{JAVA}
public class Test1 { // comments and £\kk{import}£ directives omitted for brevity
  public static void main(final String[] £\_\_£) {££
    final ASTParser p = ASTParser.newParser(AST.JLS3);
    p.setSource((
      "public class A {\n" +
      " int i = 9;\n"+ //
      " int j;\n" + //
      " ArrayList<Integer> al = new ArrayList<Integer>();\n" + //
      " j=1000;\n" + // sic (syntax error in the original)
      "}\n").toCharArray());
    final CompilationUnit u = (CompilationUnit) p.createAST(null);
    u.accept(new ASTVisitor() {££
      Set £\ignore$£$ = new HashSet();
      @Override public boolean visit(final SimpleName ¢) {££
        if (£\ignore$£$.contains(¢.getIdentifier()))
          System.out.println("Usage of '" + ¢ + //
            "' at line " + u.getLineNumber(¢.getStartPosition()));
        return true;
      }
      @Override public boolean visit(final VariableDeclarationFragment f) {££
        final SimpleName ¢ = f.getName();
        £\ignore$£$.add(¢.getIdentifier());
        System.out.println("Declaration of '" + ¢ + //
          "' at line" + u.getLineNumber(¢.getStartPosition()));
        return false; // do not continue to avoid usage info
      }
    });
  }
}
\end{code}
  \end{adjustbox}
\end{figure}
Each of the several wrings applied to obtain \cref{figure:eclipse:automatic}
  reduced at least one of our minimization ends.
At several occasions, cleanup increased these.
Overall, the result of automatic application of spartanization
  is a reduction in the number of lines, words, and characters
  by factors of~$1.28$,~$1.05$, and~$1.24$, respectively.†{Part of this improvement
    is due to cleanup operation that optimizes the use of \kk{import}; our empirical
  study ignores \kk{import}s.}
Evidently spartanization shrinks even pedestrian code such as that of
\cref{figure:eclipse}.

Follow the example to appreciate the automatic spartanization process involved:

\begin{enumerate}
  \item \emph{Cleanup}, adding \kk{final}s and \cc{@Override} annotations,
      and simplifying the convoluted reference to data member \cc{names}.
    \item \emph{Structural spartanization}, whose operation was limited to the
      removal of one pair of redundant curly brackets.
    \item \emph{Nominal spartanization}, including these operations:
      \begin{description}
        \item[N1] Renaming of the unused
      parameter \cc{args} to \cc{main($·$)} to the spartan name
      \cc{\_\_}, which means ‟unused” in reminiscence of \Prolog's~‟\cc\_”
      name of unused variables.
    \item[N2] Renaming certain variables to~\cc{¢}. The
      spartan name~\cc{¢} is for the most imminent argument or manipulated
      entity that is not \cc{this}, and reminiscent of \kk{it} of \ML and
        \matteo HyperCard~\cite{there:must:be:some:citation:in:our:huge:bib:repository}:
        the variable which is so prevalent in the current piece of code that
        naming it would be waste of words. Stated differently, the
        meaning of a variable named~\cc{¢} should be so clear to the reader that
        reader, that naming it would be a waste of words.

      \item[N3] Renaming variable \cc{parser} to~\cc{p}, and variable~\cc{cu}
        to~\cc{u}. In accordance with the \emph{Linux kernel coding style}%
        \urlref{https://www.kernel.org/doc/Documentation/CodingStyle}
        and other style guides nominal spartanization abbreviates short scoped
        variables: locals and parameters to one letter that uniquely identifies
        their type. For example, type \kk{int} is denoted by~\cc{i} and the
        name of short scoped, variable \cc{int index;} is abbreviated to~\cc{i}.

        More generally, the single letter denoting a long type name is defined
        as the first letter in the last word of the component of the type name.
        Variable \cc{parser} of type \cc{ASTParser} is hence renamed~\cc{p},
        and~\cc{u} is new name of variable \cc{cu} of type \cc{CompilationUnit}.

        Abbreviation is limited to only those cases in which one letter
        uniquely identifies the variables and there are not more than four of
        these. Experience shows that only a small minority of the methods is
        excluded by this limitation.

\end{description}
  \end{enumerate}

\Cref{figure:primality}, drawn from a spartan implementation of
utility services\urlref{%
  https://github.com/SpartanRefactoring/spartan/blob/master/src/il/org/spartan/misc/Primes.java},
  further demonstrates \textbf{N3}. In the inner loop of the test for
  primality, there is no need to spend words and mental effort to name the
  candidate divisor. It is sufficient to refer to it is by the spartan name
~\cc{¢}, and no confusion should arise.

\begin{figure}
\caption{\label{figure:primality}
  Spartan functions to test for primality†{https://github.com/SpartanRefactoring/spartan}
 }
\begin{Code}{JAVA}{il.org.spartan.misc.Primes#isPrime}
/** Tests for primality.
  * @param ¢ candidate to be tested
  * @return <code><b>true</b></code> <i>iff</i> the parameter is prime. */
public static boolean isPrime(final int ¢) {££
  return ¢ < -1 && isPrime(-¢) // deal with negative values
      || ¢ > 1 && isPrime¢(¢); // any integer >= 2
}
private static boolean isPrime¢(final int i) {££
  assert i >= 2;
  for (int ¢ = 2; ¢ * ¢ <= i; ++¢)
    if (i % ¢ == 0)
      return false;
  return true;
}
\end{Code}
\end{figure}

Note that the single parameter to function \cc{isPrime($·$)} can
safely named ‟\cc{¢}”, since the scope does not include
any other bindings.

In contrast, the single parameter to auxiliary \kk{private} function
\cc{isPrime¢($·$)} must be named ‟\cc{i}”, since its scope includes yet another
binding, of some name to the loop variable. Confusion does not rise in naming
the loop variable ``\cc{¢}'' since the name \emph{always} refers to the most
recently defined variable, i.e., the one that occurs last in the most inner
lexical scope.

\Cref{figure:free}, drawn from a spartan implementation of
\Prolog\urlref{https://github.com/yossigil/Protolog}, demonstrates one more kind of
nominal spartanization:
\begin{description}
\item[N4] employ the spartan name~‟\cc{\$}” for
  the \emph{single} variable that represents the
  function's result.†{the name ‟\cc{\$}” is borrowed from many DSLs,
    particularly of regular expressions and languages, in which it denotes the
  concept of termination.}

  The automatic wring heuristic for identifying the ‟result variable”
  by running a tournament between legible candidates result variable,
  abstaining in cases of ties and vacuous matches.
\end{description}

\begin{figure}
    \caption{\label{figure:free}%
      Spartan function to test for the existence of free variables in a
      symbolic-expression, such as those of \protect\Prolog.
      }
    }
\begin{Code}{JAVA}{\scriptsize\texttt{il.org.spartan.protolog.Variable#freeVariables}}
default Set<Variable> freeVariables() {££
    final Set<Variable> £\ignore$£$ = new LinkedHashSet<>();
    if (isVariable())
     £\ignore$£$.add((Variable) this);
    else
      for (final Term ¢ : this)
       £\ignore$£$.addAll(¢.freeVariables());
    return £\ignore$£$;
  }
\end{Code}
\end{figure}

The code can be
further polished and spartanized by method extraction and other refactoring.
For now, we concentrate on the automatic spartanization, which as explained
above, comes in
two main varieties.


\section{The Scary Spartan Look}
\label{section:look}
Don't be scared warn the readers!
The spartan look could be scary.
examples of \cc{hashCode} \cc{equals}

add few examples perhaps from spartan library or plugin. 

Fig. \ref{fig:code-before-spartanization} shows an example of class 
that present some issues or, put in other words, spartanization opportunity. 





\section{Ripping Results}
\label{section:initial}
% \matteo: can you please proof read this section?
% \matteo: uncomment if you are done.

Having described the spartanization style and process, we now turn to the
question of evaluation.  Our hypothesis here is that both structural and
nominal spartanization make the code more regular. The conjecture is that
spartanization should: preserve semantics, reduce code size, and making the
code \emph{more compressible}. The challenge is in meeting the conjunction of
the three conditions in the conjecture:
\begin{itemize}
    \item The removal of all code decreases size, but does not preserve semantics.
    \item Compression by a tool such as GZip, reduces size, preserves
      semantics. However, compression power is certain to suffer. It is 
      not possible to further compress \cc{.zip} files, whereas plain code
      compresses typically in a factor of two or more.
    \item Another operation which preserves semantics while making
      the code smaller is the reducing runs of whitespace characters combined 
      with all spaces that surround non-alphabetical operators. 
      However, since white spaces are used so frequently in the code,
      this action is likely to make the result less compressible than the origin. 
    \item Even the plain removal of curly brackets in singleton blocks as
      recommended by one of the battling camps of the braces war, is not guaranteed
      to improve compressibility. A good predictor is likely to discover that a
      an opening curly bracket always occurs after the conditional of an \kk{if}. 
      It might become less efficient when this character bracket does not always occur. 
\end{itemize}

Of the four components of the process, we are most interested in structural and
nominal spartanization. The other two components are left beyond the scope of
this work.  Cleanup for spartanization is a bit less interesting, since it is
not particular to spartanization.  Upon automation of method extraction, future
research is to evaluate the quality of these. 

As shall be revealed below, there is a strong empirical evidence to support
the conjecture, when limited to structural sparnization. The results regarding
nominal spartanization are inconclusive, and the reduction we witness in 
compressibility is not statistically significant. 

From the perspective of  language software engineering, the main research
question that concerns us here can be phrased as follows:
\begin{description}
  \item[RQ1.] \emph{Is spartan code more natural than non-spartanized code?} also,
        and more concretely,
        \begin{description}
          \item[RQ1.1.] \emph{Does structural spartanization contribute software naturalness?}
          \item[RQ1.2.] \emph{Does renaming spartanization contribute software naturalness?}
        \end{description}
\end{description}
Yet another question that intrigues us here is:
\begin{description}
  \item[RQ2.] Find an approximation of software's
        naturalness~\cite{Hindle:Bar:Su:Gabel:Devanbu:12} suitable
        for experiments with spartan code.
\end{description}

\subsection{Data Corpus}
We used the ‟Gil-Lalouche”~\cite{Gil:Lalouche:16} software corpus,
assembled from~26 popular \Java open source software artifacts collected
from~\emph{GitHub's Trending
  repositories}\urlref{https://github.com/trending?l=java&since=monthly} and
the \emph{GitHub \Java Corpus}%
\urlref{http://groups.inf.ed.ac.uk/cup/javaGithub/}
list due to Allamanis and Sutton~\cite{Allamanis:Sutton:13}.

The ‟Size Before” column of \Cref{table:corpus} provides the chief size
statistics of the constituting artifacts.

\begin{table}
  \caption{\label{table:corpus}%
    Aggregating statistics, over artifacts in the corpus,
  of compression power of BZip2 and size, before and after compression.}
  \par\vspace{10pt plus 6pt minus 4pt}
  \centering
  \begin{adjustbox}{max width=\columnwidth}
    \scriptsize
    \begin{tabular}{l*3r}
      \toprule
      & \multicolumn{2}{c}{\textit{Size (bytes)}}⏎
      \cmidrule(r){2-3}
      & \textit{Before}
      & \textit{After}
      & \multicolumn1c{\textit{Power}}⏎
      \midrule % VIM: +,/bottom/-!column -t|sed 's/^/ /'
      \sffamily  Ave.  & 12,370,223 & 1,272,331 & 9.25⏎
      \sffamily  Min.  & 1,019,125  & 129,114   & 7.22⏎
      \sffamily  Max.  & 46,965,422 & 5,607,599 & 12.96⏎
      \sffamily  Med.  & 5,128,541  & 607,535   & 8.91⏎
      \sffamily  Range & 45,946,297 & 5,478,485 & 5.74⏎
      \bottomrule
    \end{tabular}
  \end{adjustbox}
\end{table}

Notably, sizes†span more than one
order of magnitude, from one megabyte of source to circa~47, the median being
around five megabytes, the average around twelve. The GL corpus was used before
for the study of software metrics.  See~\cite{Gil:Lalouche:16,Gil:Lalouche:16b}
for reproducibility details and for a description of the artifacts' selection
process. Sufficient to us here is that the selection was independent of the
current research and that the artifacts are of considerable size, wide use,
evolution history and, development effort.

\subsection{Burrows-Wheeler Compression}
Compression power is used in this work-in-progress report as a quick estimate
of software's naturalness~\cite{Hindle:Bar:Su:Gabel:Devanbu:12}. The measures
are similar but not identical: Compression algorithms search (and exploit) the
input for repetitive patterns of unbounded length, while naturalness limits the
search to~$n$-grams for some small integer~$n$. The difficulty with applying
vanilla naturalness measures is that they become prohibitively slow
for~$n>5$~(say). Alas, patterns of spartanizations, just like refactoring,
often manipulate many more than~$5$ tokens.

Quality in compression algorithms is measured by \emph{compression power},
defined as the ratio of uncompressed size to the compressed size. Compression
power, our \emph{approximate naturalness} should be correlated with
\emph{original naturalness}, but it distincts from it.

We found that with respect to \Java software the compression power of BZip2 is
greater than that of GZip. This greater power is probably because Gzip make a
greedy search for repetitions starting at the input's prefix, while BZip2
starts the search in a broader context, as implied by the Burrows-Wheeler
algorithm~\cite{Burrows:Wheeler:94}.

The second and third columns of \cref{table:corpus} provide statistics of
compressed size and compression power of BZip2\urlref{http://bzip.org/}. As in
the study of naturalness of software, we witness the fact that projects can be
very different. There is two fold range variation of their associated
compression power, as there is in their naturalness.

\subsection{The Jack Preprocessor}

Even though software artifacts often includes more than just \cc{.java} files,
size and other data are of these files only. The rationale is that we wish to
concentrate on the naturalness of the \Java code itself, rather auxiliary
material such as configuration information. 
Note that \cc{.java} files often contain more text than just plain code.  To
better focus on the code itself, this study employs \emph{Jack}, a special
purpose \Java preprocessor applied prior to compression.

Jack replaces each keyword and operator with a single one byte token. The
rationale is that keywords such as \kk{class} are read and produced by
programmers as an individual cohesive token, rather than a sequence of letters
that form an identifier, or sequence of digits that make a numeric literal.

Jack also eliminates from the \Java code all of the following five: white space
characters, \kk{package} declaration, \kk{import} directives, comments of all
sorts, and, the body of all string literals. The assumption here is that each
of these is not a product of pure coding effort: White space characters are
generated and frequently change by automatic code formatting. Also, \kk{import}
directives are generated and optimized automatically, and are subject to
underlying project guideline. Comments also follow their own style, which is
very different from that of the code, \emph{and}, string literals, at least
those with significant content, are ought to be managed by configuration and
internationalization processes rather than coding per-se.
Jack currently does not remove unit-tests, but is planned to do so in the
future.

\Cref{table:virgin} presents the compression power of Jack, BZip, when
applied after Jack, and the accumulative compression power of Jack and BZip2
when applied in this order.
\begin{table}
  \caption{\label{table:virgin}%
    Aggregating statistics, over artifacts in the corpus,
    of size and compression power of Jack and Jack combined with BZip2 
    relative to the original software.
  }
  \par\vspace{10pt plus 6pt minus 4pt}
  \centering
  \begin{adjustbox}{max width = \columnwidth}
    \begin{tabular}{l*5r}
      \toprule
      & \multicolumn2c{\textit{Jack}}
      & \multicolumn2c{\textit{BZip2 after Jack}}
      & \multicolumn1c{\textit{Jack+BZip2}}⏎
      \cmidrule(r){2-3} \cmidrule(r){4-5} \cmidrule(r){6-6}
      & \textit{Size (bytes)}
      & \textit{Powr}
      & \textit{Size (bytes)}
      & \textit{Powr}
      & \textit{Power}⏎
      \midrule % VIM: +,/bottom/-!column -t|sed 's/^/ /'
      \sffamily  Ave.  & 5,078,034  & 2.78  & 729,934   & 6.60 & 18.37⏎
      \sffamily  Min.  & 456,050    & 1.73  & 77,706    & 4.89 & 10.18⏎
      \sffamily  Max.  & 19,697,634 & 12.85 & 3,414,889 & 8.68 & 83.65⏎
      \sffamily  Med.  & 1,822,510  & 2.34  & 343,815   & 6.50 & 15.95⏎
      \sffamily  Range & 19,241,584 & 11.12 & 3,337,183 & 3.79 & 73.47⏎
      \bottomrule
    \end{tabular}
  \end{adjustbox}
\end{table}
In the table, we can observe that Jack typically compress files to half their
size, although there are some artifacts (with many \kk{import}s and comments),
in which the removal of boilerplate material from the program make a ten fold
reduction in size.

The combined compression power of Jack and BZip2 reaches greater values,
ranging from circa 10 fold compression to circa 80 fold. However, the more
% With regards to the Jack algorithm for the spartanized code, we can see that the statistics is
% lightly worse that in the original code. 
% It means that Jack is not able to compress spartanized code better than in the case of the original
% code. 
% It something expected since Spartanization removes

% which reports the difference 
% in the aggregatint statistics for the Jacked code and for the It is
% worth to note that we have an improvement in any (???) statistics, in other words the
% spartanization affected positively the compression, increasing the compression
% power. This is counter intuitive, being a small file most difficult to
% compress than a larger one (REFORMULATE)
interesting measure is the ``BZip2 after Jack'' compression power. This measure is
our \emph{approximate naturalness}, which answers \textbf{RQ2}. Approximate
naturalness is telling how much the essence of the code itself, discarding
boilerplate is compressible (and hence predictable, or, approximately natural).

Comparing the penultimate column (``Powr'') of the \cref{table:virgin} 
with the final column (``Power'') of
\cref{table:corpus}, we see that overall, Jack reduces the compression power of
BZip2. All five statistics: average, min, max, median, and range, are higher
when BZip2 is applied to the original code, than if it is applied to the Jacked
code. In other words, approximate naturalness of non-filtered code is better
than that of jacked code. This should not be a surprise: there are good reasons
to assume that compression of comments, strings, and boilerplate \kk{import}s,
is better than that of the code itself. If these are eliminated from the input,
then compression power is expected to decrease.

\subsection{Structural Spartanization and Approximate Naturalness}

In the first set of experiments, designed to answer \textbf{RQ1.1}, \Java code
in the software artifacts were subjected to our automatic spartanization
tool, configured to employ structural wrings. 
The results are reported in \cref{table:structural} where are reported 
the compression power of the Jacked code and its approximate 
naturaleness (Jack+BZip2). 
We compared these results % approximate naturalness of the automatically spartanized code 
with those related to the original code.
We specifically subtract the aggregating statistics of the latter with those related to the former.
\cref{table:difference} reports the outcome of this comparison. 
% of the aggragating statistics for the latter, related to 
% the spartanized code, and the compression power of the original code. 

The first column of \cref{table:difference} regards the application of the Jack algorithm to the spartanized code. 
As we can see some of the results are negative.
For example, the average value of the compression power is negative, meaning that the Spartanized Jacked code 
il less compressed than the original.
However we are not considering a significative difference.
Also the Max value is negative. As a matter of fact the Max value for the Spartanized code is 
significantly different from the corresponding value in the original, being the latter greater.
This might be due to the fact that there could be case that Spartanized code 
tend to be more laconic, with many variables removed, and code inlined. 
When Jack algorithm removes the tokens that are repeated several times in a .java file 
(i.e. modifiers: abstract, public, private; import
instructions; terms like extends, implements, try, catch).
From this point of view, boilerplate code is less repetitive
% meaning that the spartanized jacked code is less compressed 
% than the Jacked original one. 
% This can be explained if we take into account the fact that Spartanization removes some 
% of the elements that are removed also by Jack.

The second column of the same Table regards the approximate naturalness of the 
spartanized jacked code.  All difference are positive, implying the
answer to \textbf{RQ1.1}: approximate naturalness of a structurally spartanized code
is higher than that of the original.

\begin{table}
  \caption{Aggregating statistics of compression power of Jack+BZip2 after
  automatic \textbf{structural} spartanization, compared with non-spartanized code. 
  }
  \label{table:structural}
  \par\vspace{10pt plus 6pt minus 4pt}
  \centering
  \begin{adjustbox}{max width = \columnwidth}
    \begin{tabular}{l*6r}
      \toprule
      & \multicolumn2c{\textit{Jack}}
      & \multicolumn2c{\textit{Jack+BZip2}}
      & \multicolumn2c{\textit{Combined}}\\
      \cmidrule(r){2-3} \cmidrule(r){4-5} \cmidrule(r){6-6}
      & \textit{Size (bytes)}
      & \textit{Powr}
      & \textit{Size (bytes)}
      & \textit{Powr}
      & \textit{Power}⏎
      \midrule % VIM: +,/bottom/-!column -t|sed 's/^/ /'
      \sffamily  Ave.  & 5,050,647  & 2.46 & 719,488   & 6.66 & 0.06  & 1.01 \\
      \sffamily  Min.  & 439,825    & 1.79 & 75,391    & 4.90 & -0.37 & 0.21 \\  
      \sffamily  Max.  & 19,173,224 & 3.87 & 3,275,918 & 8.79 & 0.49  & 1.07 \\
      \sffamily  Med.  & 1,959,917  & 2.41 & 341,851   & 6.79 & 0.86  & 0.87 \\
      \sffamily  Range & 18,733,399 & 2.08 & 3,200,527 & 3.89 & 0.08  & 0.08 \\
      \bottomrule
    \end{tabular}
  \end{adjustbox}
\end{table}

\begin{table}
  \caption{%
  Difference between the aggregating statistics for the compression power of the spartanized and the original code.
  }
  \label{table:difference}
  \par\vspace{10pt plus 6pt minus 4pt}
  \centering
  \begin{adjustbox}{max width = \columnwidth}
    \begin{tabular}{l*3r}
      \toprule
      & \multicolumn1c{\textit{Jack}}
      & \multicolumn1c{\textit{Jack+BZip2}}
      & \multicolumn1c{\textit{Combined}}⏎
%       \cmidrule(r){2-3} \cmidrule(r){4-5} \cmidrule(r){6-6}
%       & \textit{Size (bytes)}
      & \textit{Powr}
%       & \textit{Size (bytes)}
      & \textit{Powr}
      & \textit{Power}⏎
      \midrule % VIM: +,/bottom/-!column -t|sed 's/^/ /'
\sffamily  Ave.  & -0.32 & 0.06  & -1.95  \\
\sffamily  Min.  &  0.06 & 0.01  &  0.40  \\  
\sffamily  Max.  & -8.98 & 0.11  & -55.46 \\
\sffamily  Med.  &  0.07 & 0.29  &  0.62  \\
\sffamily  Range & -9.04 & 0.10  & -55.86 \\
      \bottomrule
    \end{tabular}
  \end{adjustbox}
\end{table}

\begin{table}
  \caption{CHECK ME I think only last two columns make sense: Aggregating
  statistics of compression power of Jack+BZip2 after automatic structural
spartanization, compared with non-spartanized code.}
  \label{table:structural-comparison}
  \par\vspace{10pt plus 6pt minus 4pt}
  \centering
  \begin{adjustbox}{max width=\columnwidth}
    \begin{tabular}{l*6r}
      \toprule
      & \multicolumn2c{\textit{Size (bytes)}}
      & \multicolumn2c{\textit{Powr}}
      & \multicolumn1c{\textit{Improvement}}\\
%       \cmidrule(r){2-3}\cmidrule(r){4-5} \cmidrule(r){6-6}
      & \textit{Jack}
      & \textit{Jack+BZip2}
      & \textit{BZip2 on Jack}
      & \textit{Combined}
      & \textit{with. Spartan}
      & \textit{total with partial}\\
      \midrule % VIM: +,/bottom/-!column -t|sed 's/^/ /'
      \sffamily  Ave.  & 5,050,647  & 719,488   & 6.66 & 15.88 & 16.42 & 1.01 \\
      \sffamily  Min.  & 439,825    & 75,391    & 4.90 & 10.58 & 10.58 & 0.21 \\  
      \sffamily  Max.  & 19,173,224 & 3,275,918 & 8.79 & 23.21 & 28.19 & 1.07 \\
      \sffamily  Med.  & 1,959,917  & 341,851   & 6.79 & 12.63 & 16.57 & 0.87 \\
      \sffamily  Range & 18,733,399 & 3,200,527 & 3.89 & 15.27 & 17.61 & 1.04 \\
      \bottomrule
    \end{tabular}
  \end{adjustbox}
\end{table}

\begin{table}
  \caption{\label{table:structural2}%
    Aggregating statistics of compression power of Jack+BZip2 after
  automatic structural spartanization.}
  \par\vspace{10pt plus 6pt minus 4pt}
  \centering
  \begin{adjustbox}{max width = \columnwidth}
    \begin{tabular}{l*5r}
      \toprule
      & \multicolumn2c{\textit{Jack}}
      & \multicolumn2c{\textit{BZip2 after Jack}}
      & \multicolumn1c{\textit{Jack+BZip2}}⏎
      \cmidrule(r){2-3} \cmidrule(r){4-5} \cmidrule(r){6-6}
      & \textit{Size (bytes)}
      & \textit{Powr}
      & \textit{Size (bytes)}
      & \textit{Powr}
      & \textit{Power}⏎
      \midrule % VIM: +,/bottom/-!column -t|sed 's/^/ /'
      \sffamily  Ave.  & 5,050,647  & 6.66 & 719,488   & 15.88 &  \\
      \sffamily  Min.  & 439,825    & 4.90 & 75,391    & 10.58 &  \\  
      \sffamily  Max.  & 19,173,224 & 8.79 & 3,275,918 & 23.21 &  \\
      \sffamily  Med.  & 1,959,917  & 6.79 & 341,851   & 12.63 &  \\
      \sffamily  Range & 18,733,399 & 3.89 & 3,200,527 & 15.27 &  \\
      \bottomrule
    \end{tabular}
  \end{adjustbox}
\end{table}

\subsection{Renaming Spartanization}
\begin{table}
  \caption{\label{table:total}%
    Aggregating statistics of compression power of Jack+BZip2 after
  automatic structural+renaming spartanization, compared with non-spartanized code.}
  \par\vspace{10pt plus 6pt minus 4pt}
  \centering
  \begin{adjustbox}{max width=\columnwidth}
    \begin{tabular}{l*5r}
      \toprule
      & \multicolumn{2}{c}{\textit{Size (bytes)}}
      & \textit{Power}
      &\multicolumn{2}{c}{\textit{Improvement over}}⏎
      \cmidrule(r){2-3} \cmidrule(r){5-6}
      & \textit{Jack}
      & \textit{Jack+BZip2}
      & & \textit{Overall}
      & \textit{Partial Spartan.}⏎
      \midrule % VIM: +,/bottom/-!column -t|sed 's/^/ /'
      \sffamily  Ave\@. & 4,947,319  & 713,857   & 6.56 & -0.03 & -0.09⏎
      \sffamily  Min\@. & 430,375    & 75,160    & 4.89 & -0.40 & -0.28⏎
      \sffamily  Max\@. & 18,890,899 & 3,244,851 & 8.72 & 0.41  & -0.01⏎
      \sffamily  Med\@. & 1,923,056  & 339,930   & 6.70 & 0.00  & -0.09⏎
      \sffamily  Range  & 18,460,524 & 3,169,691 & 3.83 & 0.81  & 0.27⏎
      \bottomrule
    \end{tabular}
  \end{adjustbox}
\end{table}

% Preliminary results supporting the
% ‟Hypothesis that can never be proved”
% Because this is not what you like.


% \section{Discussion}
\label{section:zz}
We believe that the new metrics we suggested can provide a fresh take on the field of complexity metrics, which has been relatively quiet for the last two decades.
While the metrics' theoretical bases differ, that is no assurance that they will not be highly correlated to each other or to other metrics such as lines of code.
Moreover, it is important to find the metrics' capability in predicting errors.
We will answer both of these questions will in our research.

\balance
\bibliographystyle{abbrv}\small
\bibliography{%
\jobname,%
other-shorthands,%
author-names,%
publishers-abbreviated,%
conferences-abbreviated,%
journals-abbreviated,%for
journals-full,%
yogi-tr,%
yogi-book,%
yogi-practice,%
yogi-misc,%
yogi-journal,%
yogi-theory,%
yogi-confs,%
GPCE,%
OOPSLA,%
PLDI,%
USENIX,%
ECOOP%
}

\appendix
\section{Main Techniques of Butchering}
\label{section:techniques}
Spartanization aims at minimizing:
\begin{itemize}
 \item Vertical complexity: the number of lines of code.
 \item Horizontal complexity.
 \item Characters, tokens, parameters and variable.
 \item Loops.
 \item Conditional statements.
\end{itemize}

Spartan programming prescribes, a careful use of the variables, stressing the need
to inline variables that are used only once and emphasizing the use of foreach loops
in languages implementing it. Variables visibility and accessibility should also be
possibly minimized, favoring the use of private over public variables whenever it
is possible.

Also the names of variables should be taken short for local variables, with short scope,
leaving to their class names to explicit their behavior. Minimizing variability using
final and @NonNull annotation. Stress the use of generic names technique. Minimize lifespan
avoiding persistent variables when it is possible. Favor the use of collection instead
of arrays. Interfaces should present a minimal number of parameters, also minimizing
the interaction between them. Favor ternarization over the use of if-then-else constructs
and use a simplified version of the conditionals that implies an early return. Moreover,
use an early exit logic with continue, break and return instructions.

Everything should be formatted in order to make an efficient use of the screen space,
by removing unnecessary tokens (i.e., redundant parentheses, etc.).
Methods should be as short as possible.

\subsection{Syntactic Baggage}
A set of rules are aimed at removing redundants tokens (i.e. brackets, modifiers, etc.).
\begin{itemize}
  \item Redundant parenthesis and curly brackets.
  \item Removing redundant modifiers:
  \begin{itemize}
    \item \kk{final} in private methods.
    \item \kk{final} from static methods.
    \item \kk{abstract} before interface.
    \item \kk{abstract} before methods in interfaces.
    \item \kk{final} from all methods in final classes.
    \item \kk{private} and \kk{static} from everything defined in anonymous classes.
  \end{itemize}
  \item Removing visibility of all members of an interface
  \item Extra super() calls.
\end{itemize}

\subsection{Shorter phrases}
\begin{itemize}
  \item Distributive rule of booelan expressions
  \item Distributive rule of assignment: 
  \begin{code}{JAVA}
a = x; 
b = x;   
  \end{code}
  becomes
    \begin{code}{JAVA}
a = b = x;   
  \end{code}
  \item Distributive rule of arithmetics.
  \item Common prefix of If
  \item Common Suffix of If
  \item Pull out arithmetical negation
  \item Use ternary instead of conditionals.
  \item convert prefix ++ and--into postfix when possible
  \item Example:
\begin{code}{JAVA}
x.toString()
\end{code}
\begin{code}{JAVA}
"" + x
\end{code}
\end{itemize}

\subsection{Canonical Form of Common Expressions}
Use canonical form for the most common expressions, as much as possibile.
\begin{itemize}
  \item Follow patterns: Pattern: A*x + B
  \item Sorting by size, but also as per the previous rule.
  \item De morgan pushdown logical negation.
  \item Pullup arirthmetic negation.
  \item Pushdown ternarizations. For example:
%   \item Example:
  \begin{code}{java}
a ? b(x) : b(y) => b(a ? x: y)
  \end{code}
  \item Apply the distributive rule on booleans. For instance:
%   \item Example:
    \begin{code}{JAVA}
a && b || c && b => (a || c) && c
    \end{code}
      This is one variant (there are other three).
  \item More specific first: How should you order X == Y.
        Rule: you start from the more specific and increase specifity,.
        I.e., S(X) > S(Y)
        \begin{enumerate}
          \item null/this/true/false
          \item 0, 1, ""
          \item 12, 13.4
          \item Classes: public, protected, package, private
          \item Fields: Static fields, Fields,
          \item parameters,
          \item local variables ordered by scope.
          \item ¢
        \end{enumerate}
\end{itemize}

\subsection{Names}
With regards to the naming, Spartan Programming applies three principles:
\begin{itemize}
 \item Generic names policy.
 \item Use short names for return values.
 \item Use short anonymous names.
\end{itemize}
Generally speaking, variable names should be chosen based on their scope.
The shorter the scope the shorter the variable name.
% Scope plays a rule:
% in short scopes use short names.
Some (suggested) common names might be:
\begin{enumerate}
  \item \$ for the return variable.
%   \item \_\_ don't use.
  \item ¢ penny for the sent argument.
\end{enumerate}
¢ is is in place of "that" (that is a ``placeholder'' like this) or for "sent" argument.
\footnote{The term and symbol of "Penny" is chosen to convey the meaning that is something so little you do not want to worry about.
\emph{Take a penny, leave a penny!}}
Brevity for the sake of brevity is \textbf{not} a prescription in Spartan programming. 
It is worth to say that descriptiveness of variable is still a cornerstone principle. 
At the same time, Spartan Programming emphasize the need to chose short names, whenever is possible.
In other word, one word name is better than a multi-word name.

Other prescriptions are the following:
\begin{itemize}
 \item Use one letter abbreviations.
 \item Also, use ‟x”
 \item Take last word, don't use acronyms.
 \item Use plurals with 's'
 \item Do not use unpronouncable names.
 \item Do not use acronyms, unless they are well established (i.e. ASCII, LAN, etc.)
 \item Do not include type (i.e. Integer, Boolean, etc.) or the kind (i.e. public, private, etc.) into the names.
\end{itemize}

Variable name might be also a design guideline. If a developers chose a multi-word variable name for an 
entity, maybe is try to collapse on that entity multiple functionalities. 
There might be the case to re-think the design, splitting the entity according to the different functionalityies 

% Like the "it" variable in ML programming language or HyperCard programming language on macintosh. Ancient!!!


\subsection{Misc}
\begin{itemize}
\item Prefix to postifx
x.equals("a:) to "a.equals(this" (???)


\item Associative rule of ifs: 
  \begin{code}{JAVA}
if (a) if (b) x;
\end{code}
  \begin{code}{JAVA}
if (a && b) x
  \end{code}
% \item Associative rule of ifs: if (a) 
% \begin{code}{JAVA}if (b) x;\end{code}
% \begin{code}{JAVA}if (a && b) x\end{code}
\begin{code}{JAVA}
if (this == null) return super();
????\end{code}
  
  \end{itemize}

\subsection{Early returns and the such}
\begin{itemize}
\item Off load exceptional cases quickly.
\item inline single use of variables.
\item Inline multiple use of variables: if 1) result is shorter and 2) the expression has no side effects.
\item Pseudo inlining with initializers.
\item Scope:
\begin{itemize}
 \item Move into for
 \item Move into try
 \item Move next to use
\end{itemize}
\item Variability (add final when possible.)
\item Variables (inlining)
\item What’s first
\begin{itemize}
 \item Most external escape
 \item Shortest: 2*112*a*ab
 \item Least specific
\end{itemize}
\item Simplificatons
\begin{itemize}
 \item Algebraic simplifications
 \item Logical simplification
 \item Literal simplification
 \item Arithmetical simplification
\end{itemize}
\item String simplification
\begin{itemize}
 \item String transformations
 \item .toString() -> “” +
 \item x.equals(“a”) -> “a”.equals(b)
 \item Auto insert “”
\end{itemize}
\item Patterns: 2 * x + 1
\item Pushdown:
\item Logical Negation
\item Arithmetical Negation
\item Auto distributive
\item Repeated deterministic expressions
\item Auto
\begin{itemize}
 \item Auto Collect
 \item Auto insert/remove 0+ in ternarization.
 \item Auto insert/remove 1*
 \item Auto add “” +
\end{itemize}

\item Control simplification
\begin{itemize}
 \item \item Ternarize
\item Early return
\item Example: 
\begin{code}{JAVA}
@SuppressWarnings("unused") private
void earlyReturn(int a) {
  if (new Random().nextInt() <= 3) {
    earlyReturn(12);
    earlyReturn(124);
    } else {
    earlyReturn(4);
    earlyReturn(5);
    earlyReturn(44);
  }
}
\end{code}
\end{itemize}
\end{itemize}

\begin{itemize}
\item Switch Normalization
\item Renamings:
\begin{itemize}
 \item In constructors: Arguments to corresponding field names
 \item In fields: to canonical instance name or pluralized
 \item In getters: Change name to fieldName if it is a getter.
 \item In Methods
\end{itemize}

\item In local variables:
\item return variable to \$
\item Change argument to one single case letter
\item In world amalgams, use the first letter of the last word.
\item Deal with multiplicities of arrays
\item Deal with multiplicities of collections
\item Deal with nested multiplicities
\item Change return type of void to ClassName and return this
\item Single argument:
\item To \_\_ if unused
\item To \$ in case it is returned
\item To fieldName in case it is a setter
\item To ¢ in case method has no inner methods and is not a setter
\item Any number of arguments:
\item To \$ in case it is returned
\item To__(n) if it is unused and documented as such
\item Change argument to one single case letter
\item In world amalgams, use the first letter of the last word.
\item Deal with multiplicities
\end{itemize}



\end{document}

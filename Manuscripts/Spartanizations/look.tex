Spartan Programming style favour laconic expression over the verbose, leading
to a code that is (sometimes extremely) minimized. For this reason, spartanized
code might be less readable that the original. However, code readability is not
one of the main concern of Spartan Programming. All that said, to a certain
extent, spartan(ized) code might look a little bit ‟scary” to developers not
accustomed to it.

In the following, we are presenting some examples of Spartan
Programming % \urlref{https://github.com/SpartanRefactoring/spartan/blob/master/src/il/org/spartan/misc/Primes.java}.
All of them are taken from the source code of the Spartan Refactoring Plugin
for Eclipse that has been introduced in \cref{section:eclipse}. This code was
developed following the spartan programming rules since the beginning.

% \Cref{figure:first} report an application of the principles …

% \begin{figure}[h]
% \label{figure:first}
% \caption{An example of spartan code}
% \begin{Code}{JAVA}{Where was this taken from?}
% public static <T, C extends Collection<T>> accumulate<T, C>
% to(final C c) {
% return new accumulate<T, C>() {
% @Override public C elements() {
% return c;
% }
% @Override public accumulate<T, C>
% add(final @Nullable T ¢) {
% if (¢ == null)
% return this;
% c.add(¢);
% return this;
% }
% };
% }
% \end{Code}
% \end{figure}

% In the code reported in \cref{figure:shock-2} we can see ... .
%
% \begin{figure}[h]
% \label{figure:shock-2}
% \caption{An example of spartan code}
% \begin{Code}{JAVA}{Where was this taken from?}
% default accumulate<T, C>
% addAll(final Iterable<? extends T>... tss) {
% for (final Iterable<? extends T> ¢ : tss)
% addAll(¢);
% return this;
% }
% \end{Code}
%
% \end{figure}
\begin{figure}[h]
\label{figure:shock-2}
\caption{Spartan functions to test for primality†{%
      https://github.com/SpartanRefactoring/spartan/
      }
     }
\begin{Code}{JAVA}{il.org.spartan.misc.Primes#isPrime}
/** Tests for primality.
  * @param ¢ candidate to be tested
  * @return <code><b>true</b></code> <i>iff</i> the parameter is prime. */
public static boolean isPrime(final int ¢) {££
  return ¢ < -1 && isPrime(-¢) // deal with negative values
      || ¢ > 1 && isPrime¢(¢); // any integer >= 2
}
private static boolean isPrime¢(final int i) {££
  assert i >= 2;
  for (int ¢ = 2; ¢ * ¢ <= i; ++¢)
    if (i % ¢ == 0)
      return false;#
  return true;
}
\end{Code}
\end{figure}

Finally in \cref{figure:shock-3} it is possible to see that…

\begin{figure}[ht]
    \caption{%
      Spartan function to test for setting free variables in a symbolic-expression, such as those of \protect\Prolog†{%
      https://github.com/yossigil/Protolog
      }
    }
\label{figure:shock-3}
\begin{Code}{JAVA}{\scriptsize\texttt{il.org.spartan.protolog.Variable#freeVariables}}
default Set<Variable> freeVariables() {££
    final Set<Variable> £\ignore$£$ = new LinkedHashSet<>();
    if (isVariable())
     £\ignore$£$.add((Variable) this);
    else
      for (final Term ¢ : this)
       £\ignore$£$.addAll(¢.freeVariables());
    return £\ignore$£$;
  }
\end{Code}
\end{figure}

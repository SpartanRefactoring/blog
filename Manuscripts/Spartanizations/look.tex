Spartan Programming style favour laconicity over verbosity, leading to 
a code that is (sometimes extremely) minimized. For this reason, spartanized code might 
be less readable that the original. However, code readability is not one of 
the main concern of Spartan Programming. All that said, to a certain extent, 
spartan(ized) code might look a little bit ``scary'' to developers not used to it.

In the following, we are presenting some examples of Spartan Programming.
All of them are taken from the source code of the Spartan Refactoring 
Plugin for Eclipse that has been introduced in Section \ref{Introduction}. 
This code was developed following the spartan programming rules since 
the beginning.

% \Cref{figure:first} report an application of the principles … 

% \begin{figure}[h]
% \label{figure:first}
% \caption{An example of spartan code}
%   \begin{Code}{JAVA}{Where was this taken from?}
% public static <T, C extends Collection<T>> accumulate<T, C> 
%   to(final C c) {
%     return new accumulate<T, C>() {
%       @Override public C elements() {
%         return c;
%       }
%       @Override public accumulate<T, C> 
%         add(final @Nullable T ¢) {
%           if (¢ == null)
%             return this;
%           c.add(¢);
%           return this;
%         }
%     };
%   }
% \end{Code}
% \end{figure}

% In the code reported in \cref{figure:shock-2} we can see ... .
% 
% \begin{figure}[h]
% \label{figure:shock-2}
% \caption{An example of spartan code}
% \begin{Code}{JAVA}{Where was this taken from?}
% default accumulate<T, C> 
%     addAll(final Iterable<? extends T>... tss) {
%       for (final Iterable<? extends T> ¢ : tss)
%         addAll(¢);
%       return this;
%     }
% \end{Code}
% 
% \end{figure}
\begin{figure}[h]
\label{figure:shock-2}
\caption{An example of spartan code}
\begin{Code}{JAVA}{Where was this taken from?}
/** Tests for primality.
  * @param ¢ candidate to be tested
  * @return <code><b>true</b></code> <i>iff</i> the parameter is prime. */
public static boolean isPrime(final int ¢) {
  return ¢ < -1 && isPrime(-¢) // deal with negative values
      || ¢ > 1 && isPrime¢(¢); // any integer >- 2
}
private static boolean isPrime¢(final int i) {
  for (int ¢ = 2; ¢ * ¢ <= i; ++¢)
    if (i % ¢ == 0)
      return false;
  return true;
}
\end{Code}
\end{figure}

Finally in \cref{figure:shock-3} it is possible to see that ... .

\begin{figure}[ht]
\label{figure:shock-3}
\caption{An example of spartan code}
\begin{Code}{JAVA}{Where was this taken from?}
default Set<Variable> freeVariables() {
    final Set<Variable> $ = new LinkedHashSet<>();
    if (isVariable())
      $.add((Variable) this);
    else
      for (final Term ¢ : this)
        $.addAll(¢.freeVariables());
    return $;
  }
\end{Code}
\end{figure}



\matteo[Please check out]

Spartan programming is a programming style, which like the laconic speech, tries
  to minimize the elements of speech. 
In the context of code, the minimized elements of speech include 
  lines, characters, lines, arguments, nesting, use of \kk{if}s and
  \kk{while}s, etc.  
The spartan programming style is achieved by repeated application
  of code transformation techniques, or refactoring operations, drawn 
  from the spartan toolbox of \emph{wrings}. 
Each wring improves on at least one of the code size metrics, but
  not on the others.
This manuscript makes the case for the spartan style, describes in brief 
  the three main kinds of wrings: structural, nominal, and modular. 
The evaluation part of this work gives evidence that the application of
  structural wrings contributes to the naturalness of software. 



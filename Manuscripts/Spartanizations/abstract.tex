% \matteo[Please check out]

Spartan programming is a coding  style which tries to minimize the elements of
speech, like in a laconic speech.
In the context of code, the minimized elements of speech include 
lines, characters, lines, arguments, nesting, use of \kk{if}s and
\kk{while}s, etc.  

The style is achieved by the process of repeated application of code
transformation techniques, or refactoring operations, drawn from the spartan
toolbox of \emph{wrings}.  Each wring improves at least one of the code size
metrics, without degrading any of the others.  We present the unique 
look of spartan code, and the process of achieving it, including 
the three main kinds of wrings: structural, nominal, and modular. 
We do not make the case for the spartan style here, leaving readers to find
beauty or savageness in it. 

The evaluation part of this work gives evidence that the application of
structural wrings contributes to the naturalness of software. 




\subsection{Syntactic Baggage}
\begin{itemize}
  \item Redundant parenthesis and curly brackets.
  \item Remove redundant modifiers: Remove final in private methods
        Remove final from static methods
        Remove visibility of all members of in an interface
        Remove abstract before interface
        Remove abstract before methods in interfaces
        Remove final from all methods in final class
        Remove private/static from everything defined in anonymous classes
  \item Extra super() calls.
\end{itemize}

\subsection{Shorter phrases}
\begin{itemize}
  \item Distributive rule of booelan expressions
  \item Distributive rule of assignment: a = x; b = x; -> a = b =x
  \item Distributive rule of arithmetics.
  \item Common prefix of If
  \item Common Suffix of If
  \item Pull out arithmetical negation
  \item Use ternary instead of conditionals.
  \item convert prefix ++ and--into postfix when possible
  \item
\begin{code}{JAVA}
x.toString()
\end{code}
\begin{code}{JAVA}
"" + x
\end{code}
\end{itemize}

\subsection{Canonical Form of Common Expressions}
\begin{itemize}
  \item Follow patterns: Pattern: A*x + B
  \item Sorting by size, but also as per the previous rule.
  \item De morgan pushdown logical negation.
  \item Pullup arirthmetic negation.
  \item Pushdown ternarizations..
        a ? b(x) : b(y) => b(a ? x: y)
  \item Distributive rule on booleans
    \begin{code}{JAVA}
a && b || c && b => (a || c) && c
    \end{code}
      there are four variants.
  \item More specific first: How should you order X == Y.
        Rule: you start from the more specific and increase specifity,.
        I.e., S(X) > S(Y)

        \begin{enumerate}
          \item null/this/true/false
          \item 0, 1, ""
          \item 12, 13.4
          \item Classes: public, protected, package, private
          \item Fields: Static fields, Fields,
          \item parameters,
          \item local variables ordered by scope.
          \item ¢
        \end{enumerate}
\end{itemize}

\subsection{Names}
Scope plays a rule:
in short scopes use short names.
some common names:
\begin{enumerate}
  \item \$ the return variable.
  \item \_\_ don't use.
  \item ¢ penny: the sent argument,
\end{enumerate}
One letter abbreviations.
Also, use ‟x”
Take last word, don't use acronyms.
Use plurals with 's'

Like the "it" variable in ML programming language or HyperCard programming language on macintosh. Ancient!!!
¢ is short for "that" (like this) or for "sent" argument, or of "Penny" something so little you do not want to worry about.
Take a penny, leave a penny

\subsection{Misc}
Prefix to postifx
x.equals("a:) to "a.equals(this"

\begin{itemize}
\item Associative rule of ifs: 
  \begin{code}{JAVA}
  if (a) if (b) x;
\end{code}
  \begin{code}{JAVA}
  if (a && b) x
  \end{code}
\item Associative rule of ifs: if (a) 
\begin{code}{JAVA}if (b) x;\end{code}
\begin{code}{JAVA}if (a && b) x\end{code}
\begin{java}
if (this == null) return super();
\end{java}
  
  \end{itemize}

\subsection{Early returns and the such}
Off load exceptional cases quickly.
inline single use of variables.
Inline multiple use of variables: if 1) result is shorter and 2) the expression has no side effects.
Pseudo inlining with initializers.
Scope:
Move into for
Move into try
Move next to use
Variability (add final when possible.)
Variables (inlining)
What’s first
Most external escape
Shortest: 2*112*a*ab
Least specific
Algebraic simplifications
Patterns: 2 * x + 1
Pushdown:
Logical Negation
Arithmetical Negation
String transformations
.toString() -> “” +
x.equals(“a”) -> “a”.equals(b)
Auto insert “”
Auto distributive
Logical simplification
Literal simplification
Repeated deterministic expressions
Arithmetical simplification
Auto Collect
Auto insert/remove 0+ in ternarization.
Auto insert/remove 1*
Auto add “” +
Control simplification
Ternarize
Early return
@SuppressWarnings("unused") private
void earlyReturn(int a) {
  if (new Random().nextInt() <= 3) {
    earlyReturn(12);
    earlyReturn(124);
    } else {
    earlyReturn(4);
    earlyReturn(5);
    earlyReturn(44);
  }
}

Switch Normalization
Renamings:
In constructors: Arguments to corresponding field names
In fields: to canonical instance name or pluralized
In getters: Change name to fieldName if it is a getter.
In Methods
In local variables:
return variable to \$
Change argument to one single case letter
In world amalgams, use the first letter of the last word.
Deal with multiplicities of arrays
Deal with multiplicities of collections
Deal with nested multiplicities
Change return type of void to ClassName and return this
Single argument:
To \_\_ if unused
To \$ in case it is returned
To fieldName in case it is a setter
To ¢ in case method has no inner methods and is not a setter
Any number of arguments:
To \$ in case it is returned
To__(n) if it is unused and documented as such
Change argument to one single case letter
In world amalgams, use the first letter of the last word.
Deal with multiplicities

a

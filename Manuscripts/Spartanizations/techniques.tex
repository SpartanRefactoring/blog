\subsection{Syntactic Baggage}
\begin{itemize}
  \item Redundant parenthesis and curly brackets.
  \item Remove redundant modifiers: Remove final in private methods
  \item Remove final from static methods
  \item Remove visibility of all members of in an interface
  \item Remove abstract before interface
  \item Remove abstract before methods in interfaces
  \item Remove final from all methods in final class
  \item Remove private/static from everything defined in anonymous classes
  \item Extra super() calls.
\end{itemize}

\subsection{Shorter phrases}
\begin{itemize}
  \item Distributive rule of booelan expressions
  \item Distributive rule of assignment: a = x; b = x; -> a = b =x
  \item Distributive rule of arithmetics.
  \item Common prefix of If
  \item Common Suffix of If
  \item Pull out arithmetical negation
  \item Use ternary instead of conditionals.
  \item convert prefix ++ and--into postfix when possible
  \item
\begin{code}{JAVA}
x.toString()
\end{code}
\begin{code}{JAVA}
"" + x
\end{code}
\end{itemize}

\subsection{Canonical Form of Common Expressions}
\begin{itemize}
  \item Follow patterns: Pattern: A*x + B
  \item Sorting by size, but also as per the previous rule.
  \item De morgan pushdown logical negation.
  \item Pullup arirthmetic negation.
  \item Pushdown ternarizations.
  \begin{code}{java}
   a ? b(x) : b(y) => b(a ? x: y)
  \end{code}
  \item Distributive rule on booleans
    \begin{code}{JAVA}
a && b || c && b => (a || c) && c
    \end{code}
      there are four variants.
  \item More specific first: How should you order X == Y.
        Rule: you start from the more specific and increase specifity,.
        I.e., S(X) > S(Y)

        \begin{enumerate}
          \item null/this/true/false
          \item 0, 1, ""
          \item 12, 13.4
          \item Classes: public, protected, package, private
          \item Fields: Static fields, Fields,
          \item parameters,
          \item local variables ordered by scope.
          \item ¢
        \end{enumerate}
\end{itemize}

\subsection{Names}
Scope plays a rule:
in short scopes use short names.
some common names:
\begin{enumerate}
  \item \$ the return variable.
  \item \_\_ don't use.
  \item ¢ penny: the sent argument,
\end{enumerate}
One letter abbreviations.
Also, use ‟x”
Take last word, don't use acronyms.
Use plurals with 's'

Like the "it" variable in ML programming language or HyperCard programming language on macintosh. Ancient!!!
¢ is short for "that" (like this) or for "sent" argument, or of "Penny" something so little you do not want to worry about.
Take a penny, leave a penny

\subsection{Misc}
Prefix to postifx
x.equals("a:) to "a.equals(this"

\begin{itemize}
\item Associative rule of ifs: 
  \begin{code}{JAVA}
  if (a) if (b) x;
\end{code}
  \begin{code}{JAVA}
  if (a && b) x
  \end{code}
\item Associative rule of ifs: if (a) 
\begin{code}{JAVA}if (b) x;\end{code}
\begin{code}{JAVA}if (a && b) x\end{code}
\begin{java}
if (this == null) return super();
\end{java}
  
  \end{itemize}

\subsection{Early returns and the such}
\begin{itemize}
\item Off load exceptional cases quickly.
\item inline single use of variables.
\item Inline multiple use of variables: if 1) result is shorter and 2) the expression has no side effects.
\item Pseudo inlining with initializers.
\item Scope:
\begin{itemize}
 \item Move into for
 \item Move into try
 \item Move next to use
\end{itemize}
\item Variability (add final when possible.)
\item Variables (inlining)
\item What’s first
\item Most external escape
\item Shortest: 2*112*a*ab
\item Least specific
\item Algebraic simplifications
\item Patterns: 2 * x + 1
\item Pushdown:
\item Logical Negation
\item Arithmetical Negation
\item String transformations
\item .toString() -> “” +
\item x.equals(“a”) -> “a”.equals(b)
\item Auto insert “”
\item Auto distributive
\item Logical simplification
\item Literal simplification
\item Repeated deterministic expressions
\item Arithmetical simplification
\item Auto Collect
\item Auto insert/remove 0+ in ternarization.
\item Auto insert/remove 1*
\item Auto add “” +
\item Control simplification
\item Ternarize
\item Early return
\begin{java}
@SuppressWarnings("unused") private
void earlyReturn(int a) {
  if (new Random().nextInt() <= 3) {
    earlyReturn(12);
    earlyReturn(124);
    } else {
    earlyReturn(4);
    earlyReturn(5);
    earlyReturn(44);
  }
}
\end{java}
 
\end{itemize}

\begin{itemize}
\item Switch Normalization
\item Renamings:
\item In constructors: Arguments to corresponding field names
\item In fields: to canonical instance name or pluralized
\item In getters: Change name to fieldName if it is a getter.
\item In Methods
\item In local variables:
\item return variable to \$
\item Change argument to one single case letter
\item In world amalgams, use the first letter of the last word.
\item Deal with multiplicities of arrays
\item Deal with multiplicities of collections
\item Deal with nested multiplicities
\item Change return type of void to ClassName and return this
\item Single argument:
\item To \_\_ if unused
\item To \$ in case it is returned
\item To fieldName in case it is a setter
\item To ¢ in case method has no inner methods and is not a setter
\item Any number of arguments:
\item To \$ in case it is returned
\item To__(n) if it is unused and documented as such
\item Change argument to one single case letter
\item In world amalgams, use the first letter of the last word.
\item Deal with multiplicities
\end{itemize}


With the advances in network speeds a single processor cannot cope anymore with
the growing number of data streams from a single network card.  The problem is
aggravated in virtual computing environment, where each I/O operation requires
switching context to the host on which the virtual machine runs.

Previous research employed a ``side-core'' in attempt to ameliorate the problem:
all I/O operations are delegated to a dedicated core, whereby obviating the need 
for context switch. However, performance boost was demonstrated only if the number 
of virtual machines was sufficiently large.

This research proposal is to employ the idea of sidecores for I/O in a computing 
environment where there is a single virtual machine. 

We show how using multiple sidecores for the I/O operations we
can tremendously improve the performance of a single VM and reduce context
switches to the host.

We introduce an innovative approach to route i/o streams
of a single VM to several sidecores, and show how it reduces context switches to the host.




I/O operations are famous for being a bottleneck of virtualized architectures,
  in which many virtual machines are emulated run on a one more more hardware
  machines. In this research, I will explore the prospects of a new idea for
  meeting the I/O challenge: with ``side-core(s)'', a dedicated single core (or
      perhaps several cores) are taken away from the hardware pool, and are
  dedicated to conducting I/O operations for all virtual machines.



Dan suggested that he has students that work on side-core.  We allocate a dedicated` core for accelerating I/O operations of 12 virtual machines.operations of 12 virtual machines with one virtual machine.


There is something called TEE. We think of adding a feature to TEE.  Nodes his
head. Possibly to SGX Of Intel.We think of feature Gamma I/O. In hardware.
We wish to see if there is something of research type that can be done with it.

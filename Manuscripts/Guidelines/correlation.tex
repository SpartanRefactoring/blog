\label{auto-correlation}
Previous studies have shown the metrics can be highly correlated to each-other\cite{Herraiz:et:al:07, Shepperd:88}, despite having a different theoretical bases.
Since non-trivial metrics (e.g., regularity, see \ref{regularity}) can be slower to compute than trivial (e.g., lines of code, see \ref{LOC}) ones, it might be preferable to use a simpler metric.
It is therefore important to measure the degree of independence between metrics.

Although each metric gives a numerical value to a piece of code, different metrics measure code in different units (e.g., there's no sense comparing a method's line count to its cyclomatic complexity).
We will instead be comparing the \emph{induced rankings} of the metrics on $n$ code units. Assuming that after evaluating the code units we have two \emph{ranking vectors} $\overrightarrow{x}, \overrightarrow{y}$, such that~$\abs{\overrightarrow{x}}=\abs{\overrightarrow{y}}=n$, we will calculate the \emph{Kendall $\tau$ rank correlation coefficient}.
\begin{define}
	Let $\overrightarrow{x}, \overrightarrow{y}$ be two ranking vectors, and let~$(x_1,y_1),(x_2,y_2),\ldots,(x_n,y_n)$ be the zipped tuples of~$\overrightarrow{x}, \overrightarrow{y}$. A pair of tuples $(x_i,y_i)$ and $(x_j,y_j)$ are said to \emph{concordant} if both $x_i<x_j$ and $y_i<y_j$ or if both~$x_i>x_j$ and $y_i<y_j$. They are said to \emph{discordant} if $x_i>x_j$ and $y_i<y_j$ or if both~$x_i<x_j$ and $y_i<y_j$. If either~$x_i=x_j$ or $y_i=y_j$ then they are neither. \emph{The Kendall $\tau$ coefficient} is thus
	$$\tau(\overrightarrow{x}, \overrightarrow{y})\triangleq \frac{\abs{\set{\text{concordant~pairs}}} - \abs{\set{\text{discordant pairs}}}}{{n \choose 2}}.$$
\end{define}
So identical metrics would have a coefficient value of $1$, $-1$ for opposite metrics (i.e., reverse ordering), and $0$ for totally uncorrelated metrics.
In our correlation analysis, we will assume that all the metrics are equivalent\footnote{Formally, $H_0: \tau^2(\overrightarrow{x}, \overrightarrow{y}))=1$, for all $\overrightarrow{x}, \overrightarrow{y}$}.
\begin{code}{JAVA}
.Time {££
  Integer h 9; // Hours, defaults to 9
  Integer m 11; // Minutes, defaults to 11
  Integer s; // Seconds, defaults to £$⊥$£
}
\end{code}
Remove the "=" sign in initialization but not assignment.`
\begin{java}
class Time extends Object {££
  Integer h = 9; // Hours, initially, £$\cc{9}$ 9
  Integer m = 11; // Minutes, initially £$\cc{11}$ 11
  Integer s; // Seconds, initially £$\kk{null}$
}
\end{java}

\begin{code}{JAVA}
.Time.Display {££
  String ¢ = h + ":" + m + ":" + s;
}
\end{code}
reaches fields directly, just like inheritance.

\begin{code}{JAVA}
class Display extends Time {££
  String ¢() {££
     h + ":" + m + ":" + s;
  }
}
\end{code}

\begin{code}{JAVA}
class Time {££
  public static class Display extends Time {££
    String ¢() {££
       h + ":" + m + ":" + s;
    }
  }
}
\end{code}

\begin{code}{JAVA}
.Time.AMPM {££
  Boolean isAM super.h < 12;
  Boolean isPM !isAM;
  String ampm isAM ? "AM" : "PM";
  String h super.h - (isAM ? 0 : 12);
}
\end{code}

\begin{code}{JAVA}
.Time.Fraction {££
  Real ¢() -> ((60*h + m) + s)/24./60 ;
}
\end{code}
reads as
\begin{code}{JAVA}
class Fraction extends Time {££
  Real ¢ = ()-> ((60*h + m) + s)/24./60 ; 
}
\end{code}
or so
\begin{code}{JAVA}
class Fraction extends Time {££
  Real ¢() {££ return ((60*h + m) + s)/24./60 ; }
}
\end{code}

\begin{code}{JAVA}
class Time {££
  static class AMPM extends Time {££ /* £$⋯$ £*/ }
  static class Fraction extends Time {££ /* £$⋯$ £*/}
}
\end{code}

\matteo{read about mixins and traits.start with mixins}

\begin{code}{JAVA}
.Time.AMPM.Fraction;
\end{code}

\begin{code}{JAVA}
class Time {££
  static class AMPM extends Time {££
      /* £$⋯$ £*/
    static class Fraction extends Time {££ /* £$⋯$ £*/}
  }
  static class Fraction extends Time {££ /* £$⋯$ £*/}
}
\end{code}

\begin{code}{JAVA}
.Time.Fraction.AMPM;
\end{code}

\begin{code}{JAVA}
class Time {££
  static class AMPM extends Time {££
    /** £$⋯$£*/
    static class Fraction extends Time {££ /* £$⋯$ £*/}
  }
  static class Fraction extends Time {££
    /** £$⋯$£*/
    static class AMPM extends Fraction {££ /* £$⋯$ £*/}
  }
}
\end{code}

\begin{code}{JAVA}
.Time.Guess {££
  Integer s 0;
}
\end{code}

Whenever a property is undefined it is like a function that throws an exception.
Any time you use it, the called would also be throwing an exception.
\begin{code}{JAVA}
.Time.Fraction {££
  Real ¢() -> ((60*h + m) + s)/24./60 ;
}
\end{code}
To evaluate
\begin{java}
  ((60*h + m) + s)/24./60
\end{java}
replace~$h→9, m→11, s→⊥$
to obtain 
\begin{align}
  ((60·9 + 11) + ⊥)/24./60 & ⏎
  ((540 + 11) + ⊥)/24./60 & ⏎
  (551 + ⊥)/1440 & ⏎
  ⊥/1440 &⏎
  ⊥ &.
\end{align}

\begin{code}{JAVA}
.Fraction.Checked {££
 Ok ok() ->
   -1 < h && h < 24 &&
   -1 < m && m < 60 &&
   -1 < s && s < 60 ;
 Real ¢()-> ok && super.¢;
}
\end{code}

Note the use of \verb+&&+ and \verb+||+. they also deal with undefined.
Different from Java.
\begin{code}{JAVA}
.Checked.Greeting {££
  fraction     super.¢ || 0.5;
  String name  "David";
  String now() ->
      fraction < 0.50 ? "morning" :
      fraction < 0.75 ? "day" :
      "evening"
  ;
  ¢() -> "Hi " + name + ", good " + now + " to you.";
}
\end{code}

Linear 
\begin{code}{JAVA}
.Hamlet {££
  Boolean toBe() {££ 
    toBe = () -> !toBe; 
    return true; 
  }
}
\end{code}
behaves like flip flop or maybe a clock? bad metaphors


Penny pool notation:
\begin{code}{JAVA}
.Hamlet {££
  Boolean ¢() {££ ¢ = () -> !¢; return false; }
}
\end{code}

\begin{code}{JAVA}
.Ticker {££ 
  Integer start 1;
  Integer ¢() {££
      ¢ = () -> ¢ + 1; // Change gate of cell ¢
      return start;
  };
}
\end{code}

\begin{java}
   a() -> a + 1;
\end{java}
    means
\begin{java}
  ¢ = () -> ¢ + 1;
\end{java}

\begin{java}
  Integer curr() {££ prev = ¢;}
\end{java}
Is actually
\begin{java}
  Integer curr = () -> {££ prev = ¢;}
\end{java}

\begin{code}{JAVA}
.Fibonacci {££
  Integer prev 1;
  Integer curr() {££ prev = ¢;}
}
\end{code}

\begin{code}{JAVA}
.Constants {££ // Constant of Arithmetic Progression
  Integer a1;
  Integer d;
}
\end{code}

\begin{code}{JAVA}
.Constants.AP {££ // Arithmetic Progression
  Integer ¢() {¢=()->¢ + d; return a1;}
}
\end{code}

\begin{code}{JAVA}
.Constants.AP {££ // Arithmetic Progression
  Integer ¢() -> (a1,¢=()->¢ + d); 
}
\end{code}

\begin{code}{JAVA}
.Fibonacci {££ 
  Integer a1;
  Integer a2;
  Integer ¢() -> (a1, a2, ¢=()->¢ + d); 
}
\end{code}

\begin{code}{JAVA}
class Fibonacci {££ 
  Integer a1;
  Integer a2;
  Integer ¢() {££
    ¢ = () {££ ¢ = () -> ¢ + prev(¢); return a2;};
    return a1;
  }
}
\end{code}


Interesting step wise! Try to do Fibonacci? Try Fibonacci as a function which creates zillion cells?
\begin{code}{JAVA}
.Constants.AP {££ // Arithmetic Progression
  Integer ¢() -> a1 -> ¢ + d;
}
\end{code}

\begin{align}
  \label{eq:ok}
  ⊤ & = \kk{Ok} = \cc{Void}⏎
  \label{eq:oz}
  ⊥ & = \kk{Oz} = \kk{None} = \cc{@NonNull Void}
\end{align}
Type system is~$𝕋$. In addition to \kk{Ok} and \kk{Oz}, we have atomic types
\kk{Boolean}, \kk{String}, \kk{Integer}, \kk{Real}, and, \kk{Character}.
\begin{equation}
\label{eq:bounds}
∀τ∈𝕋 ∙ ⊥≤τ≤⊤.
\end{equation}

\begin{code}{JAVA}
.Fraction.Checked {££
  Ok ok() ->
    0 < h ** h < 24 **
    0 < m ** m < 60 **
    0 < s ** s < 60 ;
  Real ¢()-> ok ** super.¢;
}
\end{code}

\begin{code}{JAVA}
.Brothers {££
Integer c, h, g;
}
\end{code}

\begin{code}{JAVA}
.Fraction.Checked {££
  Ok ok() ->
    0 < h ** h < 24 **
    0 < m ** m < 60 **
    0 < s ** s < 60 ;
  Real ¢()-> ok ** super.¢;
}
\end{code}

\begin{code}{JAVA}
.Brothers {££
Integer c, h, g;
}
\end{code}

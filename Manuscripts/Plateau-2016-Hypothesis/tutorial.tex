\section{Introduction}

Much of the work of software engineers is to build the software they produce.
There is much automation to engineer in managing versions and branches,
regression testing, performance tests, dependency management, and, packaging.
Familiar tools to be managed by engineers include %\matteo{supply urlrefs}
\zz{git}\urlref{https://git-scm.com/}, \zz{maven}\urlref{https://maven.apache.org/}, 
\zz{jazz}\urlref{https://jazz.net}, \zz{junit}\urlref{http://junit.org/}. And, much in common to these is a
rule-based inference engine, such as that of \zz{make}\urlref{https://www.gnu.org/software/make/}, \zz{maven}, and
\zz{gradle}\urlref{https://gradle.org/}. Reasoning about and working with these tools requires a different
way of thinking than that of the declarative, object-oriented, functional, and
even aspect-oriented paradigms

Our interest here is in the programming paradigm behind these way of thinking.
Concretely, \Reap is a programming language under design whose main constructs
resemble underlying concepts of these building systems.

In \zz{make}, for example, these concepts include basics such as ‟target”,
‟prerequisite”, ‟recipe”, and ‟Makefile”, and more advanced such as ‟default
rule”, ‟variable” and ‟nested Makefile”. The concepts are not unique to
\zz{make}. But, these concepts occur, sometimes under different, in all
software building tools.

Development of \Reap begun started when we realized that the semi-declarative,
semi-procedural paradigm behind inference engines is applicable in the domain
of managing user interaction, specifically, the development of an Eclipse
plugin. Even though \Reap was conceived as a DSL for specific kinds of
computation, the language seems to be growing to be general purpose.
The examples used in this exposition demonstrate \Reap also by showing the
emulation of language constructs drawn from other paradigms: function, method,
instantiation, etc.

Currently, \Reap is at an early design stage, and for this reason, this
document is also a call for comments, questions, and thoughts. \Reap has no
compiler yet: To run a program in \Reap, the program must be manually
transliterated to \Java.

This chore of transliteration is ameliorated by the fact that \Reap is defined
as an extension of \Java. \Reap borrows, with a bit of simplification, \Java's
syntax, most importantly,~$λ$-expressions. Semantics is also similar, except
that the semantics of an \emph{environment}, the \Reap reification of a
Makefile, is different than that of a \Java class, even though environments
resembles classes in certain ways.

\cref{figure:date} shows side by a side a \Reap environment and its \Java
class lookalike.

\begin{figure}
  \caption{\label{figure:date}%
    A \Reap environment (a) and its \Java lookalike (b).
  }
  \begin{tabular}{c|c}
    \begin{reap}
    `'.Date {
    Integer d, m, y,
    String toString () ->
    y + "-" + m "-" + d;
    }
    \end{reap}

    \begin{java}
    class Date {
    Integer d, m, y;
    String toString() {
    return
    y + "-" + m + "-" + d;
      }
    }
    \end{java}
      \\ \multicolumn1c{(a)} & \multicolumn1c{(b)}
  \end{tabular}
\end{figure}

Avery environment defines a set of \emph{properties} and the interdependencies
between them, e.g., environment \cc{.Date} in the figure, contains four
properties:~\cc{d},~\cc{m},~\cc{y}, and, \cc{toString}. Conceptually,
properties are the equivalent of \emph{targets} of a Makefile.

Properties defined by \emph{recipes} (\cc{toString} in \cref{figure:date})
visually resemble function members; properties with no recipe
(\cc{d},~\cc{m},~\cc{y} in the figure) resemble data members.
The recipes themselves, e.g., \[
\cc{ () -> y + "-" + m "-" + d}
\] look like and are in fact \Java~$λ$-expressions taking no parameters.

An environment differs from its class lookalike mainly in the semantics of
evaluation: functions are re-evaluated each time they are invoked. Properties
(just like targets in a Makefile) are only evaluated if their current value
is outdated with respect to its prerequisites.

For this reason, the transliteration of environments to \Java demands close
attention to the semantics of~$λ$-expressions and anonymous functions---the
\Java clockwork used to realize the semantics of environments.

\section{Environments \emph{vs.} Spreadsheets}

An \emph{environment} is a directed graph of computational dependencies between
values. Nodes in the graph are called \emph{properties}, which are named
container of values.

\Cref{figure:rational} compares the \Reap code of an environment side by side
with its properties' graph . In the figure, we see seven properties
named,~\cc{n},~\cc{d},~\cc{r}, \cc{abs}, \cc{positive}, and, \cc{infinite} and
\cc{zero}.

\begin{figure}
  \caption{\label{figure:rational}%
    The \Reap code of environment \cc{.Rational} (a) and
    its dependency graph (b).
  }
  % vim:: /\/\//+,/^\s*}/-!awk -F' *// *' '{printf"\%-28s// \%s\n",$1,$2}'
  \begin{tabular}{m{0.5\columnwidth}m{0.5\columnwidth}}
    \begin{minipage}[t]{\columnwidth}
    \begin{reap}
    `'.Rational {`'
    Integer n; // Numerator
    Integer d; // Denominator
    Real r() -> n/d; // Real value
    Real abs() -> r > 0 ? r : -r;
    Boolean positive() -> n < 0 == d < 0;
    Boolean infinite() -> d == 0;
    Boolean zero() -> abs == 0;
    }
    \end{reap}
    \end{minipage}
    &
    \begin{minipage}[t]{0.5\columnwidth}
    \begin{adjustbox}{width=\columnwidth}
    \usegdlibrary{layered}

\begin{tikzpicture}[
    shorten >=3pt,
    -stealth,
    p/.style = {% Property
      label=above left:\cc{#1},
      rounded corners,
      font=\footnotesize,
      rectangle split parts=2,
    },
    f/.style = {fill=olive!30,p=#1}, % cvalue
    v/.style = {fill=Yellow,p=#1,circle,minimum height=0pt,minimum width=0pt,}, % final 
    r/.style = {fill=YellowGreen,
minimum height=2.3em,
      minimum width=3.5em,
      p=#1,
      trapezium,
      trapezium left angle=130, 
      trapezium right angle=130,
    },   % recipe
    l/.style = { % label
      ellipse,
      font=\scriptsize,
      fill=ProcessBlue,
    },
    valueIsSet/.style = {% Set 
      draw,
      very thick,
    }, 
    invalidated/.style = {% Set 
fill=ProcessBlue!20
    },
]
\def\cc#1{{\normalsize\texttt{\textbf{#1}}}}
  \graph [ 
     layered layout,
    level distance=12ex,
    sibling distance=12ex
    ] {% 
    % vim: +,/^\s}$/-!column -t
    d /{$\bot$}[v=d],
    n /{$\bot$}[v=n],
    r /{\cc{n/d}}[r=r],
    infinite /{\cc{d == 0}}[r=infinite],
    abs /{\cc{r > 0 \relax? r : -r}  }[r=abs],
    zero /{\cc{abs == 0}  }[r=zero],
    positive /{\cc{n < 0 == d < 0}}[r=positive],
    infinite <- d,
    zero<-abs<-r,
    r<-{n, d},
    positive<-{n, d},
    }; %
    \node [l,at=(r.south)] {$\bot$};
    \node [l,at=(infinite.south)] {$\bot$};
    \node [l,at=(abs.south)] {$\bot$};
    \node [l,at=(zero.south)] {$\bot$};
    \node [l,at=(positive.south)] {$\bot$};
    \path[fill=yellow!30,opacity=0.4,fill=white,] 
    (current bounding box.south west) 
    rectangle 
    (current bounding box.north east); 
\end{tikzpicture}

    \end{adjustbox}
    \end{minipage}
      \\ \multicolumn1c{(a)} & \multicolumn1c{(b)}
  \end{tabular}
\end{figure}


Comparing an environment to a spreadsheet, and properties in it to cells, we
notice the following parallels and differences:

\begin{enumerate}
  \item{Addressing:} Unlike cells, which are referenced by their Cartesian coordinates,
  properties are referenced by their \emph{name}---a plain identifier.

  \item{Type:} Properties, like cells, contain scalar values, but unlike cells,
  properties are typed, with type being one of \kk{Boolean}, \kk{Character},
  \kk{Integer}, \kk{Real}, \emph{or,} \kk{String}.

  \item{Semantics of evaluation:} Cells in spreadsheets are computed
  eagerly---whenever a cell's value changes, all dependents cells change
  dynamically.

  In contrast, properties are computed lazily---a recipe is only invoked
  when the property is read. Properties cache their value, so a second read
  of the property does not usually trigger the recipe again. 
%   If the cell's value changed. 
  When a property is read, the versions of all prerequisite
  properties are examined, and if none of these are changed, then no
  recomputation is triggered.

  \item\emph{Values and how they are computed:} Just like a cell, a property
  stores a \emph{value}, or a \emph{recipe} for computing this value from the
  values of other cells.
  Recipes are called formulas in spreadsheets. In \Reap they are realized
  by \Java~$0$-arity~$λ$-expressions of \Java.

  We overload the term ‟recipe” to refer also to a property defined by a recipe.
  In \cref{figure:rational}, properties~\cc{n} and~\cc{d} contain plain values; all others properties
  are recipes.

  \item\emph{Missing value.} The value of a property, just like the value
  of a cell, might be missing. In fact, the values of all properties in
  the figure are missing. The bottom symbol,~$⊥$, denotes a missing value.
  In the transliteration, the missing value is represented as \Java's
  \kk{null}.

  Similarly to, but not exactly the same as cells, the value of a property
  is \emph{missing} if any one of the following conditions holds.

  \begin{enumerate}
    \item \emph{Free property:} The property is not a recipe and its
          value was not explicitly set. Such properties are called \emph{free};
          properties which are not free are \cc{bound}.
          Bound properties are either recipes, or properties whose value was
          set.

          In \cref{figure:rational} only properties~\cc{n} and~\cc{d} are free.

    \item \emph{Missing prerequisite:} If the property's recipe depends on
          properties whose value is missing, as it is the case for all
          remaining properties in \cref{figure:rational}.

          Consider environment \cc{.Rational} after the assignment~\cc{d=0}.
          The assignment binds a value to property~\cc{d}; this property
          is no longer free.

          The lazy caching semantics means that an assignment to a property
          invalidates all properties that depend on it, directly, or indirectly.
          Since from each recipe in the graph there is a path that leads % grape?
          to~\cc{d}, all recipes are invalidated.

          A read of property \cc{infinite} trigger a computation of the
          recipe, which returns \kk{True}. As shown in \cref{figure:assignments}(a),
          properties \cc{d} and \cc{infinite} are non-missing (border drawn).

          \begin{figure}[H]
            \caption{\label{figure:assignments}%
              The dependency graph of \cref{figure:rational},
              after the assignment \cc{d=0} and a read of \cc{infinite} (a), 
              followed by the assignment \cc{n=1} and a read of 
              \cc{positive} (b).
            }
            \begin{tabular}{m{0.5\columnwidth}m{0.5\columnwidth}}
              \begin{minipage}[t]{0.5\columnwidth}
              \begin{adjustbox}{width=\columnwidth}
              \usegdlibrary{layered}

\begin{tikzpicture}[
    shorten >=3pt,
    -stealth,
    p/.style = {% Property
      minimum width=3.5em,
      rounded corners,
      font=\footnotesize,
      rectangle split parts=2,
    },
    f/.style = {p,fill=olive!30}, % value
    v/.style = {p,fill=green!30}, % final 
    r/.style = {p,rectangle split,fill=red!30},   % recipe
]
\def\cc#1{{\scriptsize{#1}}}
  \graph [ 
    branch up,
    layered layout,
    grow'=up,
    level distance=11ex,
    sibling distance=12ex
    ] {% 
    % vim: +,/^\s}$/-!column -t
    d /{\cc{d}  $(\cc{0})$}[v,draw,thick],
    n /{\cc{n}  $(\bot)$}[v],
    r /{\cc{n/d} \nodepart{two} \cc{r} $(\bot)$}[r],
      infinite /{\cc{d == 0} \nodepart{two} \cc{infinite} $(\kk{True})$}[r,draw,thick],
    abs /{\cc{r > 0 \relax? r : -r} \nodepart{two} \cc{abs} $(\bot)$ }[r],
    zero /{\cc{abs == 0} \nodepart{two} \cc{zero} $(\bot)$ }[r],
    positive /{\cc{n < 0 == d < 0} \nodepart{two} \cc{positive} $(\bot)$}[r],
    infinite -> d,
    zero->abs->r,
    r->{n, d},
    positive->{n, d},
    }; %
\path[fill=yellow!30,opacity=0.4,fill=white,] 
    (current bounding box.south west) 
    rectangle 
    (current bounding box.north east); 
\end{tikzpicture}

              \end{adjustbox}
              \end{minipage}
              & 
              \begin{minipage}[t]{0.5\columnwidth}
              \begin{adjustbox}{width=\columnwidth}
              \usegdlibrary{layered}

\begin{tikzpicture}[
thick,
    shorten >=3pt,
    -stealth,
    p/.style = {% Property
      label=above left:\cc{#1},
      rounded corners,
      font=\footnotesize,
      rectangle split parts=2,
	minimum height=0pt,
	minimum width=0pt,
    },
    f/.style = {fill=olive!30,p=#1}, % cvalue
    v/.style = {fill=Yellow,p=#1,circle,inner sep=2pt,}, % final 
    r/.style = {fill=YellowGreen,
      p=#1,
minimum height=2em,
      trapezium,
      trapezium left angle=130, 
      trapezium right angle=130,
	inner sep=8pt,
% inner sep=0pt,
%  outer sep=0pt,
    },   % recipe
    l/.style = { % label
      ellipse,
      font=\scriptsize,
      fill=ProcessBlue!90,
inner sep=2pt,
    },
    valueIsSet/.style = {% Set 
      draw,
      very thick,
    }, 
    invalidated/.style = {% Set 
fill=ProcessBlue!20
    },
]
\def\cc#1{{\normalsize\texttt{\textbf{#1}}}}
  \graph [ 
     layered layout,
edges=rounded corners,
    level distance=12ex,
    sibling distance=12ex
    ] {% 
    % vim: +,/^\s}$/-!column -t
    d /{\cc{1}}[v=d,valueIsSet],
    n /{$\bot$}[v=n,valueIsSet],
    r /{\cc{n/d}}[r=r],
    infinite /{\cc{d == 0}}[r=infinite],
    abs /{\cc{r > 0 \relax? r : -r}  }[r=abs],
    zero /{\cc{abs == 0}  }[r=zero],
    positive /{\cc{n < 0 == d < 0}}[r=positive],
    infinite <- d,
    zero<-abs<-r,
    r<-{n, d},
    positive<-{n, d},
    }; %
    \node [l,at=(r.south)] {$\bot$};
    \node [l,valueIsSet,at=(infinite.south)] {\scriptsize\texttt{True}};
    \node [l,at=(abs.south),invalidated] {$\bot$};
    \node [l,at=(zero.south),invalidated] {$\bot$};
    \node [l,valueIsSet,at=(positive.south)] {\scriptsize\texttt{True}};
    \path[fill=yellow!30,opacity=0.4,fill=white,] 
    (current bounding box.south west) 
    rectangle 
    (current bounding box.north east); 
\end{tikzpicture}

              \end{adjustbox}
              \end{minipage}
                \\ \multicolumn1c{(a)} & \multicolumn1c{(b)}
            \end{tabular}
          \end{figure}

          However a read of any other
          recipe yields~$⊥$ \emph{without} recomputation: For example, reading
          \cc{positive} immediately returns~$⊥$: Even though~\cc{d} is defined,
          computation of the recipe \cc{n < 0 == d < 0} cannot commence
          while~\cc{n=$⊥$}. By the same consideration, property~\cc{r} remains
          in the missing prerequisite state. Since~\cc{r} does not change,
          properties \cc{abs} and \cc{zero} retain their current~$⊥$ value

          \item\emph{Errors:} If the recipe itself fails, i.e., the \Java
          evaluation of the~$λ$-expression throws an exception, then the
          recipe's value becomes~$⊥$.

          Let us add now the assignment~\cc{n = 1} to the previous \cc{d = 0}.
          With this assignment, no more free variables remain in the
          environment.

          The new assignment invalidates \cc{positive}. Examining this property
          will now trigger recomputation yielding \kk{True}. Indeed, in the
          \cc{.Rational} environment the undefined fraction~$1/0$ is considered
          both positive and infinite, and the bordered nodes in
          \cref{figure:assignments}(b) are properties \cc{d}, \cc{n}, 
          \cc{infinite} and \cc{positive}. 

          Property~\cc{r} is likewise invalidated by the assignment~\cc{n=1}.
          Since both~\cc{n} and~\cc{d}, the two prerequisites of
          recipe~\cc{n/d}, are defined, this recipe is triggered to compute a
          new value when~\cc{r} is examined. This computation however aborts
          due to division by zero. Property~\cc{r} then retains its missing
          value. No recomputation will be attempted before one of~\cc{n}
          or~\cc{d} are updated.  Note that recipe properties \cc{abs} and
          \cc{zero} retain their cached missing value.
  \end{enumerate}
\end{enumerate}

\section{Edges and Types}

Edges in the graph are defined by the recipes present on the properties. There
is a dependency edge leading from property~$u$ to property~$v$, if~$u$ is an
implicit prerequisite of~$v$. Specifically, if the recipe of~$v$ involves~$u$.

Prerequisites are computed by (manually) examining the \Java code that makes
the recipe of~$u$. All properties defined in the environment are accessible to
this \Java code. If a property~$v$ is read in this code, then~$u$ is a
prerequisite of~$v$.
For example, the recipe \[
\cc{n < 0 == d < 0}
\] of property \cc{positive},  implicitly defines that \cc{n} and \cc{d} are
prerequisites of \c{positive}.

In certain occasions, it is useful to define cells of the top type, \cc{Ok},
\kk{Character}, \kk{Integer}, \kk{Real}, \text

\begin{align}
  \label{eq:ok}
  ⊤ & = \kk{Ok} = \cc{Void}⏎
  \label{eq:oz}
  ⊥ & = \kk{Oz} = \kk{None} = \cc{@NonNull Void}
\end{align}

Let the set of all types of \Reap be denoted by $\mathbb T$, then
we have
\begin{equation}
  \label{eq:bounds}
  ∀τ∈𝕋 ∙ ⊥≤τ≤⊤.
\end{equation}

Any time you use it, the called would also be throwing an exception.

In i.e., the expression/function
making the
recipe of~$u$, uses~$v$. This uses relationship is implicit: If the
expression/function
functions,
there is a
Type system is~$𝕋$.
and edges are defined by recipes. Names are plain
identifierso called,
properties, carry ordinary identifiers as names.
identifiers.
A \emph{recipe}
is a function attached to a property, defining how the property's value
is to be computed from those of other properties. If the recipe
interdependencies between values
of the properties. The graph is not necessarily connected, and although it may
contain cycles,
Each node in the graph is a typed
\emph{property}. whose nodes are properties.
In this definition of environment \cc{.pred}
% vim:: /\/\//,/^\s*}/- !awk -F // '{printf "%-20s //%s\n",~$1,~$2}'
\begin{reap}
  `'.Temperature {`'
    Real f0 0;
    Real c()-> 5*(f0-32);
    Real f0 0;
  }
\end{reap}
in the case that no value was assigned to a property
\begin{reap} \end{reap}
The edges in the graph
A \emph{property}
contains a value, which might be computed
is replaced by that of an \emph{environment} defining \emph{properties}, some
of which are adorned by \emph{recipes}. Properties are similar to variables,
except that a property whose value was not explicitly set, may still yield a
value: \Reap's underlying inference engine applies any required recipes to
compute this value from values of other properties.
properties scalar
and,
\kk{String}.
Whenever a property is undefined it is like a function that throws an exception.


\begin{reap}
  `'.square {`'
    Real x;
    Real x2() -> x * x;
  }
\end{reap}
\begin{reap}
  `'.square {`'
    Real x;
    Real ¢() -> x * x;
  }
\end{reap}
\begin{reap}
  `'.square {`'
    Real `\_';
    Real ¢() -> x * x;
  }
\end{reap}
\begin{reap}
  `'.power {`'
    Real x;
    Real abs() -> abs(x);
    Boolean positive() -> x > 0;
    Integer n;
    Real ¢() -> exp(n * log(abs)) * (positive ? 1 : -1) || 0;
  }
\end{reap}
\
\begin{reap}
  `'.power {`' // Global math constants
    Real x;
    Integer n;
    Real ¢() ->
  }
\end{reap}
\begin{reap}
  `'.Math {`' // Global math constants
    Real E 2.7182818284590452354;
    Real PI 3.14159265358979323846;
    Real PI2() -> 2 * PI;
  }
\end{reap}
\begin{reap}
  `'.Math.Sin0 {`' // A function to compute`~$\sin(x)$' //
    Real `\_';
    Real positive() -> `\_' > 0 `'? `\_' : -`\_';
    Real cycles() -> (int)(positive() / PI2);
    Real normalized() -> positive - PI2 * cycles;
    Real ¢() -> normalized;
  }
\end{reap}
\begin{reap}
  `'.Sin0 + .Sin1 {`' // A function to compute`~$\sin(x)$' //
    Real `\_' '\^'.¢;
    Real ¢() -> '\_- \_*\_ * \_/6';
  }
\end{reap}
\begin{reap}
  `'.Math.Sin2 {`' // A function to compute'~$\sin(x)$' //
    Real ¢() -> `\^.¢ - power.n=5,x='\_`.¢ /120';
  }
\end{reap}
% vim:: /\/\//,/^\s*}/- !awk -F // '{printf "%-20s //%s\n",~$1,~$2}'
\begin{reap}
  `'.Time {`' // Time environment
    Integer h 9; // Hours, defaults to 9
    Integer m 11; // Minutes, defaults to 11
    Integer s; // Seconds, defaults to `$⊥$'
  }
\end{reap}
Remove the "=" sign in initialization but not assignment.
% vim:: /\/\//,/^\s*}/- !awk -F // '{printf "%-20s //%s\n",~$1,~$2}'
\begin{java}
  class Time extends Object {`'// Time environment transliterated to \Java
    Integer h = 9; // Hours, initially set to `$\cc{9}$'
    Integer m = 11; // Minutes, initially set to `$\cc{11}$'
    Integer s; // Seconds, initially set to `$\kk{null}$'
  }
\end{java}

\section{Methods}

\begin{reap}
  `'.Time.toString {`'
    String ¢ = h + ":" + m + ":" + s;
  }
\end{reap}
reaches fields directly, just like inheritance.
\begin{java}
  class toString extends Time {`' // `\Java' transliteration of~$\cc{Time.toString}$
    String ¢() -> h + ":" + m + ":" + s;
  }
\end{java}
\begin{java}
  class Time {`' // More accurate `\Java' transliteration of~$\cc{Time.toString}$
    public static class toString extends Time {`'
      String ¢() {`'
        h + ":" + m + ":" + s;
      }
    }
  }
\end{java}
\begin{reap}
  `'.Time.AMPM {`'
    Boolean isAM `\^'.h < 12;
    Boolean isPM `'!isAM;
    String ampm isAM `'? "AM" : "PM";
    String h `\^'.h - (isAM `'? 0 : 12);
  }
\end{reap}
\begin{reap}
  `'.Time.fraction {`'
    Real hour 24 * 60;
    Real ¢() -> ((60*h + m) + s)/24./60 ;
  }
\end{reap}
reads as
\begin{java}
  /** `\Java' transliteration of `\cc{.Time.fraction}'
  */
  class fraction extends Time {`'
    double hour() {`' return 24 * 60; }
    double ¢() {`' return ((60*h + m) + s)/hour; }
  }
\end{java}
or so
\begin{java}
  /* more efficient `\Java' transliteration of `\cc{.Time.fraction}'
  **/
  class fraction extends Time {`'
    double ¢() {`' return ((60*h + m) + s)/(double) (24*60) ; }
  }
\end{java}
\begin{java}
  class Time {`'
    static class AMPM extends Time {`' /* `$⋯$ '*/ }
    static class fraction extends Time {`' /* `$⋯$ '*/}
  }
\end{java}
\matteo{read about mixins and traits.start with mixins}
\begin{reap}
  `'.Time.AMPM.fraction;
\end{reap}
\begin{java}
  class Time {`'
    static class AMPM extends Time {`'
      /* `$⋯$ '*/
      static class fraction extends Time {`' /* `$⋯$ '*/}
    }
    static class fraction extends Time {`' /* `$⋯$ '*/}
  }
\end{java}
\begin{reap}
  `'.Time.fraction.AMPM;
\end{reap}
\begin{java}
  class Time {`'
    static class AMPM extends Time {`'
      /** `$⋯$'*/
      static class fraction extends Time {`' /* `$⋯$ '*/}
    }
    static class fraction extends Time {`'
      /** `$⋯$'*/
      static class AMPM extends fraction {`' /* `$⋯$ '*/}
    }
  }
\end{java}
\begin{reap}
  `'.Time.GuessSeconds {`' // A function to set the number of seconds to 0
    Integer s 0;
  }
\end{reap}
\begin{reap}
  `'.Time.fraction {`'
    Real ¢() -> ((60*h + m) + s)/24./60 ;
  }
\end{reap}
To evaluate
\begin{java}
  ((60*h + m) + s)/24./60
\end{java}
replace~$h→9, m→11, s→⊥$
to obtain
{\scriptsize
  \begin{equation}
    \begin{split}
      ¢ & = ((60·9 + 11) + ⊥)/24./60⏎
      & = ((540 + 11) + ⊥)/24./60⏎
      & = (551 + ⊥)/1440. &⏎
      & = ⊥/1440.⏎
      & = ⊥
    \end{split}
    `'.
  \end{equation}
}
\begin{reap}
  `'.fraction.Checked {`'
    Ok ok() ->
    -1 < h && h < 24 &&
    -1 < m && m < 60 &&
    -1 < s && s < 60 ;
    Real ¢()-> ok && `\^'.¢;
  }
\end{reap}
Note the use of \verb+&&+ and \verb+||+. they also deal with undefined.
Different from Java.
\begin{reap}
  `'.Checked.Greeting {`'
    fraction `\^'.¢ || 0.5;
    String name "David";
    String now() ->
    fraction < 0.50 `'? "morning" :
    fraction < 0.75 `'? "day" :
    "evening"
    ;
    ¢() -> "Hi " + name + ", good " + now + " to you.";
  }
\end{reap}
Linear
\begin{reap}
  `'.Hamlet {`'
    Boolean toBe() {`'
      toBe = () -> `'!toBe;
      return true;
    }
  }
\end{reap}
behaves like flip flop or maybe a clock? bad metaphors
Penny pool notation:
\begin{reap}
  `'.Hamlet {`'
    Boolean ¢() {`' ¢ = () -> `'!¢; return false; }
  }
\end{reap}
\begin{reap}
  `'.Ticker {`'
    Integer start 1;
    Integer ¢() {`'
      ¢ = () -> ¢ + 1; // Change gate of cell ¢
      return start;
    };
  }
\end{reap}
\begin{java}
  a() -> a + 1;
\end{java}
means
\begin{java}
  ¢ = () -> ¢ + 1;
\end{java}
\begin{java}
  Integer curr() {`' prev = ¢;}
\end{java}
Is actually
\begin{java}
  Integer curr = () -> {`' prev = ¢;}
\end{java}
\begin{reap}
  `'.Fibonacci {`'
    Integer prev 1;
    Integer curr() {`' prev = ¢;}
  }
\end{reap}
\begin{reap}
  `'.AP.Constants {`' // Constant of Arithmetic Progression
    Real a1;
    Real d;
  }
\end{reap}
\begin{reap}
  `'.AP.nTh {`' // Arithmetic Progression
    Integer ¢() -> (a1,¢=()->¢ + d);
  }
\end{reap}
\begin{reap}
  `'.Fibonacci {`' // Compute the Fibonacci sequence
    Integer a1;
    Integer a2;
    Integer ¢ = {`' a1, a2, ¢ = () -> ¢ + ¢[1] };
  }
\end{reap}
Interesting step wise! Try to do Fibonacci? Try Fibonacci as a function which creates zillion cells?
\begin{reap}
  `'.Constants.AP {`' // Arithmetic Progression
    Integer ¢() -> a1 -> ¢ + d;
  }
\end{reap}
\matteo{read about type Unity which is also OK which is also top in our case}
\begin{reap}
  `'.fraction.Checked {`'
    Ok ok() ->
    // Up casting into type `$⊤ = \kk{Ok}$', of `…'
    // `…' an expression of type `\kk{Boolean}'
    0 < h && h < 24 &&
    0 < m && m < 60 &&
    0 < s && s < 60
    ;
    Real ¢()-> ok && `\^'.¢;
  }
\end{reap}
\begin{reap}
  `'.Brothers {`'
    Integer c, h, g;
  }
\end{reap}
\begin{reap}
  `'.fraction.Checked {`'
    Ok ok() ->
    0 < h && h < 24 &&
    0 < m && m < 60 &&
    0 < s && s < 60 ;
    Real ¢()-> ok && `\^'.¢;
  }
\end{reap}
\begin{reap}
  `'.Brothers {`'
    Integer c, h, g;
  }
\end{reap}
Arrays may be represented as previous versions of a cellimport java.util.function.Function;
\begin{reap}
  class Test {
    public static void main(final String[] args) {
      final int Z = 0;
      final Function<Integer, Function<Integer, Integer>> f = X -> Y -> Z;
      System.out.println(f.apply(0).apply(0));
    }
  }
\end{reap}

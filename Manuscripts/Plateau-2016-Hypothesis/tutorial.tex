\begin{code}{JAVA}
.Fraction.Checked {££
  Ok ok() -> 
    0 < h ** h < 24 ** 
    0 < m ** m < 60 ** 
    0 < s ** s < 60 ;
  Real ¢()->  ok ** super.¢; 
}
\end{code}


\begin{code}{JAVA}
.Brothers {££
Integer c, h, g;
}
\end{code}

\begin{code}{JAVA}
.Time {££
  Integer h 10; // Hours, defaults to 10 
  Integer m 20; // Minutes, defaults to 20 
  Integer s;    // Seconds, defaults to £$⊥$£  
}
\end{code}
Remove the "=" sign in initialization but not assignment 
\begin{java}
class Time extends Object {££
  Integer h = 10; // Hours, initially, £$\cc{10}$ 10 
  Integer m = 20; // Minutes, initially £$\cc{20}$ 20 
  Integer s;      // Seconds, initially £$\kk{null}$ 
}
\end{java}

\begin{code}{JAVA}
.Time.Display {££
  String ¢ = h + ":" + m + ":" + s;    
}
\end{code}
reaches fields directly, just like inheritance.

\begin{code}{JAVA}
.Time.AMPM {££
  Boolean isAM  super.h < 12;
  Boolean isPM  !isAM;
  String  ampm  isAM ? "AM" : "PM"; 
  String  h     super.h - (isAM ? 0 : 12);    
}
\end{code}

\begin{code}{JAVA}
.Time.Fraction {££
  Real ¢() -> ((60*h + m) + s)/24./60 ;
}
\end{code}


\begin{code}{JAVA}
.Time.Guess {££
  Integer s = 0; 
}
\end{code}

\begin{code}{JAVA}
.Time.Fraction {££
  Real ¢() -> ((60*h + m) + s)/24./60 ;
}
\end{code}

%\begin{code}{JAVA}
%.Fraction.Checked {££
%  Ok ok() -> 
%    0 < h && h < 24 && 
%    0 < m && m < 60 && 
%    0 < s && s < 60 ;
%  Real ¢()->  ok && super.¢; 
%}
%\end{code}

\def\ignore#1{}

Note the use of \verb_&&_ and \verb_||_.
\begin{code}{JAVA}{
.Checked.Greeting {££
  fraction = super.¢ || 0.5;
  String name "David";
  String now ->
      fraction < 0.50 ? "morning" :
      fraction < 0.75 ? "day" :
      "evening"
  ;
  £\ignore$£$ -> "Hi " + name + ", good " + now + " to you.";
}
\end{code}

\begin{code}{JAVA}
.Hamlet {££
  Boolean toBe() {££ toBe = () -> !toBe; return false; } 
}
\end{code}
behaves like flip flop or maybe a clock? bad metaphors

\begin{code}{JAVA}
.Hamlet {££
  Boolean ¢() {££ ¢ = () -> !¢; return false; } 
}
\end{code}

\begin{code}{JAVA}
.Ticker [
  Integer start 1;
  Integer ¢ {££
      ¢()  -> ¢ + 1;
      return start;
  };
\end{code}

Remove the "=" sign in initialization but not assignment 
\begin{java}
   a()  -> a + 1;
\end{java}
    means
\begin{java}
  ¢ = () -> ¢ + 1;
\end{java}


\begin{java}
  Integer curr() {££ prev = ¢;}
\end{java}
Is actually 
\begin{java}
  Integer curr = () -> {££ prev = ¢;}
\end{java}

\begin{code}{JAVA}
.Fibonacci {££
  Integer prev 1;
  Integer curr() {££ prev = ¢;}
}
\end{code}

\begin{code}{JAVA}
.Constants {££ // Of Arithmetic Progression
  Integer a1;
  Integer d;
  Integer curr() {¢=()->¢ + d; prev = ¢;}
}
\end{code}

\begin{code}{JAVA}
.Constants.AP {££ // Arithmetic Progression
  Integer curr() {¢=()->¢ + d; return a1;} 
}
\end{code}

Interesting step wise! Try to do Fibonacci? Try Fibonacci as a function which creates zillion cells?
\begin{code}{JAVA}
.Constants.AP {££ // Arithmetic Progression
  Integer ¢() -> a1 -> ¢ + d; 
}
\end{code}


\begin{align}
  \label{eq:ok}
  ⊤ & = \kk{Ok} = \cc{Void}⏎
  \label{eq:oz}
  ⊥ & = \kk{Oz} = \kk{None} = \cc{@NonNull Void}
\end{align}
Type system is~$𝕋$.  In addition to \kk{Ok} and \kk{Oz}, we have atomic types
\kk{Boolean}, \kk{String}, \kk{Integer}, \kk{Real}, and, \kk{Character}.
\begin{equation}
\label{eq:bounds}
∀τ∈𝕋 ∙ ⊥≤τ≤⊤.  
\end{equation}



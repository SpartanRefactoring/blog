\section{Introduction}

Much of the work of software engineers is to build the software they produce.
There is much automation to engineer in managing versions and branches,
regression testing, performance tests, dependency management and packaging.
Familiar tools to be managed by engineers include
\zz{git}\urlref{https://git-scm.com/},
\zz{maven}\urlref{https://maven.apache.org/},
\zz{jazz}\urlref{https://jazz.net}, \zz{junit}\urlref{http://junit.org/}. And,
much in common to these is a rule-based inference engine, such as that of
\zz{make}\urlref{https://www.gnu.org/software/make/}, \zz{maven}, and
\zz{gradle}\urlref{https://gradle.org/}. Reasoning about and working with these
tools requires a different way of thinking than that of the declarative,
object-oriented, functional, and even aspect-oriented paradigms.

Our interest here is in the \emph{programming paradigm} behind these way of
thinking. Concretely, \Reap is a programming language under design whose main
constructs resemble underlying concepts of these building systems.

In \zz{make}, for example, these concepts include basics such as ‟target”,
‟prerequisite”, ‟recipe”, and ‟Makefile”, and more advanced such as ‟default
rule”, ‟variable” and ‟nested Makefile”. The concepts are not unique to
\zz{make}. But, these concepts occur, sometimes with differences, in all
software building tools.

Development of \Reap begun when we realized that the semi-declarative,
semi-procedural paradigm behind inference engines is applicable in the domain
of managing user interaction, specifically, the development of an Eclipse
plugin. Even though \Reap was conceived as a DSL for specific kinds of
computation, the language seems to be growing to be general purpose. 

%The
%examples used in this exposition demonstrate \Reap also by showing the
%emulation of language constructs drawn from other paradigms: function, method,
%instantiation, etc.

Currently, \Reap is at an early design stage, and for this reason, this
document (not an article) is also a call for comments, questions, and thoughts.
\Reap has no compiler yet: To run a program in \Reap, the program must be
manually \emph{transliterated} to \Java.

This chore of transliteration is ameliorated by the fact that \Reap is defined
as an extension of \Java. \Reap borrows, with a bit of simplification, \Java's
syntax, most importantly,~$λ$-expressions. Semantics is also similar, except
in the case of an \emph{environment}, the \Reap reification of a
Makefile, for which the semantics is different than that of a \Java class, 
even though environments resembles classes in certain ways.

To environments with classes, consider \cref{figure:date:reap} showing a \Reap
environment for the representation of dates.

\begin{figure}
  \caption{\label{figure:date:reap}%
    A \Reap environment to represent a date.
  }
\begin{reap}
`'.Date {
  Integer d, m, y,
  String toString() -> y + "-" + m "-" + d;
}
\end{reap}
\end{figure}

\Cref{figure:date:java} depicts the \Java \kk{class} lookalike of environment
\cc{.Date} defined from \cref{figure:date:reap}.

\begin{figure}
  \caption{\label{figure:date:java}%
The \Java \kk{class} lookalike of environment~\cc{.Date} of
\cref{figure:date:reap}.
  }
\begin{java}
class Date {
  Integer d, m, y;
  String toString() {
    return y + "-" + m + "-" + d;
  }
}
\end{java}
\end{figure}

Every environment defines a set of \emph{properties} and the interdependencies
between them, e.g., environment \cc{.Date} in the figure contains four
properties: properties~\cc{d},~\cc{m},~\cc{y} are free, while property
\cc{toString} depends on the three. Conceptually, properties can be thought of
as {targets} of a Makefile.

Properties defined by \emph{recipes} (\cc{toString} in \cref{figure:date:reap})
visually resemble function members; properties with no recipe
(\cc{d},~\cc{m},~\cc{y} in the figure) resemble data members.
The recipes themselves, e.g.,
\[
  \cc{() -> y + "-" + m "-" + d}
\]
look like, and are in fact,~$0$-arity \Java~$λ$-expressions.

An environment differs from its class lookalike mainly in the semantics of
evaluation: functions are re-evaluated each time they are invoked. Properties
(just like targets in a Makefile) are only evaluated if their current value is
outdated with respect to its prerequisites: Property \cc{toString}
(\cref{figure:date:reap}) caches the last value it returns, invalidating the
cache only if~\cc{d},~\cc{m}, and~\cc{y} changes, whereas no-arguments function
\cc{toString()} (\cref{figure:date:java}) recomputes whenever it is invoked.

For this reason, the transliteration of environments to \Java demands close
attention to the semantics of~$λ$-expressions and anonymous functions---the
\Java clockwork used to realize the semantics of environments.

\section{Environments and the Prerequisites' Graph}

Every \emph{environment} is essentially a textual representation of a
\emph{prerequisites graph}, which is a directed graph of computational
dependencies between values. Nodes in the graph are called \emph{properties},
which are named containers of values. 

\Cref{figure:rational} shows the \Reap code of environment \cc{.Rational},
comprising seven properties:~\cc{n},~\cc{d} for which there is no recipe, and
five properties~\cc{r}, \cc{abs}, \cc{positive}, \cc{infinite}, and, \cc{zero},
which are defined by recipes.

\begin{figure}\caption{\label{figure:rational}%
      A \Reap environment to represent a rational number.
    }
%Vim:/\/\//,/^\s*}/-!awk -F' *// *' '{printf"\%-34s// \%s\n",$1,$2}'
  \begin{adjustbox}{max width=\columnwidth}
\begin{reap}
`'.Rational {`'
  Integer n;                      // Numerator
  Integer d;                      // Denominator
  Real    R() -> n/d;             // Real value
  Real    A() -> r > -1 ? r : -r; // Absolute value
  Boolean P() -> n < 0 == d < 0;  // Positive?
  Boolean I() -> d == 0;          // Infinite?
  Boolean Z() -> abs == 0;        // Zero?
}
\end{reap}
\end{adjustbox}
\end{figure}

Let us overload the term ‟\emph{recipe}” to refer also to a property defined by a
recipe; no confusion should arise. Furthering the gastronomical parable, the
term ‟\emph{ingredient}” shall refer to a property without a recipe.

Edges in the prerequisites' represent prerequisite relationships and are
defined by the recipes present on the properties. There is a \emph{prerequisite
edge} leading from property~$v$ to property~$u$, if~$u$ is an implicit
prerequisite of~$v$. Specifically, if the recipe of~$v$ involves~$u$.

Prerequisites are computed by (manually) examining the \Java code that makes
the recipe of~$u$. All properties defined in the environment are accessible to
this \Java code. If a property~$v$ is read in this code, then~$u$ is a
prerequisite of~$v$.  For example, the recipe \[
  \cc{n < 0 == d < 0}
\] of property \cc{positive}, \emph{implicitly} defines that~\cc{n} and~\cc{d}
are prerequisites of \c{positive}.

\Cref{figure:rational:prerequisites} depicts the prerequisites' graph
of~\cc{.Rational}. Evidently, ingredients~\cc{n} and~\cc{d} are sources of the
graph. The remaining five are recipes.

\begin{figure}[!hb]
  \centering
  \caption{\label{figure:rational:prerequisites}%
    The prerequisites graph of \Reap environment \cc{.Rational} of
    \cref{figure:rational}.
  }%
  \begin{adjustbox}{width=4in}
      \usegdlibrary{layered}

\begin{tikzpicture}[
    shorten >=3pt,
    -stealth,
    p/.style = {% Property
      label=above left:\cc{#1},
      rounded corners,
      font=\footnotesize,
      rectangle split parts=2,
    },
    f/.style = {fill=olive!30,p=#1}, % cvalue
    v/.style = {fill=Yellow,p=#1,circle,minimum height=0pt,minimum width=0pt,}, % final 
    r/.style = {fill=YellowGreen,
minimum height=2.3em,
      minimum width=3.5em,
      p=#1,
      trapezium,
      trapezium left angle=130, 
      trapezium right angle=130,
    },   % recipe
    l/.style = { % label
      ellipse,
      font=\scriptsize,
      fill=ProcessBlue,
    },
    valueIsSet/.style = {% Set 
      draw,
      very thick,
    }, 
    invalidated/.style = {% Set 
fill=ProcessBlue!20
    },
]
\def\cc#1{{\normalsize\texttt{\textbf{#1}}}}
  \graph [ 
     layered layout,
    level distance=12ex,
    sibling distance=12ex
    ] {% 
    % vim: +,/^\s}$/-!column -t
    d /{$\bot$}[v=d],
    n /{$\bot$}[v=n],
    r /{\cc{n/d}}[r=r],
    infinite /{\cc{d == 0}}[r=infinite],
    abs /{\cc{r > 0 \relax? r : -r}  }[r=abs],
    zero /{\cc{abs == 0}  }[r=zero],
    positive /{\cc{n < 0 == d < 0}}[r=positive],
    infinite <- d,
    zero<-abs<-r,
    r<-{n, d},
    positive<-{n, d},
    }; %
    \node [l,at=(r.south)] {$\bot$};
    \node [l,at=(infinite.south)] {$\bot$};
    \node [l,at=(abs.south)] {$\bot$};
    \node [l,at=(zero.south)] {$\bot$};
    \node [l,at=(positive.south)] {$\bot$};
    \path[fill=yellow!30,opacity=0.4,fill=white,] 
    (current bounding box.south west) 
    rectangle 
    (current bounding box.north east); 
\end{tikzpicture}

  \end{adjustbox}
\end{figure}

We also see in the figure that recipes~\cc{r} and \cc{positive} depend directly
on ingredient~\cc{n}, and that recipe \cc{zero} depends on recipe \cc{abs},
which in turn, depends on~\cc{r}. Therefore, \cc{abs} and \cc{zero}
\emph{indirectly} depend on~\cc{n} and~\cc{r}.

\section{Environments \emph{vs.} Spreadsheets}
Comparing an environment with its properties to a spreadsheet with its cells,
we notice these parallels and differences:

\begin{enumerate}
  \item{Addressing:} Unlike cells, which are referenced by their Cartesian coordinates,
  properties are referenced by their \emph{name}---a plain identifier.

  \item{Type:} Properties, like cells, contain scalar values, but unlike cells,
  properties are typed, with type being one of \kk{Boolean}, \kk{Character},
  \kk{Integer}, \kk{Real}, \emph{or,} \kk{String}.

  \item{Semantics of evaluation:} Cells in spreadsheets are computed
  eagerly---whenever a cell's value changes, all dependents cells change
  dynamically.

  In contrast, properties are computed lazily---a recipe is only invoked
  when the property is read. Properties cache their value, so a second read
  of the property does not usually trigger the recipe again.

  When a property is read, the versions of all prerequisite properties are
  examined, and if none of these changed, no recomputation is
  triggered.

  \item\emph{Values and how they are computed:} Just like a cell, a property
    stores a \emph{value}, or a \emph{recipe} for computing this value from the
    values of other cells.

  Recipes are called formulas in spreadsheets. In \Reap, they are realized
  by \Java~$0$-arity~$λ$-expressions.

  \item\emph{Missing values.} The value of a property, just like the value
  of a cell, might be missing. In fact, the values of all properties in the
  figure are missing. The bottom symbol,~$⊥$, denotes a missing value. In
  transliteration, the missing value is converted to literal \kk{null} which
  belongs to all types.

\end{enumerate}

\section{Invalidation and Caching}

The value of a property is \emph{missing} or~$⊥$ in one of three cases: free
ingredients, missing prerequisite, and errors in computation.

\paragraph{Free ingredients.}
An ingredient is said to be \emph{free} if its value was \emph{not
set}. Free properties resemble the empty cells of a spreadsheet---placeholders
to be filled in later. In \cref{figure:rational} both ingredients~\cc{n}
and~\cc{d} are initially free.

\paragraph{Missing prerequisites.}
Values of \emph{free} are missing by definition. Inductively, if a certain
recipe depends on properties whose value is missing, then this recipe's value
is also missing.

Values of all properties in \cref{figure:rational} are missing. To create
concrete non-missing values, assignments to ingredients must take place.
\Cref{figure:rational:1} depicts the state of affairs in
environment~\cc{.Rational} after two operations have taken place:
\begin{enumerate}
  \item assignment of~\cc{0} to property~\cc{d}; followed by,
  \item a read of property~\cc{infinite}.
\end{enumerate}

\begin{figure}
  \caption{\label{figure:rational:1}%
    The prerequisites' graph of \cref{figure:rational},
    after the assignment \cc{d=0} and a read of \cc{infinite}.
  }
  \begin{adjustbox}{max width=\columnwidth}
    \usegdlibrary{layered}

\begin{tikzpicture}[
    shorten >=3pt,
    -stealth,
    p/.style = {% Property
      minimum width=3.5em,
      rounded corners,
      font=\footnotesize,
      rectangle split parts=2,
    },
    f/.style = {p,fill=olive!30}, % value
    v/.style = {p,fill=green!30}, % final 
    r/.style = {p,rectangle split,fill=red!30},   % recipe
]
\def\cc#1{{\scriptsize{#1}}}
  \graph [ 
    branch up,
    layered layout,
    grow'=up,
    level distance=11ex,
    sibling distance=12ex
    ] {% 
    % vim: +,/^\s}$/-!column -t
    d /{\cc{d}  $(\cc{0})$}[v,draw,thick],
    n /{\cc{n}  $(\bot)$}[v],
    r /{\cc{n/d} \nodepart{two} \cc{r} $(\bot)$}[r],
      infinite /{\cc{d == 0} \nodepart{two} \cc{infinite} $(\kk{True})$}[r,draw,thick],
    abs /{\cc{r > 0 \relax? r : -r} \nodepart{two} \cc{abs} $(\bot)$ }[r],
    zero /{\cc{abs == 0} \nodepart{two} \cc{zero} $(\bot)$ }[r],
    positive /{\cc{n < 0 == d < 0} \nodepart{two} \cc{positive} $(\bot)$}[r],
    infinite -> d,
    zero->abs->r,
    r->{n, d},
    positive->{n, d},
    }; %
\path[fill=yellow!30,opacity=0.4,fill=white,] 
    (current bounding box.south west) 
    rectangle 
    (current bounding box.north east); 
\end{tikzpicture}

  \end{adjustbox}
\end{figure}

By employing a visual convention of drawing a border around non-missing values
the figure shows that values of~\cc{d} and \cc{infinite} are non-missing.

To explain, lazy caching semantics means the assignment to~\cc{d} invalidates
the cached value of properties that depend on it directly
or indirectly. Since each recipe in the graph has a directed path
leading to to~\cc{d}, the assignment to~\cc{d} invalidates all recipes.

As a visual hint, invalidated cache values are drawn on a ligther background.
We see that property \cc{infinite} is different than the others in this
respect: the read of of this property triggered computation of recipe
\cc{d==0}, which returned \kk{True}. Storing this returned value in the cache
of \cc{infinite} marks this cache valid.

Even though the cache is invalidated on the other recipes, they stil
yield~$⊥$ \emph{without} recomputation: For example, reading
\cc{positive} immediately returns~$⊥$: Even though~\cc{d} is defined,
computation of the recipe \[
  \cc{n < 0 == d < 0}
\] cannot commence while~\cc{n=$⊥$}.

By the same consideration, property~\cc{r} remains in the missing
prerequisite state. Since~\cc{r} does not change, properties \cc{abs}
and \cc{zero} retain their current~$⊥$ value as well: An attempt to
read these will check that~\cc{⊥} present in the cache still cannot
be computed. If this is the case, the cache is simply marked valid
without changing its value.

\paragraph{Errors.}
The value of a property might be missing even if none of its prerequisites is
missing. This happens if computing the recipe fails, e.g., the \Java evaluation
of the~$λ$-expression throwing an exception, which sets its value to to~$⊥$.

Consider the state of affairs in~\cc{.Rational}, after we applying further:
\begin{enumerate}
  \item assignment of~\cc{1} to property~\cc{d}; followed by,
  \item a read of property~\cc{positive}; and (collaterally),
  \item a read of property~\cc{r}.
\end{enumerate}

The prerequisites' graph after these is depicted in \Cref{figure:rational:2}.

\begin{figure}
  \caption{\label{figure:rational:2}%
    The prerequisites' graph of \cref{figure:rational:1},
    followed by the assignment \cc{n=1} and a read
    of~\cc{positive} and of~\cc{r}.
  }
  \begin{adjustbox}{max width=\columnwidth}
    \usegdlibrary{layered}

\begin{tikzpicture}[
thick,
    shorten >=3pt,
    -stealth,
    p/.style = {% Property
      label=above left:\cc{#1},
      rounded corners,
      font=\footnotesize,
      rectangle split parts=2,
	minimum height=0pt,
	minimum width=0pt,
    },
    f/.style = {fill=olive!30,p=#1}, % cvalue
    v/.style = {fill=Yellow,p=#1,circle,inner sep=2pt,}, % final 
    r/.style = {fill=YellowGreen,
      p=#1,
minimum height=2em,
      trapezium,
      trapezium left angle=130, 
      trapezium right angle=130,
	inner sep=8pt,
% inner sep=0pt,
%  outer sep=0pt,
    },   % recipe
    l/.style = { % label
      ellipse,
      font=\scriptsize,
      fill=ProcessBlue!90,
inner sep=2pt,
    },
    valueIsSet/.style = {% Set 
      draw,
      very thick,
    }, 
    invalidated/.style = {% Set 
fill=ProcessBlue!20
    },
]
\def\cc#1{{\normalsize\texttt{\textbf{#1}}}}
  \graph [ 
     layered layout,
edges=rounded corners,
    level distance=12ex,
    sibling distance=12ex
    ] {% 
    % vim: +,/^\s}$/-!column -t
    d /{\cc{1}}[v=d,valueIsSet],
    n /{$\bot$}[v=n,valueIsSet],
    r /{\cc{n/d}}[r=r],
    infinite /{\cc{d == 0}}[r=infinite],
    abs /{\cc{r > 0 \relax? r : -r}  }[r=abs],
    zero /{\cc{abs == 0}  }[r=zero],
    positive /{\cc{n < 0 == d < 0}}[r=positive],
    infinite <- d,
    zero<-abs<-r,
    r<-{n, d},
    positive<-{n, d},
    }; %
    \node [l,at=(r.south)] {$\bot$};
    \node [l,valueIsSet,at=(infinite.south)] {\scriptsize\texttt{True}};
    \node [l,at=(abs.south),invalidated] {$\bot$};
    \node [l,at=(zero.south),invalidated] {$\bot$};
    \node [l,valueIsSet,at=(positive.south)] {\scriptsize\texttt{True}};
    \path[fill=yellow!30,opacity=0.4,fill=white,] 
    (current bounding box.south west) 
    rectangle 
    (current bounding box.north east); 
\end{tikzpicture}

  \end{adjustbox}
\end{figure}

In the figure, we see that even though there are no more free ingredients the
values of some properties are still missing. To understand why, notice that
assignment \cc{n=1} invalidated all properties except for \cc{infinite}. In
particular, property~\cc{r} is invalidated by the assignment~\cc{n=1}. In
reading~\cc{r}, recipe can be triggered, since both of~\cc{n} and~\cc{d}, the
two prerequisites are no longer free.

The attempted division by zero aborts the triggered recipe. Consequently,
property~\cc{r} retains its previous~$⊥$ value.  A subsequent read of~\cc{r}
will \emph{not} trigger the recipe, unless one or more of~\cc{n} or~\cc{d} was
updated in between the reads. 

The result of~\cc{r} retaining its missing value is that if recipes \cc{abs}
and \cc{zero} are accessed, they will retain their cached missing value as
well. 

The read of \cc{positive} triggers recomputation of of its recipe, and this
recomputation returns \kk{True}.†{Indeed, in the~\cc{.Rational} environment we
have~${1\over0}=+∞$.}

Let the set of all types of \Reap be denoted by~$𝕋$, then
we have
\begin{equation}
  \label{eq:bounds}
  ∀τ∈𝕋 ∙ ⊥≤τ≤⊤.
\end{equation}


\endinput

\section{Emulating Functions, Procedures, and ,Methods}
\begin{reap}
  `'.square {`'
    Real x;
    Real x2() -> x * x;
  }
\end{reap}

Also, change all \cc{¢} to \$.

\begin{reap}
  `'.square {`'
    Real x;
    Real $`\ingore$'() -> x * x;
  }
\end{reap}

the utlimate argument is \_.
\begin{reap}
  `'.square {`'
    Real `\_';
    Real ¢() -> x * x;
  }
\end{reap}

\begin{reap}
.square(2) 
\end{reap}
is short hand for
\begin{reap}
.square.\_=2.() 
\end{reap}
which is short hand for 
\begin{reap}
.square.\_=2.$
\end{reap}

\endinput

Any time you use it, the called would also be throwing an exception.
% \endinput
In i.e., the expression/function
making the
recipe of~$u$, uses~$v$. This uses relationship is implicit: If the
expression/function
functions,
there is a
Type system is~$𝕋$.
and edges are defined by recipes. Names are plain
identifierso called,
properties, carry ordinary identifiers as names.
identifiers.
A \emph{recipe}
is a function attached to a property, defining how the property's value
is to be computed from those of other properties. If the recipe
interdependencies between values
of the properties. The graph is not necessarily connected, and although it may
contain cycles,
Each node in the graph is a typed
\emph{property}. whose nodes are properties.
In this definition of environment~\cc{.pred}
% vim:: /\/\//,/^\s*}/- !awk -F // '{printf "%-20s //%s\n",~$1,~$2}'
\begin{reap}
  `'.Temperature {`'
    Real f0 0;
    Real c()-> 5*(f0-32);
    Real f0 0;
  }
\end{reap}
in the case that no value was assigned to a property
\begin{reap} \end{reap}
The edges in the graph
A \emph{property}
contains a value, which might be computed
is replaced by that of an \emph{environment} defining \emph{properties}, some
of which are adorned by \emph{recipes}. Properties are similar to variables,
except that a property whose value was not explicitly set, may still yield a
value: \Reap's underlying inference engine applies any required recipes to
compute this value from values of other properties.
properties scalar
and,
\kk{String}.
Whenever a property is undefined it is like a function that throws an exception.
\begin{reap}
  `'.power {`'
    Real x;
    Real abs() -> abs(x);
    Boolean positive() -> x > 0;
    Integer n;
    Real ¢() -> exp(n * log(abs)) * (positive ? 1 : -1) || 0;
  }
\end{reap}
\
\begin{reap}
  `'.power {`' // Global math constants
    Real x;
    Integer n;
    Real ¢() ->
  }
\end{reap}
\begin{reap}
  `'.Math {`' // Global math constants
    Real E 2.7182818284590452354;
    Real PI 3.14159265358979323846;
    Real PI2() -> 2 * PI;
  }
\end{reap}
\begin{reap}
  `'.Math.Sin0 {`' // A function to compute`~$\sin(x)$' //
    Real `\_';
    Real positive() -> `\_' > 0 `'? `\_' : -`\_';
    Real cycles() -> (int)(positive() / PI2);
    Real normalized() -> positive - PI2 * cycles;
    Real ¢() -> normalized;
  }
\end{reap}
\begin{reap}
  `'.Sin0 + .Sin1 {`' // A function to compute`~$\sin(x)$' //
    Real `\_' '\^'.¢;
    Real ¢() -> '\_- \_*\_ * \_/6';
  }
\end{reap}
\begin{reap}
  `'.Math.Sin2 {`' // A function to compute'~$\sin(x)$' //
    Real ¢() -> `\^.¢ - power.n=5,x='\_`.¢ /120';
  }
\end{reap}
% vim:: /\/\//,/^\s*}/- !awk -F // '{printf "%-20s //%s\n",~$1,~$2}'
\begin{reap}
  `'.Time {`' // Time environment
    Integer h 9; // Hours, defaults to 9
    Integer m 11; // Minutes, defaults to 11
    Integer s; // Seconds, defaults to `$⊥$'
  }
\end{reap}
Remove the "=" sign in initialization but not assignment.
% vim:: /\/\//,/^\s*}/- !awk -F // '{printf "%-20s //%s\n",~$1,~$2}'
\begin{java}
  class Time extends Object {`'// Time environment transliterated to \Java
    Integer h = 9; // Hours, initially set to `$\cc{9}$'
    Integer m = 11; // Minutes, initially set to `$\cc{11}$'
    Integer s; // Seconds, initially set to `$\kk{null}$'
  }
\end{java}

\section{Methods}

\begin{reap}
  `'.Time.toString {`'
    String ¢ = h + ":" + m + ":" + s;
  }
\end{reap}
reaches fields directly, just like inheritance.
\begin{java}
  class toString extends Time {`' // `\Java' transliteration of~$\cc{Time.toString}$
    String ¢() -> h + ":" + m + ":" + s;
  }
\end{java}
\begin{java}
  class Time {`' // More accurate `\Java' transliteration of~$\cc{Time.toString}$
    public static class toString extends Time {`'
      String ¢() {`'
        h + ":" + m + ":" + s;
      }
    }
  }
\end{java}
\begin{reap}
  `'.Time.AMPM {`'
    Boolean isAM `\^'.h < 12;
    Boolean isPM `'!isAM;
    String ampm isAM `'? "AM" : "PM";
    String h `\^'.h - (isAM `'? 0 : 12);
  }
\end{reap}
\begin{reap}
  `'.Time.fraction {`'
    Real hour 24 * 60;
    Real ¢() -> ((60*h + m) + s)/24./60 ;
  }
\end{reap}
reads as
\begin{java}
  /** `\Java' transliteration of `\cc{.Time.fraction}'
  */
  class fraction extends Time {`'
    double hour() {`' return 24 * 60; }
    double ¢() {`' return ((60*h + m) + s)/hour; }
  }
\end{java}
or so
\begin{java}
  /* more efficient `\Java' transliteration of `\cc{.Time.fraction}'
  **/
  class fraction extends Time {`'
    double ¢() {`' return ((60*h + m) + s)/(double) (24*60) ; }
  }
\end{java}
\begin{java}
  class Time {`'
    static class AMPM extends Time {`' /* `$⋯$ '*/ }
    static class fraction extends Time {`' /* `$⋯$ '*/}
  }
\end{java}
\matteo{read about mixins and traits.start with mixins}
\begin{reap}
  `'.Time.AMPM.fraction;
\end{reap}
\begin{java}
  class Time {`'
    static class AMPM extends Time {`'
      /* `$⋯$ '*/
      static class fraction extends Time {`' /* `$⋯$ '*/}
    }
    static class fraction extends Time {`' /* `$⋯$ '*/}
  }
\end{java}
\begin{reap}
  `'.Time.fraction.AMPM;
\end{reap}
\begin{java}
  class Time {`'
    static class AMPM extends Time {`'
      /** `$⋯$'*/
      static class fraction extends Time {`' /* `$⋯$ '*/}
    }
    static class fraction extends Time {`'
      /** `$⋯$'*/
      static class AMPM extends fraction {`' /* `$⋯$ '*/}
    }
  }
\end{java}
\begin{reap}
  `'.Time.GuessSeconds {`' // A function to set the number of seconds to 0
    Integer s 0;
  }
\end{reap}
\begin{reap}
  `'.Time.fraction {`'
    Real ¢() -> ((60*h + m) + s)/24./60 ;
  }
\end{reap}
To evaluate
\begin{java}
  ((60*h + m) + s)/24./60
\end{java}
replace~$h→9, m→11, s→⊥$
to obtain
{\scriptsize
  \begin{equation}
    \begin{split}
      ¢ & = ((60·9 + 11) + ⊥)/24./60⏎
      & = ((540 + 11) + ⊥)/24./60⏎
      & = (551 + ⊥)/1440. &⏎
      & = ⊥/1440.⏎
      & = ⊥
    \end{split}
    `'.
  \end{equation}
}
\begin{reap}
  `'.fraction.Checked {`'
    Ok ok() ->
    -1 < h && h < 24 &&
    -1 < m && m < 60 &&
    -1 < s && s < 60 ;
    Real ¢()-> ok && `\^'.¢;
  }
\end{reap}
Note the use of \verb+&&+ and \verb+||+. they also deal with undefined.
Different from Java.
\begin{reap}
  `'.Checked.Greeting {`'
    fraction `\^'.¢ || 0.5;
    String name "David";
    String now() ->
    fraction < 0.50 `'? "morning" :
    fraction < 0.75 `'? "day" :
    "evening"
    ;
    ¢() -> "Hi " + name + ", good " + now + " to you.";
  }
\end{reap}
Linear
\begin{reap}
  `'.Hamlet {`'
    Boolean toBe() {`'
      toBe = () -> `'!toBe;
      return true;
    }
  }
\end{reap}
behaves like flip flop or maybe a clock? bad metaphors
Penny pool notation:
\begin{reap}
  `'.Hamlet {`'
    Boolean ¢() {`' ¢ = () -> `'!¢; return false; }
  }
\end{reap}
\begin{reap}
  `'.Ticker {`'
    Integer start 1;
    Integer ¢() {`'
      ¢ = () -> ¢ + 1; // Change gate of cell ¢
      return start;
    };
  }
\end{reap}
\begin{java}
  a() -> a + 1;
\end{java}
means
\begin{java}
  ¢ = () -> ¢ + 1;
\end{java}
\begin{java}
  Integer curr() {`' prev = ¢;}
\end{java}
Is actually
\begin{java}
  Integer curr = () -> {`' prev = ¢;}
\end{java}
\begin{reap}
  `'.Fibonacci {`'
    Integer prev 1;
    Integer curr() {`' prev = ¢;}
  }
\end{reap}
\begin{reap}
  `'.AP.Constants {`' // Constant of Arithmetic Progression
    Real a1;
    Real d;
  }
\end{reap}
\begin{reap}
  `'.AP.nTh {`' // Arithmetic Progression
    Integer ¢() -> (a1,¢=()->¢ + d);
  }
\end{reap}
\begin{reap}
  `'.Fibonacci {`' // Compute the Fibonacci sequence
    Integer a1;
    Integer a2;
    Integer ¢ = {`' a1, a2, ¢ = () -> ¢ + ¢[1] };
  }
\end{reap}
Interesting step wise! Try to do Fibonacci? Try Fibonacci as a function which creates zillion cells?
\begin{reap}
  `'.Constants.AP {`' // Arithmetic Progression
    Integer ¢() -> a1 -> ¢ + d;
  }
\end{reap}
\matteo{read about type Unity which is also OK which is also top in our case}
\begin{reap}
  `'.fraction.Checked {`'
    Ok ok() ->
    // Up casting into type `$⊤ = \kk{Ok}$', of `…'
    // `…' an expression of type `\kk{Boolean}'
    0 < h && h < 24 &&
    0 < m && m < 60 &&
    0 < s && s < 60
    ;
    Real ¢()-> ok && `\^'.¢;
  }
\end{reap}
\begin{reap}
  `'.Brothers {`'
    Integer c, h, g;
  }
\end{reap}
\begin{reap}
  `'.fraction.Checked {`'
    Ok ok() ->
    0 < h && h < 24 &&
    0 < m && m < 60 &&
    0 < s && s < 60 ;
    Real ¢()-> ok && `\^'.¢;
  }
\end{reap}
\begin{reap}
  `'.Brothers {`'
    Integer c, h, g;
  }
\end{reap}
Arrays may be represented as previous versions of a cellimport java.util.function.Function;
\begin{reap}
  class Test {
    public static void main(final String[] args) {
      final int Z = 0;
      final Function<Integer, Function<Integer, Integer>> f = X -> Y -> Z;
      System.out.println(f.apply(0).apply(0));
    }
  }
\end{reap}

In certain occasions, it is useful to define properties 
of the top type, \cc{Ok},
\kk{Character}, \kk{Integer}, \kk{Real},

\begin{align}
  \label{eq:ok}
  ⊤ & = \kk{Ok} = \cc{Void}⏎
  \label{eq:oz}
  ⊥ & = \kk{Oz} = \kk{None} = \cc{@NonNull Void}
\end{align}




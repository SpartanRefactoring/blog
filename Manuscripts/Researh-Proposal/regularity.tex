\label{regularity}
Sometimes, it is not enough to simply judge the complexity of code by a monotonous function.
Just because a piece of code is long, does not mean that it is unreasonably complicated.
The long code might have a well-structured design, or a particular pattern to it that helps readers understand it better.

The idea of measuring for \emph{regularity} in the code was first suggested by Feitelson \& Jbara \cite{Jbara:Feitelson:13}.
In their research, they studied how developers react to regularity.
We wish to expand on this idea by measuring the correlation of regular code to bugs and other metrics.
Using their definition, we will work on the assumption that the more regulated the code is, the more \textit{compressible} it is.
\begin{define}
Let $A$ be a compression algorithm and $c$ a piece of code, then $$\mathrm{regularity}_A(c)\triangleq\frac{\abs{c}}{\abs{A(c)}}.$$
\end{define}
Meaning, the regularity of the code is its compression ratio.